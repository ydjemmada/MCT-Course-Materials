\documentclass[11pt]{article}

    \usepackage[breakable]{tcolorbox}
    \usepackage{parskip} % Stop auto-indenting (to mimic markdown behaviour)
    
    \usepackage{iftex}
    \ifPDFTeX
    	\usepackage[T1]{fontenc}
    	\usepackage{mathpazo}
    \else
    	\usepackage{fontspec}
    \fi

    % Basic figure setup, for now with no caption control since it's done
    % automatically by Pandoc (which extracts ![](path) syntax from Markdown).
    \usepackage{graphicx}
    % Maintain compatibility with old templates. Remove in nbconvert 6.0
    \let\Oldincludegraphics\includegraphics
    % Ensure that by default, figures have no caption (until we provide a
    % proper Figure object with a Caption API and a way to capture that
    % in the conversion process - todo).
    \usepackage{caption}
    \DeclareCaptionFormat{nocaption}{}
    \captionsetup{format=nocaption,aboveskip=0pt,belowskip=0pt}

    \usepackage{float}
    \floatplacement{figure}{H} % forces figures to be placed at the correct location
    \usepackage{xcolor} % Allow colors to be defined
    \usepackage{enumerate} % Needed for markdown enumerations to work
    \usepackage{geometry} % Used to adjust the document margins
    \usepackage{amsmath} % Equations
    \usepackage{amssymb} % Equations
    \usepackage{textcomp} % defines textquotesingle
    % Hack from http://tex.stackexchange.com/a/47451/13684:
    \AtBeginDocument{%
        \def\PYZsq{\textquotesingle}% Upright quotes in Pygmentized code
    }
    \usepackage{upquote} % Upright quotes for verbatim code
    \usepackage{eurosym} % defines \euro
    \usepackage[mathletters]{ucs} % Extended unicode (utf-8) support
    \usepackage{fancyvrb} % verbatim replacement that allows latex
    \usepackage{grffile} % extends the file name processing of package graphics 
                         % to support a larger range
    \makeatletter % fix for old versions of grffile with XeLaTeX
    \@ifpackagelater{grffile}{2019/11/01}
    {
      % Do nothing on new versions
    }
    {
      \def\Gread@@xetex#1{%
        \IfFileExists{"\Gin@base".bb}%
        {\Gread@eps{\Gin@base.bb}}%
        {\Gread@@xetex@aux#1}%
      }
    }
    \makeatother
    \usepackage[Export]{adjustbox} % Used to constrain images to a maximum size
    \adjustboxset{max size={0.9\linewidth}{0.9\paperheight}}

    % The hyperref package gives us a pdf with properly built
    % internal navigation ('pdf bookmarks' for the table of contents,
    % internal cross-reference links, web links for URLs, etc.)
    \usepackage{hyperref}
    % The default LaTeX title has an obnoxious amount of whitespace. By default,
    % titling removes some of it. It also provides customization options.
    \usepackage{titling}
    \usepackage{longtable} % longtable support required by pandoc >1.10
    \usepackage{booktabs}  % table support for pandoc > 1.12.2
    \usepackage[inline]{enumitem} % IRkernel/repr support (it uses the enumerate* environment)
    \usepackage[normalem]{ulem} % ulem is needed to support strikethroughs (\sout)
                                % normalem makes italics be italics, not underlines
    \usepackage{mathrsfs}
    

    
    % Colors for the hyperref package
    \definecolor{urlcolor}{rgb}{0,.145,.698}
    \definecolor{linkcolor}{rgb}{.71,0.21,0.01}
    \definecolor{citecolor}{rgb}{.12,.54,.11}

    % ANSI colors
    \definecolor{ansi-black}{HTML}{3E424D}
    \definecolor{ansi-black-intense}{HTML}{282C36}
    \definecolor{ansi-red}{HTML}{E75C58}
    \definecolor{ansi-red-intense}{HTML}{B22B31}
    \definecolor{ansi-green}{HTML}{00A250}
    \definecolor{ansi-green-intense}{HTML}{007427}
    \definecolor{ansi-yellow}{HTML}{DDB62B}
    \definecolor{ansi-yellow-intense}{HTML}{B27D12}
    \definecolor{ansi-blue}{HTML}{208FFB}
    \definecolor{ansi-blue-intense}{HTML}{0065CA}
    \definecolor{ansi-magenta}{HTML}{D160C4}
    \definecolor{ansi-magenta-intense}{HTML}{A03196}
    \definecolor{ansi-cyan}{HTML}{60C6C8}
    \definecolor{ansi-cyan-intense}{HTML}{258F8F}
    \definecolor{ansi-white}{HTML}{C5C1B4}
    \definecolor{ansi-white-intense}{HTML}{A1A6B2}
    \definecolor{ansi-default-inverse-fg}{HTML}{FFFFFF}
    \definecolor{ansi-default-inverse-bg}{HTML}{000000}

    % common color for the border for error outputs.
    \definecolor{outerrorbackground}{HTML}{FFDFDF}

    % commands and environments needed by pandoc snippets
    % extracted from the output of `pandoc -s`
    \providecommand{\tightlist}{%
      \setlength{\itemsep}{0pt}\setlength{\parskip}{0pt}}
    \DefineVerbatimEnvironment{Highlighting}{Verbatim}{commandchars=\\\{\}}
    % Add ',fontsize=\small' for more characters per line
    \newenvironment{Shaded}{}{}
    \newcommand{\KeywordTok}[1]{\textcolor[rgb]{0.00,0.44,0.13}{\textbf{{#1}}}}
    \newcommand{\DataTypeTok}[1]{\textcolor[rgb]{0.56,0.13,0.00}{{#1}}}
    \newcommand{\DecValTok}[1]{\textcolor[rgb]{0.25,0.63,0.44}{{#1}}}
    \newcommand{\BaseNTok}[1]{\textcolor[rgb]{0.25,0.63,0.44}{{#1}}}
    \newcommand{\FloatTok}[1]{\textcolor[rgb]{0.25,0.63,0.44}{{#1}}}
    \newcommand{\CharTok}[1]{\textcolor[rgb]{0.25,0.44,0.63}{{#1}}}
    \newcommand{\StringTok}[1]{\textcolor[rgb]{0.25,0.44,0.63}{{#1}}}
    \newcommand{\CommentTok}[1]{\textcolor[rgb]{0.38,0.63,0.69}{\textit{{#1}}}}
    \newcommand{\OtherTok}[1]{\textcolor[rgb]{0.00,0.44,0.13}{{#1}}}
    \newcommand{\AlertTok}[1]{\textcolor[rgb]{1.00,0.00,0.00}{\textbf{{#1}}}}
    \newcommand{\FunctionTok}[1]{\textcolor[rgb]{0.02,0.16,0.49}{{#1}}}
    \newcommand{\RegionMarkerTok}[1]{{#1}}
    \newcommand{\ErrorTok}[1]{\textcolor[rgb]{1.00,0.00,0.00}{\textbf{{#1}}}}
    \newcommand{\NormalTok}[1]{{#1}}
    
    % Additional commands for more recent versions of Pandoc
    \newcommand{\ConstantTok}[1]{\textcolor[rgb]{0.53,0.00,0.00}{{#1}}}
    \newcommand{\SpecialCharTok}[1]{\textcolor[rgb]{0.25,0.44,0.63}{{#1}}}
    \newcommand{\VerbatimStringTok}[1]{\textcolor[rgb]{0.25,0.44,0.63}{{#1}}}
    \newcommand{\SpecialStringTok}[1]{\textcolor[rgb]{0.73,0.40,0.53}{{#1}}}
    \newcommand{\ImportTok}[1]{{#1}}
    \newcommand{\DocumentationTok}[1]{\textcolor[rgb]{0.73,0.13,0.13}{\textit{{#1}}}}
    \newcommand{\AnnotationTok}[1]{\textcolor[rgb]{0.38,0.63,0.69}{\textbf{\textit{{#1}}}}}
    \newcommand{\CommentVarTok}[1]{\textcolor[rgb]{0.38,0.63,0.69}{\textbf{\textit{{#1}}}}}
    \newcommand{\VariableTok}[1]{\textcolor[rgb]{0.10,0.09,0.49}{{#1}}}
    \newcommand{\ControlFlowTok}[1]{\textcolor[rgb]{0.00,0.44,0.13}{\textbf{{#1}}}}
    \newcommand{\OperatorTok}[1]{\textcolor[rgb]{0.40,0.40,0.40}{{#1}}}
    \newcommand{\BuiltInTok}[1]{{#1}}
    \newcommand{\ExtensionTok}[1]{{#1}}
    \newcommand{\PreprocessorTok}[1]{\textcolor[rgb]{0.74,0.48,0.00}{{#1}}}
    \newcommand{\AttributeTok}[1]{\textcolor[rgb]{0.49,0.56,0.16}{{#1}}}
    \newcommand{\InformationTok}[1]{\textcolor[rgb]{0.38,0.63,0.69}{\textbf{\textit{{#1}}}}}
    \newcommand{\WarningTok}[1]{\textcolor[rgb]{0.38,0.63,0.69}{\textbf{\textit{{#1}}}}}
    
    
    % Define a nice break command that doesn't care if a line doesn't already
    % exist.
    \def\br{\hspace*{\fill} \\* }
    % Math Jax compatibility definitions
    \def\gt{>}
    \def\lt{<}
    \let\Oldtex\TeX
    \let\Oldlatex\LaTeX
    \renewcommand{\TeX}{\textrm{\Oldtex}}
    \renewcommand{\LaTeX}{\textrm{\Oldlatex}}
    % Document parameters
    % Document title
    \title{Lecture 0 - Part 2: Data structures}
    \date{}
    
    
    
    
% Pygments definitions
\makeatletter
\def\PY@reset{\let\PY@it=\relax \let\PY@bf=\relax%
    \let\PY@ul=\relax \let\PY@tc=\relax%
    \let\PY@bc=\relax \let\PY@ff=\relax}
\def\PY@tok#1{\csname PY@tok@#1\endcsname}
\def\PY@toks#1+{\ifx\relax#1\empty\else%
    \PY@tok{#1}\expandafter\PY@toks\fi}
\def\PY@do#1{\PY@bc{\PY@tc{\PY@ul{%
    \PY@it{\PY@bf{\PY@ff{#1}}}}}}}
\def\PY#1#2{\PY@reset\PY@toks#1+\relax+\PY@do{#2}}

\@namedef{PY@tok@w}{\def\PY@tc##1{\textcolor[rgb]{0.73,0.73,0.73}{##1}}}
\@namedef{PY@tok@c}{\let\PY@it=\textit\def\PY@tc##1{\textcolor[rgb]{0.25,0.50,0.50}{##1}}}
\@namedef{PY@tok@cp}{\def\PY@tc##1{\textcolor[rgb]{0.74,0.48,0.00}{##1}}}
\@namedef{PY@tok@k}{\let\PY@bf=\textbf\def\PY@tc##1{\textcolor[rgb]{0.00,0.50,0.00}{##1}}}
\@namedef{PY@tok@kp}{\def\PY@tc##1{\textcolor[rgb]{0.00,0.50,0.00}{##1}}}
\@namedef{PY@tok@kt}{\def\PY@tc##1{\textcolor[rgb]{0.69,0.00,0.25}{##1}}}
\@namedef{PY@tok@o}{\def\PY@tc##1{\textcolor[rgb]{0.40,0.40,0.40}{##1}}}
\@namedef{PY@tok@ow}{\let\PY@bf=\textbf\def\PY@tc##1{\textcolor[rgb]{0.67,0.13,1.00}{##1}}}
\@namedef{PY@tok@nb}{\def\PY@tc##1{\textcolor[rgb]{0.00,0.50,0.00}{##1}}}
\@namedef{PY@tok@nf}{\def\PY@tc##1{\textcolor[rgb]{0.00,0.00,1.00}{##1}}}
\@namedef{PY@tok@nc}{\let\PY@bf=\textbf\def\PY@tc##1{\textcolor[rgb]{0.00,0.00,1.00}{##1}}}
\@namedef{PY@tok@nn}{\let\PY@bf=\textbf\def\PY@tc##1{\textcolor[rgb]{0.00,0.00,1.00}{##1}}}
\@namedef{PY@tok@ne}{\let\PY@bf=\textbf\def\PY@tc##1{\textcolor[rgb]{0.82,0.25,0.23}{##1}}}
\@namedef{PY@tok@nv}{\def\PY@tc##1{\textcolor[rgb]{0.10,0.09,0.49}{##1}}}
\@namedef{PY@tok@no}{\def\PY@tc##1{\textcolor[rgb]{0.53,0.00,0.00}{##1}}}
\@namedef{PY@tok@nl}{\def\PY@tc##1{\textcolor[rgb]{0.63,0.63,0.00}{##1}}}
\@namedef{PY@tok@ni}{\let\PY@bf=\textbf\def\PY@tc##1{\textcolor[rgb]{0.60,0.60,0.60}{##1}}}
\@namedef{PY@tok@na}{\def\PY@tc##1{\textcolor[rgb]{0.49,0.56,0.16}{##1}}}
\@namedef{PY@tok@nt}{\let\PY@bf=\textbf\def\PY@tc##1{\textcolor[rgb]{0.00,0.50,0.00}{##1}}}
\@namedef{PY@tok@nd}{\def\PY@tc##1{\textcolor[rgb]{0.67,0.13,1.00}{##1}}}
\@namedef{PY@tok@s}{\def\PY@tc##1{\textcolor[rgb]{0.73,0.13,0.13}{##1}}}
\@namedef{PY@tok@sd}{\let\PY@it=\textit\def\PY@tc##1{\textcolor[rgb]{0.73,0.13,0.13}{##1}}}
\@namedef{PY@tok@si}{\let\PY@bf=\textbf\def\PY@tc##1{\textcolor[rgb]{0.73,0.40,0.53}{##1}}}
\@namedef{PY@tok@se}{\let\PY@bf=\textbf\def\PY@tc##1{\textcolor[rgb]{0.73,0.40,0.13}{##1}}}
\@namedef{PY@tok@sr}{\def\PY@tc##1{\textcolor[rgb]{0.73,0.40,0.53}{##1}}}
\@namedef{PY@tok@ss}{\def\PY@tc##1{\textcolor[rgb]{0.10,0.09,0.49}{##1}}}
\@namedef{PY@tok@sx}{\def\PY@tc##1{\textcolor[rgb]{0.00,0.50,0.00}{##1}}}
\@namedef{PY@tok@m}{\def\PY@tc##1{\textcolor[rgb]{0.40,0.40,0.40}{##1}}}
\@namedef{PY@tok@gh}{\let\PY@bf=\textbf\def\PY@tc##1{\textcolor[rgb]{0.00,0.00,0.50}{##1}}}
\@namedef{PY@tok@gu}{\let\PY@bf=\textbf\def\PY@tc##1{\textcolor[rgb]{0.50,0.00,0.50}{##1}}}
\@namedef{PY@tok@gd}{\def\PY@tc##1{\textcolor[rgb]{0.63,0.00,0.00}{##1}}}
\@namedef{PY@tok@gi}{\def\PY@tc##1{\textcolor[rgb]{0.00,0.63,0.00}{##1}}}
\@namedef{PY@tok@gr}{\def\PY@tc##1{\textcolor[rgb]{1.00,0.00,0.00}{##1}}}
\@namedef{PY@tok@ge}{\let\PY@it=\textit}
\@namedef{PY@tok@gs}{\let\PY@bf=\textbf}
\@namedef{PY@tok@gp}{\let\PY@bf=\textbf\def\PY@tc##1{\textcolor[rgb]{0.00,0.00,0.50}{##1}}}
\@namedef{PY@tok@go}{\def\PY@tc##1{\textcolor[rgb]{0.53,0.53,0.53}{##1}}}
\@namedef{PY@tok@gt}{\def\PY@tc##1{\textcolor[rgb]{0.00,0.27,0.87}{##1}}}
\@namedef{PY@tok@err}{\def\PY@bc##1{{\setlength{\fboxsep}{\string -\fboxrule}\fcolorbox[rgb]{1.00,0.00,0.00}{1,1,1}{\strut ##1}}}}
\@namedef{PY@tok@kc}{\let\PY@bf=\textbf\def\PY@tc##1{\textcolor[rgb]{0.00,0.50,0.00}{##1}}}
\@namedef{PY@tok@kd}{\let\PY@bf=\textbf\def\PY@tc##1{\textcolor[rgb]{0.00,0.50,0.00}{##1}}}
\@namedef{PY@tok@kn}{\let\PY@bf=\textbf\def\PY@tc##1{\textcolor[rgb]{0.00,0.50,0.00}{##1}}}
\@namedef{PY@tok@kr}{\let\PY@bf=\textbf\def\PY@tc##1{\textcolor[rgb]{0.00,0.50,0.00}{##1}}}
\@namedef{PY@tok@bp}{\def\PY@tc##1{\textcolor[rgb]{0.00,0.50,0.00}{##1}}}
\@namedef{PY@tok@fm}{\def\PY@tc##1{\textcolor[rgb]{0.00,0.00,1.00}{##1}}}
\@namedef{PY@tok@vc}{\def\PY@tc##1{\textcolor[rgb]{0.10,0.09,0.49}{##1}}}
\@namedef{PY@tok@vg}{\def\PY@tc##1{\textcolor[rgb]{0.10,0.09,0.49}{##1}}}
\@namedef{PY@tok@vi}{\def\PY@tc##1{\textcolor[rgb]{0.10,0.09,0.49}{##1}}}
\@namedef{PY@tok@vm}{\def\PY@tc##1{\textcolor[rgb]{0.10,0.09,0.49}{##1}}}
\@namedef{PY@tok@sa}{\def\PY@tc##1{\textcolor[rgb]{0.73,0.13,0.13}{##1}}}
\@namedef{PY@tok@sb}{\def\PY@tc##1{\textcolor[rgb]{0.73,0.13,0.13}{##1}}}
\@namedef{PY@tok@sc}{\def\PY@tc##1{\textcolor[rgb]{0.73,0.13,0.13}{##1}}}
\@namedef{PY@tok@dl}{\def\PY@tc##1{\textcolor[rgb]{0.73,0.13,0.13}{##1}}}
\@namedef{PY@tok@s2}{\def\PY@tc##1{\textcolor[rgb]{0.73,0.13,0.13}{##1}}}
\@namedef{PY@tok@sh}{\def\PY@tc##1{\textcolor[rgb]{0.73,0.13,0.13}{##1}}}
\@namedef{PY@tok@s1}{\def\PY@tc##1{\textcolor[rgb]{0.73,0.13,0.13}{##1}}}
\@namedef{PY@tok@mb}{\def\PY@tc##1{\textcolor[rgb]{0.40,0.40,0.40}{##1}}}
\@namedef{PY@tok@mf}{\def\PY@tc##1{\textcolor[rgb]{0.40,0.40,0.40}{##1}}}
\@namedef{PY@tok@mh}{\def\PY@tc##1{\textcolor[rgb]{0.40,0.40,0.40}{##1}}}
\@namedef{PY@tok@mi}{\def\PY@tc##1{\textcolor[rgb]{0.40,0.40,0.40}{##1}}}
\@namedef{PY@tok@il}{\def\PY@tc##1{\textcolor[rgb]{0.40,0.40,0.40}{##1}}}
\@namedef{PY@tok@mo}{\def\PY@tc##1{\textcolor[rgb]{0.40,0.40,0.40}{##1}}}
\@namedef{PY@tok@ch}{\let\PY@it=\textit\def\PY@tc##1{\textcolor[rgb]{0.25,0.50,0.50}{##1}}}
\@namedef{PY@tok@cm}{\let\PY@it=\textit\def\PY@tc##1{\textcolor[rgb]{0.25,0.50,0.50}{##1}}}
\@namedef{PY@tok@cpf}{\let\PY@it=\textit\def\PY@tc##1{\textcolor[rgb]{0.25,0.50,0.50}{##1}}}
\@namedef{PY@tok@c1}{\let\PY@it=\textit\def\PY@tc##1{\textcolor[rgb]{0.25,0.50,0.50}{##1}}}
\@namedef{PY@tok@cs}{\let\PY@it=\textit\def\PY@tc##1{\textcolor[rgb]{0.25,0.50,0.50}{##1}}}

\def\PYZbs{\char`\\}
\def\PYZus{\char`\_}
\def\PYZob{\char`\{}
\def\PYZcb{\char`\}}
\def\PYZca{\char`\^}
\def\PYZam{\char`\&}
\def\PYZlt{\char`\<}
\def\PYZgt{\char`\>}
\def\PYZsh{\char`\#}
\def\PYZpc{\char`\%}
\def\PYZdl{\char`\$}
\def\PYZhy{\char`\-}
\def\PYZsq{\char`\'}
\def\PYZdq{\char`\"}
\def\PYZti{\char`\~}
% for compatibility with earlier versions
\def\PYZat{@}
\def\PYZlb{[}
\def\PYZrb{]}
\makeatother


    % For linebreaks inside Verbatim environment from package fancyvrb. 
    \makeatletter
        \newbox\Wrappedcontinuationbox 
        \newbox\Wrappedvisiblespacebox 
        \newcommand*\Wrappedvisiblespace {\textcolor{red}{\textvisiblespace}} 
        \newcommand*\Wrappedcontinuationsymbol {\textcolor{red}{\llap{\tiny$\m@th\hookrightarrow$}}} 
        \newcommand*\Wrappedcontinuationindent {3ex } 
        \newcommand*\Wrappedafterbreak {\kern\Wrappedcontinuationindent\copy\Wrappedcontinuationbox} 
        % Take advantage of the already applied Pygments mark-up to insert 
        % potential linebreaks for TeX processing. 
        %        {, <, #, %, $, ' and ": go to next line. 
        %        _, }, ^, &, >, - and ~: stay at end of broken line. 
        % Use of \textquotesingle for straight quote. 
        \newcommand*\Wrappedbreaksatspecials {% 
            \def\PYGZus{\discretionary{\char`\_}{\Wrappedafterbreak}{\char`\_}}% 
            \def\PYGZob{\discretionary{}{\Wrappedafterbreak\char`\{}{\char`\{}}% 
            \def\PYGZcb{\discretionary{\char`\}}{\Wrappedafterbreak}{\char`\}}}% 
            \def\PYGZca{\discretionary{\char`\^}{\Wrappedafterbreak}{\char`\^}}% 
            \def\PYGZam{\discretionary{\char`\&}{\Wrappedafterbreak}{\char`\&}}% 
            \def\PYGZlt{\discretionary{}{\Wrappedafterbreak\char`\<}{\char`\<}}% 
            \def\PYGZgt{\discretionary{\char`\>}{\Wrappedafterbreak}{\char`\>}}% 
            \def\PYGZsh{\discretionary{}{\Wrappedafterbreak\char`\#}{\char`\#}}% 
            \def\PYGZpc{\discretionary{}{\Wrappedafterbreak\char`\%}{\char`\%}}% 
            \def\PYGZdl{\discretionary{}{\Wrappedafterbreak\char`\$}{\char`\$}}% 
            \def\PYGZhy{\discretionary{\char`\-}{\Wrappedafterbreak}{\char`\-}}% 
            \def\PYGZsq{\discretionary{}{\Wrappedafterbreak\textquotesingle}{\textquotesingle}}% 
            \def\PYGZdq{\discretionary{}{\Wrappedafterbreak\char`\"}{\char`\"}}% 
            \def\PYGZti{\discretionary{\char`\~}{\Wrappedafterbreak}{\char`\~}}% 
        } 
        % Some characters . , ; ? ! / are not pygmentized. 
        % This macro makes them "active" and they will insert potential linebreaks 
        \newcommand*\Wrappedbreaksatpunct {% 
            \lccode`\~`\.\lowercase{\def~}{\discretionary{\hbox{\char`\.}}{\Wrappedafterbreak}{\hbox{\char`\.}}}% 
            \lccode`\~`\,\lowercase{\def~}{\discretionary{\hbox{\char`\,}}{\Wrappedafterbreak}{\hbox{\char`\,}}}% 
            \lccode`\~`\;\lowercase{\def~}{\discretionary{\hbox{\char`\;}}{\Wrappedafterbreak}{\hbox{\char`\;}}}% 
            \lccode`\~`\:\lowercase{\def~}{\discretionary{\hbox{\char`\:}}{\Wrappedafterbreak}{\hbox{\char`\:}}}% 
            \lccode`\~`\?\lowercase{\def~}{\discretionary{\hbox{\char`\?}}{\Wrappedafterbreak}{\hbox{\char`\?}}}% 
            \lccode`\~`\!\lowercase{\def~}{\discretionary{\hbox{\char`\!}}{\Wrappedafterbreak}{\hbox{\char`\!}}}% 
            \lccode`\~`\/\lowercase{\def~}{\discretionary{\hbox{\char`\/}}{\Wrappedafterbreak}{\hbox{\char`\/}}}% 
            \catcode`\.\active
            \catcode`\,\active 
            \catcode`\;\active
            \catcode`\:\active
            \catcode`\?\active
            \catcode`\!\active
            \catcode`\/\active 
            \lccode`\~`\~ 	
        }
    \makeatother

    \let\OriginalVerbatim=\Verbatim
    \makeatletter
    \renewcommand{\Verbatim}[1][1]{%
        %\parskip\z@skip
        \sbox\Wrappedcontinuationbox {\Wrappedcontinuationsymbol}%
        \sbox\Wrappedvisiblespacebox {\FV@SetupFont\Wrappedvisiblespace}%
        \def\FancyVerbFormatLine ##1{\hsize\linewidth
            \vtop{\raggedright\hyphenpenalty\z@\exhyphenpenalty\z@
                \doublehyphendemerits\z@\finalhyphendemerits\z@
                \strut ##1\strut}%
        }%
        % If the linebreak is at a space, the latter will be displayed as visible
        % space at end of first line, and a continuation symbol starts next line.
        % Stretch/shrink are however usually zero for typewriter font.
        \def\FV@Space {%
            \nobreak\hskip\z@ plus\fontdimen3\font minus\fontdimen4\font
            \discretionary{\copy\Wrappedvisiblespacebox}{\Wrappedafterbreak}
            {\kern\fontdimen2\font}%
        }%
        
        % Allow breaks at special characters using \PYG... macros.
        \Wrappedbreaksatspecials
        % Breaks at punctuation characters . , ; ? ! and / need catcode=\active 	
        \OriginalVerbatim[#1,codes*=\Wrappedbreaksatpunct]%
    }
    \makeatother

    % Exact colors from NB
    \definecolor{incolor}{HTML}{303F9F}
    \definecolor{outcolor}{HTML}{D84315}
    \definecolor{cellborder}{HTML}{CFCFCF}
    \definecolor{cellbackground}{HTML}{F7F7F7}
    
    % prompt
    \makeatletter
    \newcommand{\boxspacing}{\kern\kvtcb@left@rule\kern\kvtcb@boxsep}
    \makeatother
    \newcommand{\prompt}[4]{
        {\ttfamily\llap{{\color{#2}[#3]:\hspace{3pt}#4}}\vspace{-\baselineskip}}
    }
    

    
    % Prevent overflowing lines due to hard-to-break entities
    \sloppy 
    % Setup hyperref package
    \hypersetup{
      breaklinks=true,  % so long urls are correctly broken across lines
      colorlinks=true,
      urlcolor=urlcolor,
      linkcolor=linkcolor,
      citecolor=citecolor,
      }
    % Slightly bigger margins than the latex defaults
    
    \geometry{verbose,tmargin=1in,bmargin=1in,lmargin=1in,rmargin=1in}
    
    

\begin{document}
    
    \maketitle
    
    

    
    \hypertarget{data-structures}{%
\section{Data Structures}\label{data-structures}}

\begin{itemize}
\tightlist
\item
  Real-world problems often require complex \textbf{data structures} to
  solve it.
\item
  In programming, a data structure is any specific object used for
  collecting and organizing data.
\item
  The four main types of data structures in Python are \textbf{lists,
  tuples, sets,} and \textbf{dictionaries}.
\item
  Lists, tuples, sets, and dictionaries are all \textbf{iterable
  objects}, meaning they can be looped over and their values accessed
  one at a time, permitting it to be iterated over in a
  \textbf{for-loop}.
\end{itemize}

\hypertarget{mutable-vs-immutable}{%
\section{Mutable vs Immutable}\label{mutable-vs-immutable}}

\begin{itemize}
\tightlist
\item
  In Python, objects can be either mutable or immutable.
\item
  Immutable objects are objects whose values cannot be changed after
  they are created. Examples of immutable objects include integers,
  floating-point numbers, tuples, and strings.
\item
  Mutable objects are objects whose values can be changed after they are
  created. Examples of mutable objects include lists, sets, and
  dictionaries.
\item
  When a mutable object is modified, it remains the same object in
  memory, but its value is changed but when an immutable object is
  ``modified,'' a new object is created with the new value, and the
  original object remains unchanged.
\item
  Immutable objects are generally considered to be more ``safe'' because
  they cannot be accidentally changed, while mutable objects can lead to
  unexpected behavior. However, mutable objects are more flexible and
  can be more efficient in certain situations because they can be
  modified in place rather than requiring the creation of a new object.
\end{itemize}

    \begin{tcolorbox}[breakable, size=fbox, boxrule=1pt, pad at break*=1mm,colback=cellbackground, colframe=cellborder]
\prompt{In}{incolor}{ }{\boxspacing}
\begin{Verbatim}[commandchars=\\\{\}]
\PY{k+kn}{import} \PY{n+nn}{ctypes}
\PY{c+c1}{\PYZsh{}Lists are iterable objects}
\PY{n}{my\PYZus{}list}\PY{o}{=}\PY{p}{[}\PY{l+s+s2}{\PYZdq{}}\PY{l+s+s2}{Ali}\PY{l+s+s2}{\PYZdq{}}\PY{p}{,}\PY{l+s+s2}{\PYZdq{}}\PY{l+s+s2}{Python}\PY{l+s+s2}{\PYZdq{}}\PY{p}{,}\PY{l+m+mf}{15.5}\PY{p}{,}\PY{l+m+mi}{2}\PY{p}{]}\PY{p}{;}
\PY{k}{for} \PY{n}{item} \PY{o+ow}{in} \PY{n}{my\PYZus{}list}\PY{p}{:}
    \PY{n+nb}{print}\PY{p}{(}\PY{n}{item}\PY{p}{)}
\PY{c+c1}{\PYZsh{}Lists are mutable}
\PY{n+nb}{print}\PY{p}{(}\PY{l+s+s2}{\PYZdq{}}\PY{l+s+s2}{The id of my\PYZus{}list is }\PY{l+s+s2}{\PYZdq{}}\PY{p}{,}\PY{n+nb}{id}\PY{p}{(}\PY{n}{my\PYZus{}list}\PY{p}{)}\PY{p}{)}
\PY{n}{my\PYZus{}list}\PY{p}{[}\PY{l+m+mi}{0}\PY{p}{]}\PY{o}{=}\PY{l+s+s2}{\PYZdq{}}\PY{l+s+s2}{Omar}\PY{l+s+s2}{\PYZdq{}}
\PY{n+nb}{print}\PY{p}{(}\PY{n}{my\PYZus{}list}\PY{p}{)}
\PY{n+nb}{print}\PY{p}{(}\PY{l+s+s2}{\PYZdq{}}\PY{l+s+s2}{The id of my\PYZus{}list is }\PY{l+s+s2}{\PYZdq{}}\PY{p}{,}\PY{n+nb}{id}\PY{p}{(}\PY{n}{my\PYZus{}list}\PY{p}{)}\PY{p}{)}
\PY{c+c1}{\PYZsh{}integers are immutable}
\PY{n}{x}\PY{o}{=}\PY{l+m+mi}{5}
\PY{n+nb}{print}\PY{p}{(}\PY{l+s+s2}{\PYZdq{}}\PY{l+s+s2}{The id of x is }\PY{l+s+s2}{\PYZdq{}}\PY{p}{,}\PY{n+nb}{id}\PY{p}{(}\PY{n}{x}\PY{p}{)}\PY{p}{)}
\PY{n}{y}\PY{o}{=}\PY{n}{x}
\PY{n+nb}{print}\PY{p}{(}\PY{l+s+s2}{\PYZdq{}}\PY{l+s+s2}{The id of y is }\PY{l+s+s2}{\PYZdq{}}\PY{p}{,}\PY{n+nb}{id}\PY{p}{(}\PY{n}{y}\PY{p}{)}\PY{p}{)}
\PY{n}{x}\PY{o}{=}\PY{n}{x}\PY{o}{+}\PY{l+m+mi}{1}
\PY{n+nb}{print}\PY{p}{(}\PY{l+s+s2}{\PYZdq{}}\PY{l+s+s2}{The id of x is }\PY{l+s+s2}{\PYZdq{}}\PY{p}{,}\PY{n+nb}{id}\PY{p}{(}\PY{n}{x}\PY{p}{)}\PY{p}{)}
\PY{n+nb}{print}\PY{p}{(}\PY{n}{ctypes}\PY{o}{.}\PY{n}{cast}\PY{p}{(}\PY{n+nb}{id}\PY{p}{(}\PY{n}{y}\PY{p}{)}\PY{p}{,} \PY{n}{ctypes}\PY{o}{.}\PY{n}{py\PYZus{}object}\PY{p}{)}\PY{o}{.}\PY{n}{value}\PY{p}{)}
\end{Verbatim}
\end{tcolorbox}

    \hypertarget{lists}{%
\subsection{1. Lists}\label{lists}}

\hypertarget{what-are-lists}{%
\subsubsection{What are lists?}\label{what-are-lists}}

\begin{itemize}
\item
  Lists are a type of ordered collection in Python that allow you to
  store a collection of values and you can access them by their index
  position in the list.
\item
  Lists are mutable, meaning you can add, remove, or modify elements
  after the list has been created.
\item
  Lists are dynamically sized, which means they can grow or shrink as
  needed to accommodate new elements or remove existing ones.
\item
  Lists can contain elements of different types, including numbers,
  strings, and even other lists.
\item
  Lists are iterable objects, meaning you can loop over each element in
  the list using a for loop or other iteration methods.
\end{itemize}

\hypertarget{create-a-list-in-python}{%
\subsubsection{Create a list in Python:}\label{create-a-list-in-python}}

\begin{itemize}
\tightlist
\item
  To create a list in Python, we use square brackets and separate the
  elements with commas. For example:
  \texttt{my\_list\ =\ {[}1,\ 2,\ 3,\ \textquotesingle{}hello\textquotesingle{},\ True{]}}.
\end{itemize}

\hypertarget{indexing-and-slicing}{%
\subsubsection{Indexing and slicing:}\label{indexing-and-slicing}}

\begin{itemize}
\item
  Lists are indexed starting from 0, so the first element of a list is
  at index 0, the second at index 1, and so on.
\item
  To access a specific element in a list, we can use its index with
  square brackets. For example: \texttt{my\_list{[}0{]}} would return
  the first element of the list, which is 1.
\item
  We can also use slicing to extract a portion of the list. Slicing is
  done with the colon \texttt{:} symbol. For example:
  \texttt{my\_list{[}1:3{]}} would return a new list containing the
  second and third elements of \texttt{my\_list}.
\end{itemize}

\hypertarget{addchange-list-elements}{%
\subsubsection{Add/Change List
Elements:}\label{addchange-list-elements}}

\begin{itemize}
\tightlist
\item
  We can add an element to a list using the \texttt{append()} method and
  \texttt{extend()} to add a range of items. For example:
  \texttt{my\_list.append(4)} would add the number 4 to the end of the
  list.
\item
  We can also change the value of an existing element in a list by
  assigning a new value to its index. For example:
  \texttt{my\_list{[}3{]}\ =\ \textquotesingle{}world\textquotesingle{}}
  would change the fourth element of the list to `world'.
\item
  We can concatenate two lists using the operator \texttt{+} and
  duplicate (repeat) a list using the operator \texttt{*}.

  \begin{itemize}
  \tightlist
  \item
    We can insert an item at a desirable location using the function
    \texttt{insert(\textless{}position\textgreater{},\textless{}item\textgreater{})}.
  \end{itemize}
\end{itemize}

\hypertarget{delete-list-elements}{%
\subsubsection{Delete List Elements:}\label{delete-list-elements}}

\begin{itemize}
\item
  We can delete elements from a list using the del keyword followed by
  the index of the element we want to delete. For example:
  \texttt{del\ my\_list{[}2{]}} would delete the third element from the
  list.
\item
  We can also use the \texttt{remove()} method to remove the first
  occurrence of a specific element. For example:
  \texttt{my\_list.remove(\textquotesingle{}hello\textquotesingle{})}
  would remove the first occurrence of the string `hello' from the list.
\item
  There are two other methods,
  \texttt{pop({[}\textless{}index\textgreater{}{]})} and
  \texttt{clear()} to remove an element of given index (by default the
  last one) and all elements respectively.
\end{itemize}

\hypertarget{python-list-methods}{%
\subsubsection{Python List Methods:}\label{python-list-methods}}

Python provides many \texttt{built-in} methods that can be used with
lists. Some examples include: - \texttt{index(item)}: gives the index of
first occurrence of \texttt{item}. - \texttt{copy()}: returns a copy of
the list. - \texttt{count()}: returns the number of times a specific
element appears in the list - \texttt{sort()}: sorts the list in
ascending order - \texttt{reverse()}: reverses the order of the elements
in the list.

    \begin{tcolorbox}[breakable, size=fbox, boxrule=1pt, pad at break*=1mm,colback=cellbackground, colframe=cellborder]
\prompt{In}{incolor}{ }{\boxspacing}
\begin{Verbatim}[commandchars=\\\{\}]
\PY{c+c1}{\PYZsh{} Create a list of integers}
\PY{n+nb}{print}\PY{p}{(}\PY{l+s+s2}{\PYZdq{}}\PY{l+s+s2}{\PYZsh{} Create a list of integers}\PY{l+s+s2}{\PYZdq{}}\PY{p}{)}
\PY{n}{my\PYZus{}list} \PY{o}{=} \PY{p}{[}\PY{l+m+mi}{1}\PY{p}{,} \PY{l+m+mi}{2}\PY{p}{,} \PY{l+m+mi}{3}\PY{p}{,} \PY{l+m+mi}{4}\PY{p}{,} \PY{l+m+mi}{5}\PY{p}{]}
\PY{n+nb}{print}\PY{p}{(}\PY{n}{my\PYZus{}list}\PY{p}{)}
\PY{c+c1}{\PYZsh{} Indexing and Slicing}
\PY{n+nb}{print}\PY{p}{(}\PY{l+s+s2}{\PYZdq{}}\PY{l+s+s2}{\PYZsh{} Indexing and Slicing}\PY{l+s+s2}{\PYZdq{}}\PY{p}{)}
\PY{n+nb}{print}\PY{p}{(}\PY{n}{my\PYZus{}list}\PY{p}{[}\PY{l+m+mi}{0}\PY{p}{]}\PY{p}{)}    \PY{c+c1}{\PYZsh{} Output: 1}
\PY{n+nb}{print}\PY{p}{(}\PY{n}{my\PYZus{}list}\PY{p}{[}\PY{l+m+mi}{1}\PY{p}{:}\PY{l+m+mi}{3}\PY{p}{]}\PY{p}{)}  \PY{c+c1}{\PYZsh{} Output: [2, 3]}
\PY{n+nb}{print}\PY{p}{(}\PY{n}{my\PYZus{}list}\PY{p}{[}\PY{o}{\PYZhy{}}\PY{l+m+mi}{1}\PY{p}{]}\PY{p}{)}   \PY{c+c1}{\PYZsh{} Output: 5}

\PY{c+c1}{\PYZsh{} Add/Change List Elements}
\PY{n+nb}{print}\PY{p}{(}\PY{l+s+s2}{\PYZdq{}}\PY{l+s+s2}{\PYZsh{} Add/Change List Elements}\PY{l+s+s2}{\PYZdq{}}\PY{p}{)}
\PY{n}{my\PYZus{}list}\PY{p}{[}\PY{l+m+mi}{2}\PY{p}{]} \PY{o}{=} \PY{l+m+mi}{6}      \PY{c+c1}{\PYZsh{} Change the 3rd element to 6}
\PY{n}{my\PYZus{}list}\PY{o}{.}\PY{n}{append}\PY{p}{(}\PY{l+m+mi}{7}\PY{p}{)}   \PY{c+c1}{\PYZsh{} Add a new element 7 at the end of the list}
\PY{n}{my\PYZus{}list}\PY{o}{.}\PY{n}{insert}\PY{p}{(}\PY{l+m+mi}{0}\PY{p}{,} \PY{l+m+mi}{0}\PY{p}{)}  \PY{c+c1}{\PYZsh{} Add a new element 0 at the beginning of the list}
\PY{n+nb}{print}\PY{p}{(}\PY{n}{my\PYZus{}list}\PY{p}{)}

\PY{c+c1}{\PYZsh{} Delete List Elements}
\PY{n+nb}{print}\PY{p}{(}\PY{l+s+s2}{\PYZdq{}}\PY{l+s+s2}{\PYZsh{} Delete List Elements}\PY{l+s+s2}{\PYZdq{}}\PY{p}{)}
\PY{k}{del} \PY{n}{my\PYZus{}list}\PY{p}{[}\PY{l+m+mi}{1}\PY{p}{]}      \PY{c+c1}{\PYZsh{} Delete the 2nd element}
\PY{n}{my\PYZus{}list}\PY{o}{.}\PY{n}{remove}\PY{p}{(}\PY{l+m+mi}{4}\PY{p}{)}   \PY{c+c1}{\PYZsh{} Remove the first occurrence of 4 from the list}
\PY{n}{my\PYZus{}list}\PY{o}{.}\PY{n}{pop}\PY{p}{(}\PY{p}{)}       \PY{c+c1}{\PYZsh{} Remove the last element from the list}
\PY{n+nb}{print}\PY{p}{(}\PY{n}{my\PYZus{}list}\PY{p}{)}
\PY{c+c1}{\PYZsh{} Python List Methods}
\PY{n+nb}{print}\PY{p}{(}\PY{l+s+s2}{\PYZdq{}}\PY{l+s+s2}{\PYZsh{} Python List Methods}\PY{l+s+s2}{\PYZdq{}}\PY{p}{)}
\PY{n+nb}{print}\PY{p}{(}\PY{l+s+s2}{\PYZdq{}}\PY{l+s+s2}{the first occurence of the item 5 have the index:}\PY{l+s+s2}{\PYZdq{}}\PY{p}{,}\PY{n}{my\PYZus{}list}\PY{o}{.}\PY{n}{index}\PY{p}{(}\PY{l+m+mi}{5}\PY{p}{)}\PY{p}{)}
\PY{n}{my\PYZus{}list\PYZus{}2}\PY{o}{=}\PY{n}{my\PYZus{}list}\PY{o}{.}\PY{n}{copy}\PY{p}{(}\PY{p}{)}
\PY{n+nb}{print}\PY{p}{(}\PY{l+s+s2}{\PYZdq{}}\PY{l+s+s2}{my\PYZus{}list\PYZus{}2=}\PY{l+s+s2}{\PYZdq{}}\PY{p}{,}\PY{n}{my\PYZus{}list\PYZus{}2}\PY{p}{)}
\PY{n}{my\PYZus{}list}\PY{o}{.}\PY{n}{sort}\PY{p}{(}\PY{p}{)}      \PY{c+c1}{\PYZsh{} Sort the list in ascending order}
\PY{n+nb}{print}\PY{p}{(}\PY{n}{my\PYZus{}list}\PY{p}{)}
\PY{n}{my\PYZus{}list}\PY{o}{.}\PY{n}{reverse}\PY{p}{(}\PY{p}{)}   \PY{c+c1}{\PYZsh{} Reverse the order of the list}
\PY{n+nb}{print}\PY{p}{(}\PY{n}{my\PYZus{}list}\PY{p}{)}
\PY{n+nb}{print}\PY{p}{(}\PY{n+nb}{len}\PY{p}{(}\PY{n}{my\PYZus{}list}\PY{p}{)}\PY{p}{)} \PY{c+c1}{\PYZsh{} Output: 4, because we have removed two elements from the original list}
\PY{n}{my\PYZus{}list}\PY{o}{.}\PY{n}{clear}\PY{p}{(}\PY{p}{)}
\PY{n+nb}{type}\PY{p}{(}\PY{n}{my\PYZus{}list}\PY{p}{)}
\end{Verbatim}
\end{tcolorbox}

    \hypertarget{tuples}{%
\subsection{2. Tuples}\label{tuples}}

\hypertarget{create-a-tuple-in-python}{%
\subsubsection{Create a Tuple in
Python:}\label{create-a-tuple-in-python}}

A tuple is an \textbf{ordered} and \textbf{immutable} collection of
elements in Python. Tuples are defined using parentheses \textbf{()} and
separating the elements by \textbf{commas}.

Example:

\begin{Shaded}
\begin{Highlighting}[]
\CommentTok{\# Creating a tuple}
\NormalTok{my\_tuple }\OperatorTok{=}\NormalTok{ (}\DecValTok{1}\NormalTok{, }\DecValTok{2}\NormalTok{, }\DecValTok{3}\NormalTok{, }\StringTok{"four"}\NormalTok{, }\FloatTok{5.0}\NormalTok{)}
\BuiltInTok{print}\NormalTok{(my\_tuple)  }\CommentTok{\# Output: (1, 2, 3, \textquotesingle{}four\textquotesingle{}, 5.0)}
\end{Highlighting}
\end{Shaded}

\hypertarget{access-tuple-elements}{%
\subsubsection{Access Tuple Elements:}\label{access-tuple-elements}}

You can access the elements of a tuple using their index. Indexing
starts at 0 for the first element and ends at n-1 for the nth element.

Example:

\begin{Shaded}
\begin{Highlighting}[]
\CommentTok{\# Accessing elements of a tuple}
\NormalTok{my\_tuple }\OperatorTok{=}\NormalTok{ (}\DecValTok{1}\NormalTok{, }\DecValTok{2}\NormalTok{, }\DecValTok{3}\NormalTok{, }\StringTok{"four"}\NormalTok{, }\FloatTok{5.0}\NormalTok{)}
\BuiltInTok{print}\NormalTok{(my\_tuple[}\DecValTok{0}\NormalTok{])  }\CommentTok{\# Output: 1}
\BuiltInTok{print}\NormalTok{(my\_tuple[}\DecValTok{3}\NormalTok{])  }\CommentTok{\# Output: \textquotesingle{}four\textquotesingle{}}
\end{Highlighting}
\end{Shaded}

\hypertarget{modifying-a-tuple}{%
\subsubsection{Modifying a Tuple:}\label{modifying-a-tuple}}

As tuples are immutable, you cannot modify its elements. However, you
can create a new tuple by concatenating two or more tuples.

Example:

\begin{Shaded}
\begin{Highlighting}[]
\CommentTok{\# Modifying a tuple}
\NormalTok{tuple1 }\OperatorTok{=}\NormalTok{ (}\DecValTok{1}\NormalTok{, }\DecValTok{2}\NormalTok{, }\DecValTok{3}\NormalTok{)}
\NormalTok{tuple2 }\OperatorTok{=}\NormalTok{ (}\DecValTok{4}\NormalTok{, }\DecValTok{5}\NormalTok{, }\DecValTok{6}\NormalTok{)}
\NormalTok{new\_tuple }\OperatorTok{=}\NormalTok{ tuple1 }\OperatorTok{+}\NormalTok{ tuple2}
\BuiltInTok{print}\NormalTok{(new\_tuple)  }\CommentTok{\# Output: (1, 2, 3, 4, 5, 6)}
\end{Highlighting}
\end{Shaded}

\hypertarget{delete-tuple-elements}{%
\subsubsection{Delete Tuple Elements:}\label{delete-tuple-elements}}

As tuples are immutable, you cannot delete a single element. However,
you can delete the entire tuple.

Example:

\begin{Shaded}
\begin{Highlighting}[]
\CommentTok{\# Deleting a tuple}
\NormalTok{my\_tuple }\OperatorTok{=}\NormalTok{ (}\DecValTok{1}\NormalTok{, }\DecValTok{2}\NormalTok{, }\DecValTok{3}\NormalTok{, }\StringTok{"four"}\NormalTok{, }\FloatTok{5.0}\NormalTok{)}
\KeywordTok{del}\NormalTok{ my\_tuple}
\end{Highlighting}
\end{Shaded}

\hypertarget{tuple-methods-and-operations}{%
\subsubsection{Tuple methods and
operations:}\label{tuple-methods-and-operations}}

Tuples have various \texttt{built-in} methods and operations that can be
performed on them. Some of them are:

\begin{itemize}
\tightlist
\item
  \texttt{count()}: returns the number of times a specified element
  occurs in the tuple.
\item
  \texttt{index()}: returns the index of the first occurrence of a
  specified element in the tuple.
\item
  \texttt{len()}: returns the number of elements in the tuple.
\end{itemize}

Example:

\begin{Shaded}
\begin{Highlighting}[]
\CommentTok{\# Tuple methods and operations}
\NormalTok{my\_tuple }\OperatorTok{=}\NormalTok{ (}\DecValTok{1}\NormalTok{, }\DecValTok{2}\NormalTok{, }\DecValTok{2}\NormalTok{, }\DecValTok{3}\NormalTok{, }\DecValTok{4}\NormalTok{, }\DecValTok{4}\NormalTok{, }\DecValTok{4}\NormalTok{, }\StringTok{"four"}\NormalTok{)}
\BuiltInTok{print}\NormalTok{(my\_tuple.count(}\DecValTok{4}\NormalTok{))  }\CommentTok{\# Output: 3}
\BuiltInTok{print}\NormalTok{(my\_tuple.index(}\DecValTok{2}\NormalTok{))  }\CommentTok{\# Output: 1}
\BuiltInTok{print}\NormalTok{(}\BuiltInTok{len}\NormalTok{(my\_tuple))  }\CommentTok{\# Output: 8}
\end{Highlighting}
\end{Shaded}

\hypertarget{tuples-vs-lists}{%
\subsubsection{Tuples vs Lists}\label{tuples-vs-lists}}

When it comes to comparing lists and tuples in Python, there are some
differences and advantages to consider:

\begin{itemize}
\item
  \textbf{Mutability}: Lists are mutable, meaning you can add, remove,
  or modify elements in place, while tuples are immutable, meaning once
  they are created, you cannot change them.
\item
  \textbf{Performance}: Tuples are generally faster than lists because
  they are immutable and can be optimized by the interpreter. Lists, on
  the other hand, require more memory allocation and deallocation
  operations.
\item
  \textbf{Use cases}: Lists are best suited for scenarios where you need
  to modify the contents of the data structure frequently, while tuples
  are more suitable for situations where you want to ensure data
  integrity and prevent accidental modification.
\item
  \textbf{Comparison}: Lists and tuples can be compared using the
  \texttt{==} operator, which checks if both objects have the same
  elements in the same order. However, if you want to compare the
  identity of the objects, you can use the \texttt{is} operator.
\item
  \textbf{Advantages}: Tuples have some advantages over lists, such as
  being \textbf{hashable} ( a hash is integer identificator of that
  object which never changes during its lifetime) and therefore suitable
  for use as keys in dictionaries, and being more memory-efficient for
  storing small, fixed-size collections of data.
\end{itemize}

    \begin{tcolorbox}[breakable, size=fbox, boxrule=1pt, pad at break*=1mm,colback=cellbackground, colframe=cellborder]
\prompt{In}{incolor}{ }{\boxspacing}
\begin{Verbatim}[commandchars=\\\{\}]
\PY{c+c1}{\PYZsh{} Create a list and a tuple with the same elements}
\PY{n}{my\PYZus{}list} \PY{o}{=} \PY{p}{[}\PY{l+m+mi}{4}\PY{p}{,} \PY{l+m+mi}{2}\PY{p}{,} \PY{l+m+mi}{3}\PY{p}{]}
\PY{n}{my\PYZus{}tuple} \PY{o}{=} \PY{p}{(}\PY{l+m+mi}{1}\PY{p}{,} \PY{l+m+mi}{2}\PY{p}{,} \PY{l+m+mi}{3}\PY{p}{)}

\PY{c+c1}{\PYZsh{} Modify the first element of the list}
\PY{n}{my\PYZus{}list}\PY{p}{[}\PY{l+m+mi}{0}\PY{p}{]} \PY{o}{=} \PY{l+m+mi}{1}
\PY{n+nb}{print}\PY{p}{(}\PY{n}{my\PYZus{}list}\PY{p}{)}
\PY{c+c1}{\PYZsh{} Attempt to modify the first element of the tuple (will raise an error)my\PYZus{}tuple[0] = 4}
\PY{k}{try}\PY{p}{:}
    \PY{n}{my\PYZus{}tuple}\PY{p}{[}\PY{l+m+mi}{0}\PY{p}{]} \PY{o}{=} \PY{l+m+mi}{4}
\PY{k}{except} \PY{n+ne}{TypeError} \PY{k}{as} \PY{n}{e}\PY{p}{:}
    \PY{n+nb}{print}\PY{p}{(}\PY{n}{e}\PY{p}{)}

\PY{c+c1}{\PYZsh{} Compare the list and tuple}
\PY{n+nb}{print}\PY{p}{(}\PY{n}{my\PYZus{}list}\PY{p}{)}
\PY{n+nb}{print}\PY{p}{(}\PY{n}{my\PYZus{}tuple}\PY{p}{)}
\PY{n+nb}{print}\PY{p}{(}\PY{n}{my\PYZus{}list} \PY{o}{==} \PY{n+nb}{list}\PY{p}{(}\PY{n}{my\PYZus{}tuple}\PY{p}{)}\PY{p}{)}  \PY{c+c1}{\PYZsh{} True}
\PY{n+nb}{print}\PY{p}{(}\PY{n}{my\PYZus{}list} \PY{o+ow}{is} \PY{n+nb}{list}\PY{p}{(}\PY{n}{my\PYZus{}tuple}\PY{p}{)}\PY{p}{)}  \PY{c+c1}{\PYZsh{} False}
\end{Verbatim}
\end{tcolorbox}

    \hypertarget{sets}{%
\subsection{3. Sets}\label{sets}}

\hypertarget{definition-of-python-sets}{%
\subsubsection{Definition of Python
Sets:}\label{definition-of-python-sets}}

\begin{itemize}
\tightlist
\item
  A set in Python is an unordered collection of unique and immutable
  elements. It is defined by enclosing a comma-separated list of values
  within curly braces \texttt{\{\}} or by using the built-in
  \texttt{set()} function.
\item
  Sets in python imitate the mathematical sets notion.
\item
  Sets can be used to effectively prevent duplicate values. Example:
\end{itemize}

\begin{Shaded}
\begin{Highlighting}[]
\CommentTok{\# Defining a set}
\NormalTok{fruits }\OperatorTok{=}\NormalTok{ \{}\StringTok{\textquotesingle{}apple\textquotesingle{}}\NormalTok{, }\StringTok{\textquotesingle{}banana\textquotesingle{}}\NormalTok{, }\StringTok{\textquotesingle{}orange\textquotesingle{}}\NormalTok{\}}
\BuiltInTok{print}\NormalTok{(fruits) }\CommentTok{\# Output: \{\textquotesingle{}orange\textquotesingle{}, \textquotesingle{}apple\textquotesingle{}, \textquotesingle{}banana\textquotesingle{}\}}
\CommentTok{\# Using set() function}
\NormalTok{numbers }\OperatorTok{=} \BuiltInTok{set}\NormalTok{([}\DecValTok{1}\NormalTok{, }\DecValTok{2}\NormalTok{, }\DecValTok{3}\NormalTok{, }\DecValTok{4}\NormalTok{])}
\BuiltInTok{print}\NormalTok{(numbers) }\CommentTok{\# Output: \{1, 2, 3, 4\}}
\end{Highlighting}
\end{Shaded}

\hypertarget{creating-python-sets}{%
\subsubsection{Creating Python Sets:}\label{creating-python-sets}}

As mentioned before, we can create a set by enclosing a comma-separated
list of values within curly braces \texttt{\{\}} or using the
\texttt{set()} function. However, if we want to create an empty set, we
cannot use the curly braces as it will create an empty dictionary.
Instead, we need to use the \texttt{set()} function. Example:

\begin{Shaded}
\begin{Highlighting}[]
\CommentTok{\# Creating an empty set}
\NormalTok{empty\_set }\OperatorTok{=} \BuiltInTok{set}\NormalTok{()}
\BuiltInTok{print}\NormalTok{(empty\_set) }\CommentTok{\# Output: set()}

\CommentTok{\# Creating a non{-}empty set}
\NormalTok{fruits }\OperatorTok{=}\NormalTok{ \{}\StringTok{\textquotesingle{}apple\textquotesingle{}}\NormalTok{, }\StringTok{\textquotesingle{}banana\textquotesingle{}}\NormalTok{, }\StringTok{\textquotesingle{}orange\textquotesingle{}}\NormalTok{\}}
\BuiltInTok{print}\NormalTok{(fruits) }\CommentTok{\# Output: \{\textquotesingle{}orange\textquotesingle{}, \textquotesingle{}apple\textquotesingle{}, \textquotesingle{}banana\textquotesingle{}\}}
\end{Highlighting}
\end{Shaded}

\hypertarget{modifying-a-set-in-python}{%
\subsubsection{Modifying a set in
Python:}\label{modifying-a-set-in-python}}

\begin{itemize}
\tightlist
\item
  Since sets are mutable in Python, we can add or remove elements from
  it. We can add an element to a set using the \texttt{add()} method or
  add multiple elements using the \texttt{update()} method.
\item
  However, since they are unordered, indexing has no meaning.
\item
  We cannot access or change an element of a set using indexing or
  slicing.
\end{itemize}

Example:

\begin{Shaded}
\begin{Highlighting}[]
\CommentTok{\# Adding an element to a set}
\NormalTok{fruits }\OperatorTok{=}\NormalTok{ \{}\StringTok{\textquotesingle{}apple\textquotesingle{}}\NormalTok{, }\StringTok{\textquotesingle{}banana\textquotesingle{}}\NormalTok{, }\StringTok{\textquotesingle{}orange\textquotesingle{}}\NormalTok{\}}
\NormalTok{fruits.add(}\StringTok{\textquotesingle{}grapes\textquotesingle{}}\NormalTok{)}
\BuiltInTok{print}\NormalTok{(fruits) }\CommentTok{\# Output: \{\textquotesingle{}orange\textquotesingle{}, \textquotesingle{}apple\textquotesingle{}, \textquotesingle{}grapes\textquotesingle{}, \textquotesingle{}banana\textquotesingle{}\}}

\CommentTok{\# Adding multiple elements to a set}
\NormalTok{fruits.update([}\StringTok{\textquotesingle{}pineapple\textquotesingle{}}\NormalTok{, }\StringTok{\textquotesingle{}watermelon\textquotesingle{}}\NormalTok{])}
\BuiltInTok{print}\NormalTok{(fruits) }\CommentTok{\# Output: \{\textquotesingle{}orange\textquotesingle{}, \textquotesingle{}apple\textquotesingle{}, \textquotesingle{}pineapple\textquotesingle{}, \textquotesingle{}grapes\textquotesingle{}, \textquotesingle{}banana\textquotesingle{}, \textquotesingle{}watermelon\textquotesingle{}\}}

\CommentTok{\# Removing an element from a set}
\NormalTok{fruits.remove(}\StringTok{\textquotesingle{}banana\textquotesingle{}}\NormalTok{)}
\BuiltInTok{print}\NormalTok{(fruits) }\CommentTok{\# Output: \{\textquotesingle{}orange\textquotesingle{}, \textquotesingle{}apple\textquotesingle{}, \textquotesingle{}pineapple\textquotesingle{}, \textquotesingle{}grapes\textquotesingle{}, \textquotesingle{}watermelon\textquotesingle{}\}}

\CommentTok{\# Removing an element using discard()}
\NormalTok{fruits.discard(}\StringTok{\textquotesingle{}banana\textquotesingle{}}\NormalTok{)}
\BuiltInTok{print}\NormalTok{(fruits) }\CommentTok{\# Output: \{\textquotesingle{}orange\textquotesingle{}, \textquotesingle{}apple\textquotesingle{}, \textquotesingle{}pineapple\textquotesingle{}, \textquotesingle{}grapes\textquotesingle{}, \textquotesingle{}watermelon\textquotesingle{}\}}
\end{Highlighting}
\end{Shaded}

\hypertarget{removing-elements-from-a-set}{%
\subsubsection{Removing elements from a
set:}\label{removing-elements-from-a-set}}

Similarly, we can remove an element from a set using the
\texttt{remove()} or \texttt{discard()} method. However, if we try to
remove an element that is not present in the set using the
\texttt{remove()} method, it will raise a \texttt{KeyError}. To avoid
this error, we can use the \texttt{discard()} method instead. Another
method to remove elements from a set is the \texttt{pop()} method. This
method removes and returns an arbitrary element from the set.

Example:

\begin{Shaded}
\begin{Highlighting}[]
\CommentTok{\# Using remove() method}
\NormalTok{fruits }\OperatorTok{=}\NormalTok{ \{}\StringTok{\textquotesingle{}apple\textquotesingle{}}\NormalTok{, }\StringTok{\textquotesingle{}banana\textquotesingle{}}\NormalTok{, }\StringTok{\textquotesingle{}orange\textquotesingle{}}\NormalTok{\}}
\NormalTok{fruits.remove(}\StringTok{\textquotesingle{}banana\textquotesingle{}}\NormalTok{)}
\BuiltInTok{print}\NormalTok{(fruits) }\CommentTok{\# Output: \{\textquotesingle{}orange\textquotesingle{}, \textquotesingle{}apple\textquotesingle{}\}}

\CommentTok{\# Using discard() method}
\NormalTok{fruits.discard(}\StringTok{\textquotesingle{}banana\textquotesingle{}}\NormalTok{) }\CommentTok{\# No error is raised}
\BuiltInTok{print}\NormalTok{(fruits) }\CommentTok{\# Output: \{\textquotesingle{}orange\textquotesingle{}, \textquotesingle{}apple\textquotesingle{}\}}

\CommentTok{\# Using pop() method}
\NormalTok{fruits.pop()}
\BuiltInTok{print}\NormalTok{(fruits) }\CommentTok{\# Output: \{\textquotesingle{}apple\textquotesingle{}\}}
\end{Highlighting}
\end{Shaded}

\hypertarget{python-set-methods-and-operations}{%
\subsubsection{Python Set Methods and
Operations:}\label{python-set-methods-and-operations}}

Python sets come with a variety of built-in methods and operations that
can be used to perform various set operations. Some of these methods and
operations include \texttt{union()}, \texttt{intersection()},
\texttt{difference()}, \texttt{symmetric\_difference()},
\texttt{issubset()}, \texttt{issuperset()}, \texttt{copy()},
\texttt{clear()} and more. - \textbf{Union}: returns a new set
containing all the unique elements from both sets.

\begin{Shaded}
\begin{Highlighting}[]
\NormalTok{set1 }\OperatorTok{=}\NormalTok{ \{}\DecValTok{1}\NormalTok{, }\DecValTok{2}\NormalTok{, }\DecValTok{3}\NormalTok{\}}
\NormalTok{set2 }\OperatorTok{=}\NormalTok{ \{}\DecValTok{3}\NormalTok{, }\DecValTok{4}\NormalTok{, }\DecValTok{5}\NormalTok{\}}
\NormalTok{union\_set }\OperatorTok{=}\NormalTok{ set1.union(set2)}
\BuiltInTok{print}\NormalTok{(union\_set) }\CommentTok{\# output: \{1, 2, 3, 4, 5\}}
\end{Highlighting}
\end{Shaded}

\begin{itemize}
\tightlist
\item
  \textbf{Intersection}: returns a new set containing only the common
  elements from both sets.
\end{itemize}

\begin{Shaded}
\begin{Highlighting}[]
\NormalTok{set1 }\OperatorTok{=}\NormalTok{ \{}\DecValTok{1}\NormalTok{, }\DecValTok{2}\NormalTok{, }\DecValTok{3}\NormalTok{\}}
\NormalTok{set2 }\OperatorTok{=}\NormalTok{ \{}\DecValTok{3}\NormalTok{, }\DecValTok{4}\NormalTok{, }\DecValTok{5}\NormalTok{\}}
\NormalTok{intersection\_set }\OperatorTok{=}\NormalTok{ set1.intersection(set2)}
\BuiltInTok{print}\NormalTok{(intersection\_set) }\CommentTok{\# output: \{3\}}
\end{Highlighting}
\end{Shaded}

\begin{itemize}
\tightlist
\item
  \textbf{Difference}: returns a new set containing the elements from
  the first set that are not in the second set.
\end{itemize}

\begin{Shaded}
\begin{Highlighting}[]
\NormalTok{set1 }\OperatorTok{=}\NormalTok{ \{}\DecValTok{1}\NormalTok{, }\DecValTok{2}\NormalTok{, }\DecValTok{3}\NormalTok{\}}
\NormalTok{set2 }\OperatorTok{=}\NormalTok{ \{}\DecValTok{3}\NormalTok{, }\DecValTok{4}\NormalTok{, }\DecValTok{5}\NormalTok{\}}
\NormalTok{difference\_set }\OperatorTok{=}\NormalTok{ set1.difference(set2)}
\BuiltInTok{print}\NormalTok{(difference\_set) }\CommentTok{\# output: \{1, 2\}}
\end{Highlighting}
\end{Shaded}

\begin{itemize}
\tightlist
\item
  \textbf{Symmetric Difference}: returns a new set containing the
  elements that are in either of the sets, but not in both.
\end{itemize}

\begin{Shaded}
\begin{Highlighting}[]
\NormalTok{set1 }\OperatorTok{=}\NormalTok{ \{}\DecValTok{1}\NormalTok{, }\DecValTok{2}\NormalTok{, }\DecValTok{3}\NormalTok{\}}
\NormalTok{set2 }\OperatorTok{=}\NormalTok{ \{}\DecValTok{3}\NormalTok{, }\DecValTok{4}\NormalTok{, }\DecValTok{5}\NormalTok{\}}
\NormalTok{symmetric\_difference\_set }\OperatorTok{=}\NormalTok{ set1.symmetric\_difference(set2)}
\BuiltInTok{print}\NormalTok{(symmetric\_difference\_set) }\CommentTok{\# output: \{1, 2, 4, 5\}}
\end{Highlighting}
\end{Shaded}

\begin{itemize}
\tightlist
\item
  \textbf{Subset}: checks if one set is a subset of another set.
\end{itemize}

\begin{Shaded}
\begin{Highlighting}[]
\NormalTok{set1 }\OperatorTok{=}\NormalTok{ \{}\DecValTok{1}\NormalTok{, }\DecValTok{2}\NormalTok{, }\DecValTok{3}\NormalTok{, }\DecValTok{4}\NormalTok{\}}
\NormalTok{set2 }\OperatorTok{=}\NormalTok{ \{}\DecValTok{1}\NormalTok{, }\DecValTok{2}\NormalTok{, }\DecValTok{3}\NormalTok{\}}
\NormalTok{is\_subset }\OperatorTok{=}\NormalTok{ set2.issubset(set1)}
\BuiltInTok{print}\NormalTok{(is\_subset) }\CommentTok{\# output: True}
\end{Highlighting}
\end{Shaded}

\begin{itemize}
\tightlist
\item
  \textbf{Superset}: checks if one set is a superset of another set.
\end{itemize}

\begin{Shaded}
\begin{Highlighting}[]
\NormalTok{set1 }\OperatorTok{=}\NormalTok{ \{}\DecValTok{1}\NormalTok{, }\DecValTok{2}\NormalTok{, }\DecValTok{3}\NormalTok{, }\DecValTok{4}\NormalTok{\}}
\NormalTok{set2 }\OperatorTok{=}\NormalTok{ \{}\DecValTok{1}\NormalTok{, }\DecValTok{2}\NormalTok{, }\DecValTok{3}\NormalTok{\}}
\NormalTok{is\_superset }\OperatorTok{=}\NormalTok{ set1.issuperset(set2)}
\BuiltInTok{print}\NormalTok{(is\_superset) }\CommentTok{\# output: True}
\end{Highlighting}
\end{Shaded}

\begin{itemize}
\tightlist
\item
  \textbf{Copy}: creates a copy of the set.
\end{itemize}

\begin{Shaded}
\begin{Highlighting}[]
\NormalTok{set1 }\OperatorTok{=}\NormalTok{ \{}\DecValTok{1}\NormalTok{, }\DecValTok{2}\NormalTok{, }\DecValTok{3}\NormalTok{\}}
\NormalTok{set2 }\OperatorTok{=}\NormalTok{ set1.copy()}
\BuiltInTok{print}\NormalTok{(set2) }\CommentTok{\# output: \{1, 2, 3\}}
\end{Highlighting}
\end{Shaded}

\begin{itemize}
\tightlist
\item
  \textbf{Clear}: to remove all elements of a set.
\end{itemize}

\begin{Shaded}
\begin{Highlighting}[]
\NormalTok{set1 }\OperatorTok{=}\NormalTok{ \{}\DecValTok{1}\NormalTok{, }\DecValTok{2}\NormalTok{, }\DecValTok{3}\NormalTok{\}}
\NormalTok{set2 }\OperatorTok{=}\NormalTok{ set1.clear()}
\BuiltInTok{print}\NormalTok{(set2) }\CommentTok{\# output: None}
\end{Highlighting}
\end{Shaded}

\hypertarget{dictionnaries}{%
\subsection{4. Dictionnaries}\label{dictionnaries}}

\hypertarget{whats-a-dictionary}{%
\subsubsection{What's a Dictionary?}\label{whats-a-dictionary}}

A dictionary is a built-in data type in Python that allows you to store
data in \texttt{key:value} pairs. In other words, it is a collection of
elements that are stored as a \texttt{key:value} pair, where each key
maps to a corresponding value.

Example:

\begin{Shaded}
\begin{Highlighting}[]
\CommentTok{\# creating a dictionary}
\NormalTok{my\_dict }\OperatorTok{=}\NormalTok{ \{}\StringTok{"name"}\NormalTok{: }\StringTok{"John"}\NormalTok{, }\StringTok{"age"}\NormalTok{: }\DecValTok{25}\NormalTok{, }\StringTok{"gender"}\NormalTok{: }\StringTok{"Male"}\NormalTok{\}}

\CommentTok{\# accessing values using keys}
\BuiltInTok{print}\NormalTok{(my\_dict[}\StringTok{"name"}\NormalTok{])  }\CommentTok{\# output: John}

\CommentTok{\# modifying value of a key}
\NormalTok{my\_dict[}\StringTok{"age"}\NormalTok{] }\OperatorTok{=} \DecValTok{30}

\CommentTok{\# adding a new key{-}value pair}
\NormalTok{my\_dict[}\StringTok{"city"}\NormalTok{] }\OperatorTok{=} \StringTok{"New York"}

\CommentTok{\# removing a key{-}value pair}
\KeywordTok{del}\NormalTok{ my\_dict[}\StringTok{"gender"}\NormalTok{]}
\end{Highlighting}
\end{Shaded}

\hypertarget{how-to-create-a-dictionary}{%
\subsubsection{How to Create a
Dictionary?}\label{how-to-create-a-dictionary}}

A dictionary is created using curly braces \texttt{\{\}} and the
key-value pairs are separated by a colon (\texttt{:}). Each key-value
pair is separated by a comma (\texttt{,}).

Example:

\begin{Shaded}
\begin{Highlighting}[]
\CommentTok{\# creating an empty dictionary}
\NormalTok{empty\_dict }\OperatorTok{=}\NormalTok{ \{\}}

\CommentTok{\# creating a dictionary with some key{-}value pairs}
\NormalTok{my\_dict }\OperatorTok{=}\NormalTok{ \{}\StringTok{"name"}\NormalTok{: }\StringTok{"John"}\NormalTok{, }\StringTok{"age"}\NormalTok{: }\DecValTok{25}\NormalTok{, }\StringTok{"gender"}\NormalTok{: }\StringTok{"Male"}\NormalTok{\}}

\CommentTok{\# creating a dictionary using the dict() constructor}
\NormalTok{another\_dict }\OperatorTok{=} \BuiltInTok{dict}\NormalTok{(name}\OperatorTok{=}\StringTok{"Jane"}\NormalTok{, age}\OperatorTok{=}\DecValTok{30}\NormalTok{, city}\OperatorTok{=}\StringTok{"San Francisco"}\NormalTok{)}
\end{Highlighting}
\end{Shaded}

\hypertarget{types-of-values-and-keys}{%
\subsubsection{Types of Values and
Keys}\label{types-of-values-and-keys}}

In a dictionary, keys must be unique and immutable (cannot be changed).
Values can be of any data type and can be changed. There are several
built-in data types that can be used as keys, such as integers, strings,
and tuples.

Example:

\begin{Shaded}
\begin{Highlighting}[]
\CommentTok{\# dictionary with integer keys}
\NormalTok{my\_dict }\OperatorTok{=}\NormalTok{ \{}\DecValTok{1}\NormalTok{: }\StringTok{"apple"}\NormalTok{, }\DecValTok{2}\NormalTok{: }\StringTok{"banana"}\NormalTok{, }\DecValTok{3}\NormalTok{: }\StringTok{"cherry"}\NormalTok{\}}

\CommentTok{\# dictionary with string keys}
\NormalTok{my\_dict }\OperatorTok{=}\NormalTok{ \{}\StringTok{"name"}\NormalTok{: }\StringTok{"John"}\NormalTok{, }\StringTok{"age"}\NormalTok{: }\DecValTok{25}\NormalTok{\}}

\CommentTok{\# dictionary with tuple keys}
\NormalTok{my\_dict }\OperatorTok{=}\NormalTok{ \{(}\StringTok{"name"}\NormalTok{, }\DecValTok{1}\NormalTok{): }\StringTok{"John"}\NormalTok{, (}\StringTok{"age"}\NormalTok{, }\DecValTok{1}\NormalTok{): }\DecValTok{25}\NormalTok{\}}
\end{Highlighting}
\end{Shaded}

\hypertarget{accessing-elements-from-dictionary}{%
\subsubsection{Accessing Elements from
Dictionary}\label{accessing-elements-from-dictionary}}

You can access the values of a dictionary by specifying its
corresponding key in square brackets. If the key is not present in the
dictionary, it will raise a KeyError.

Example:

\begin{Shaded}
\begin{Highlighting}[]
\CommentTok{\# accessing a value using a key}
\NormalTok{my\_dict }\OperatorTok{=}\NormalTok{ \{}\StringTok{"name"}\NormalTok{: }\StringTok{"John"}\NormalTok{, }\StringTok{"age"}\NormalTok{: }\DecValTok{25}\NormalTok{\}}
\BuiltInTok{print}\NormalTok{(my\_dict[}\StringTok{"name"}\NormalTok{])  }\CommentTok{\# output: John}

\CommentTok{\# using get() method to access a value}
\BuiltInTok{print}\NormalTok{(my\_dict.get(}\StringTok{"age"}\NormalTok{))  }\CommentTok{\# output: 25}

\CommentTok{\# handling KeyError}
\BuiltInTok{print}\NormalTok{(my\_dict.get(}\StringTok{"city"}\NormalTok{))  }\CommentTok{\# output: None}
\BuiltInTok{print}\NormalTok{(my\_dict[}\StringTok{"city"}\NormalTok{])  }\CommentTok{\# raises KeyError}
\end{Highlighting}
\end{Shaded}

\hypertarget{changing-and-adding-dictionary-elements}{%
\subsubsection{Changing and Adding Dictionary
Elements}\label{changing-and-adding-dictionary-elements}}

You can change the value of an existing key or add a new key-value pair
to a dictionary using the assignment operator (\texttt{=}). We can also
access a using the method \texttt{get(\textless{}key\textgreater{})}.

Example:

\begin{Shaded}
\begin{Highlighting}[]
\CommentTok{\# changing value of a key}
\NormalTok{my\_dict }\OperatorTok{=}\NormalTok{ \{}\StringTok{"name"}\NormalTok{: }\StringTok{"John"}\NormalTok{, }\StringTok{"age"}\NormalTok{: }\DecValTok{25}\NormalTok{\}}
\NormalTok{my\_dict[}\StringTok{"age"}\NormalTok{] }\OperatorTok{=} \DecValTok{30}

\CommentTok{\# adding a new key{-}value pair}
\NormalTok{my\_dict[}\StringTok{"city"}\NormalTok{] }\OperatorTok{=} \StringTok{"New York"}
\end{Highlighting}
\end{Shaded}

\hypertarget{removing-elements-from-dictionary}{%
\subsubsection{Removing Elements from
Dictionary}\label{removing-elements-from-dictionary}}

You can remove a key-value pair from a dictionary using the \texttt{del}
statement or the \texttt{pop()} method. The \texttt{pop()} method also
returns the value of the deleted key. We can also use the methods
\texttt{popitem()} to remove and return an arbitrary (key, value) item
pair from the dictionary, and \texttt{clear()} to remove all the items
at once.

Example:

\begin{Shaded}
\begin{Highlighting}[]
\CommentTok{\# removing a key{-}value pair using del statement}
\NormalTok{my\_dict }\OperatorTok{=}\NormalTok{ \{}\StringTok{"name"}\NormalTok{: }\StringTok{"John"}\NormalTok{, }\StringTok{"age"}\NormalTok{: }\DecValTok{25}\NormalTok{, }\StringTok{"city"}\NormalTok{: }\StringTok{"New York"}\NormalTok{\}}
\KeywordTok{del}\NormalTok{ my\_dict[}\StringTok{"city"}\NormalTok{]}

\CommentTok{\# removing a key{-}value pair using pop() method}
\NormalTok{my\_dict }\OperatorTok{=}\NormalTok{ \{}\StringTok{"name"}\NormalTok{: }\StringTok{"John"}\NormalTok{, }\StringTok{"age"}\NormalTok{: }\DecValTok{25}\NormalTok{, }\StringTok{"city"}\NormalTok{: }\StringTok{"New York"}\NormalTok{\}}
\NormalTok{age }\OperatorTok{=}\NormalTok{ my\_dict.pop(}\StringTok{"age"}\NormalTok{)}
\NormalTok{my\_dict.popitem()}
\end{Highlighting}
\end{Shaded}


    % Add a bibliography block to the postdoc
    
    
    
\end{document}
