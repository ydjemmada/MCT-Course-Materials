\documentclass[11pt]{article}

    \usepackage[breakable]{tcolorbox}
    \usepackage{parskip} % Stop auto-indenting (to mimic markdown behaviour)
    
    \usepackage{iftex}
    \ifPDFTeX
    	\usepackage[T1]{fontenc}
    	\usepackage{mathpazo}
    \else
    	\usepackage{fontspec}
    \fi

    % Basic figure setup, for now with no caption control since it's done
    % automatically by Pandoc (which extracts ![](path) syntax from Markdown).
    \usepackage{graphicx}
    % Maintain compatibility with old templates. Remove in nbconvert 6.0
    \let\Oldincludegraphics\includegraphics
    % Ensure that by default, figures have no caption (until we provide a
    % proper Figure object with a Caption API and a way to capture that
    % in the conversion process - todo).
    \usepackage{caption}
    \DeclareCaptionFormat{nocaption}{}
    \captionsetup{format=nocaption,aboveskip=0pt,belowskip=0pt}

    \usepackage{float}
    \floatplacement{figure}{H} % forces figures to be placed at the correct location
    \usepackage{xcolor} % Allow colors to be defined
    \usepackage{enumerate} % Needed for markdown enumerations to work
    \usepackage{geometry} % Used to adjust the document margins
    \usepackage{amsmath} % Equations
    \usepackage{amssymb} % Equations
    \usepackage{textcomp} % defines textquotesingle
    % Hack from http://tex.stackexchange.com/a/47451/13684:
    \AtBeginDocument{%
        \def\PYZsq{\textquotesingle}% Upright quotes in Pygmentized code
    }
    \usepackage{upquote} % Upright quotes for verbatim code
    \usepackage{eurosym} % defines \euro
    \usepackage[mathletters]{ucs} % Extended unicode (utf-8) support
    \usepackage{fancyvrb} % verbatim replacement that allows latex
    \usepackage{grffile} % extends the file name processing of package graphics 
                         % to support a larger range
    \makeatletter % fix for old versions of grffile with XeLaTeX
    \@ifpackagelater{grffile}{2019/11/01}
    {
      % Do nothing on new versions
    }
    {
      \def\Gread@@xetex#1{%
        \IfFileExists{"\Gin@base".bb}%
        {\Gread@eps{\Gin@base.bb}}%
        {\Gread@@xetex@aux#1}%
      }
    }
    \makeatother
    \usepackage[Export]{adjustbox} % Used to constrain images to a maximum size
    \adjustboxset{max size={0.9\linewidth}{0.9\paperheight}}

    % The hyperref package gives us a pdf with properly built
    % internal navigation ('pdf bookmarks' for the table of contents,
    % internal cross-reference links, web links for URLs, etc.)
    \usepackage{hyperref}
    % The default LaTeX title has an obnoxious amount of whitespace. By default,
    % titling removes some of it. It also provides customization options.
    \usepackage{titling}
    \usepackage{longtable} % longtable support required by pandoc >1.10
    \usepackage{booktabs}  % table support for pandoc > 1.12.2
    \usepackage[inline]{enumitem} % IRkernel/repr support (it uses the enumerate* environment)
    \usepackage[normalem]{ulem} % ulem is needed to support strikethroughs (\sout)
                                % normalem makes italics be italics, not underlines
    \usepackage{mathrsfs}
    

    
    % Colors for the hyperref package
    \definecolor{urlcolor}{rgb}{0,.145,.698}
    \definecolor{linkcolor}{rgb}{.71,0.21,0.01}
    \definecolor{citecolor}{rgb}{.12,.54,.11}

    % ANSI colors
    \definecolor{ansi-black}{HTML}{3E424D}
    \definecolor{ansi-black-intense}{HTML}{282C36}
    \definecolor{ansi-red}{HTML}{E75C58}
    \definecolor{ansi-red-intense}{HTML}{B22B31}
    \definecolor{ansi-green}{HTML}{00A250}
    \definecolor{ansi-green-intense}{HTML}{007427}
    \definecolor{ansi-yellow}{HTML}{DDB62B}
    \definecolor{ansi-yellow-intense}{HTML}{B27D12}
    \definecolor{ansi-blue}{HTML}{208FFB}
    \definecolor{ansi-blue-intense}{HTML}{0065CA}
    \definecolor{ansi-magenta}{HTML}{D160C4}
    \definecolor{ansi-magenta-intense}{HTML}{A03196}
    \definecolor{ansi-cyan}{HTML}{60C6C8}
    \definecolor{ansi-cyan-intense}{HTML}{258F8F}
    \definecolor{ansi-white}{HTML}{C5C1B4}
    \definecolor{ansi-white-intense}{HTML}{A1A6B2}
    \definecolor{ansi-default-inverse-fg}{HTML}{FFFFFF}
    \definecolor{ansi-default-inverse-bg}{HTML}{000000}

    % common color for the border for error outputs.
    \definecolor{outerrorbackground}{HTML}{FFDFDF}

    % commands and environments needed by pandoc snippets
    % extracted from the output of `pandoc -s`
    \providecommand{\tightlist}{%
      \setlength{\itemsep}{0pt}\setlength{\parskip}{0pt}}
    \DefineVerbatimEnvironment{Highlighting}{Verbatim}{commandchars=\\\{\}}
    % Add ',fontsize=\small' for more characters per line
    \newenvironment{Shaded}{}{}
    \newcommand{\KeywordTok}[1]{\textcolor[rgb]{0.00,0.44,0.13}{\textbf{{#1}}}}
    \newcommand{\DataTypeTok}[1]{\textcolor[rgb]{0.56,0.13,0.00}{{#1}}}
    \newcommand{\DecValTok}[1]{\textcolor[rgb]{0.25,0.63,0.44}{{#1}}}
    \newcommand{\BaseNTok}[1]{\textcolor[rgb]{0.25,0.63,0.44}{{#1}}}
    \newcommand{\FloatTok}[1]{\textcolor[rgb]{0.25,0.63,0.44}{{#1}}}
    \newcommand{\CharTok}[1]{\textcolor[rgb]{0.25,0.44,0.63}{{#1}}}
    \newcommand{\StringTok}[1]{\textcolor[rgb]{0.25,0.44,0.63}{{#1}}}
    \newcommand{\CommentTok}[1]{\textcolor[rgb]{0.38,0.63,0.69}{\textit{{#1}}}}
    \newcommand{\OtherTok}[1]{\textcolor[rgb]{0.00,0.44,0.13}{{#1}}}
    \newcommand{\AlertTok}[1]{\textcolor[rgb]{1.00,0.00,0.00}{\textbf{{#1}}}}
    \newcommand{\FunctionTok}[1]{\textcolor[rgb]{0.02,0.16,0.49}{{#1}}}
    \newcommand{\RegionMarkerTok}[1]{{#1}}
    \newcommand{\ErrorTok}[1]{\textcolor[rgb]{1.00,0.00,0.00}{\textbf{{#1}}}}
    \newcommand{\NormalTok}[1]{{#1}}
    
    % Additional commands for more recent versions of Pandoc
    \newcommand{\ConstantTok}[1]{\textcolor[rgb]{0.53,0.00,0.00}{{#1}}}
    \newcommand{\SpecialCharTok}[1]{\textcolor[rgb]{0.25,0.44,0.63}{{#1}}}
    \newcommand{\VerbatimStringTok}[1]{\textcolor[rgb]{0.25,0.44,0.63}{{#1}}}
    \newcommand{\SpecialStringTok}[1]{\textcolor[rgb]{0.73,0.40,0.53}{{#1}}}
    \newcommand{\ImportTok}[1]{{#1}}
    \newcommand{\DocumentationTok}[1]{\textcolor[rgb]{0.73,0.13,0.13}{\textit{{#1}}}}
    \newcommand{\AnnotationTok}[1]{\textcolor[rgb]{0.38,0.63,0.69}{\textbf{\textit{{#1}}}}}
    \newcommand{\CommentVarTok}[1]{\textcolor[rgb]{0.38,0.63,0.69}{\textbf{\textit{{#1}}}}}
    \newcommand{\VariableTok}[1]{\textcolor[rgb]{0.10,0.09,0.49}{{#1}}}
    \newcommand{\ControlFlowTok}[1]{\textcolor[rgb]{0.00,0.44,0.13}{\textbf{{#1}}}}
    \newcommand{\OperatorTok}[1]{\textcolor[rgb]{0.40,0.40,0.40}{{#1}}}
    \newcommand{\BuiltInTok}[1]{{#1}}
    \newcommand{\ExtensionTok}[1]{{#1}}
    \newcommand{\PreprocessorTok}[1]{\textcolor[rgb]{0.74,0.48,0.00}{{#1}}}
    \newcommand{\AttributeTok}[1]{\textcolor[rgb]{0.49,0.56,0.16}{{#1}}}
    \newcommand{\InformationTok}[1]{\textcolor[rgb]{0.38,0.63,0.69}{\textbf{\textit{{#1}}}}}
    \newcommand{\WarningTok}[1]{\textcolor[rgb]{0.38,0.63,0.69}{\textbf{\textit{{#1}}}}}
    
    
    % Define a nice break command that doesn't care if a line doesn't already
    % exist.
    \def\br{\hspace*{\fill} \\* }
    % Math Jax compatibility definitions
    \def\gt{>}
    \def\lt{<}
    \let\Oldtex\TeX
    \let\Oldlatex\LaTeX
    \renewcommand{\TeX}{\textrm{\Oldtex}}
    \renewcommand{\LaTeX}{\textrm{\Oldlatex}}
    % Document parameters
    % Document title
    \title{Worksheet 02: Introduction to NumPy and its application in image manipulation }
    
    
    
    
    
% Pygments definitions
\makeatletter
\def\PY@reset{\let\PY@it=\relax \let\PY@bf=\relax%
    \let\PY@ul=\relax \let\PY@tc=\relax%
    \let\PY@bc=\relax \let\PY@ff=\relax}
\def\PY@tok#1{\csname PY@tok@#1\endcsname}
\def\PY@toks#1+{\ifx\relax#1\empty\else%
    \PY@tok{#1}\expandafter\PY@toks\fi}
\def\PY@do#1{\PY@bc{\PY@tc{\PY@ul{%
    \PY@it{\PY@bf{\PY@ff{#1}}}}}}}
\def\PY#1#2{\PY@reset\PY@toks#1+\relax+\PY@do{#2}}

\@namedef{PY@tok@w}{\def\PY@tc##1{\textcolor[rgb]{0.73,0.73,0.73}{##1}}}
\@namedef{PY@tok@c}{\let\PY@it=\textit\def\PY@tc##1{\textcolor[rgb]{0.25,0.50,0.50}{##1}}}
\@namedef{PY@tok@cp}{\def\PY@tc##1{\textcolor[rgb]{0.74,0.48,0.00}{##1}}}
\@namedef{PY@tok@k}{\let\PY@bf=\textbf\def\PY@tc##1{\textcolor[rgb]{0.00,0.50,0.00}{##1}}}
\@namedef{PY@tok@kp}{\def\PY@tc##1{\textcolor[rgb]{0.00,0.50,0.00}{##1}}}
\@namedef{PY@tok@kt}{\def\PY@tc##1{\textcolor[rgb]{0.69,0.00,0.25}{##1}}}
\@namedef{PY@tok@o}{\def\PY@tc##1{\textcolor[rgb]{0.40,0.40,0.40}{##1}}}
\@namedef{PY@tok@ow}{\let\PY@bf=\textbf\def\PY@tc##1{\textcolor[rgb]{0.67,0.13,1.00}{##1}}}
\@namedef{PY@tok@nb}{\def\PY@tc##1{\textcolor[rgb]{0.00,0.50,0.00}{##1}}}
\@namedef{PY@tok@nf}{\def\PY@tc##1{\textcolor[rgb]{0.00,0.00,1.00}{##1}}}
\@namedef{PY@tok@nc}{\let\PY@bf=\textbf\def\PY@tc##1{\textcolor[rgb]{0.00,0.00,1.00}{##1}}}
\@namedef{PY@tok@nn}{\let\PY@bf=\textbf\def\PY@tc##1{\textcolor[rgb]{0.00,0.00,1.00}{##1}}}
\@namedef{PY@tok@ne}{\let\PY@bf=\textbf\def\PY@tc##1{\textcolor[rgb]{0.82,0.25,0.23}{##1}}}
\@namedef{PY@tok@nv}{\def\PY@tc##1{\textcolor[rgb]{0.10,0.09,0.49}{##1}}}
\@namedef{PY@tok@no}{\def\PY@tc##1{\textcolor[rgb]{0.53,0.00,0.00}{##1}}}
\@namedef{PY@tok@nl}{\def\PY@tc##1{\textcolor[rgb]{0.63,0.63,0.00}{##1}}}
\@namedef{PY@tok@ni}{\let\PY@bf=\textbf\def\PY@tc##1{\textcolor[rgb]{0.60,0.60,0.60}{##1}}}
\@namedef{PY@tok@na}{\def\PY@tc##1{\textcolor[rgb]{0.49,0.56,0.16}{##1}}}
\@namedef{PY@tok@nt}{\let\PY@bf=\textbf\def\PY@tc##1{\textcolor[rgb]{0.00,0.50,0.00}{##1}}}
\@namedef{PY@tok@nd}{\def\PY@tc##1{\textcolor[rgb]{0.67,0.13,1.00}{##1}}}
\@namedef{PY@tok@s}{\def\PY@tc##1{\textcolor[rgb]{0.73,0.13,0.13}{##1}}}
\@namedef{PY@tok@sd}{\let\PY@it=\textit\def\PY@tc##1{\textcolor[rgb]{0.73,0.13,0.13}{##1}}}
\@namedef{PY@tok@si}{\let\PY@bf=\textbf\def\PY@tc##1{\textcolor[rgb]{0.73,0.40,0.53}{##1}}}
\@namedef{PY@tok@se}{\let\PY@bf=\textbf\def\PY@tc##1{\textcolor[rgb]{0.73,0.40,0.13}{##1}}}
\@namedef{PY@tok@sr}{\def\PY@tc##1{\textcolor[rgb]{0.73,0.40,0.53}{##1}}}
\@namedef{PY@tok@ss}{\def\PY@tc##1{\textcolor[rgb]{0.10,0.09,0.49}{##1}}}
\@namedef{PY@tok@sx}{\def\PY@tc##1{\textcolor[rgb]{0.00,0.50,0.00}{##1}}}
\@namedef{PY@tok@m}{\def\PY@tc##1{\textcolor[rgb]{0.40,0.40,0.40}{##1}}}
\@namedef{PY@tok@gh}{\let\PY@bf=\textbf\def\PY@tc##1{\textcolor[rgb]{0.00,0.00,0.50}{##1}}}
\@namedef{PY@tok@gu}{\let\PY@bf=\textbf\def\PY@tc##1{\textcolor[rgb]{0.50,0.00,0.50}{##1}}}
\@namedef{PY@tok@gd}{\def\PY@tc##1{\textcolor[rgb]{0.63,0.00,0.00}{##1}}}
\@namedef{PY@tok@gi}{\def\PY@tc##1{\textcolor[rgb]{0.00,0.63,0.00}{##1}}}
\@namedef{PY@tok@gr}{\def\PY@tc##1{\textcolor[rgb]{1.00,0.00,0.00}{##1}}}
\@namedef{PY@tok@ge}{\let\PY@it=\textit}
\@namedef{PY@tok@gs}{\let\PY@bf=\textbf}
\@namedef{PY@tok@gp}{\let\PY@bf=\textbf\def\PY@tc##1{\textcolor[rgb]{0.00,0.00,0.50}{##1}}}
\@namedef{PY@tok@go}{\def\PY@tc##1{\textcolor[rgb]{0.53,0.53,0.53}{##1}}}
\@namedef{PY@tok@gt}{\def\PY@tc##1{\textcolor[rgb]{0.00,0.27,0.87}{##1}}}
\@namedef{PY@tok@err}{\def\PY@bc##1{{\setlength{\fboxsep}{\string -\fboxrule}\fcolorbox[rgb]{1.00,0.00,0.00}{1,1,1}{\strut ##1}}}}
\@namedef{PY@tok@kc}{\let\PY@bf=\textbf\def\PY@tc##1{\textcolor[rgb]{0.00,0.50,0.00}{##1}}}
\@namedef{PY@tok@kd}{\let\PY@bf=\textbf\def\PY@tc##1{\textcolor[rgb]{0.00,0.50,0.00}{##1}}}
\@namedef{PY@tok@kn}{\let\PY@bf=\textbf\def\PY@tc##1{\textcolor[rgb]{0.00,0.50,0.00}{##1}}}
\@namedef{PY@tok@kr}{\let\PY@bf=\textbf\def\PY@tc##1{\textcolor[rgb]{0.00,0.50,0.00}{##1}}}
\@namedef{PY@tok@bp}{\def\PY@tc##1{\textcolor[rgb]{0.00,0.50,0.00}{##1}}}
\@namedef{PY@tok@fm}{\def\PY@tc##1{\textcolor[rgb]{0.00,0.00,1.00}{##1}}}
\@namedef{PY@tok@vc}{\def\PY@tc##1{\textcolor[rgb]{0.10,0.09,0.49}{##1}}}
\@namedef{PY@tok@vg}{\def\PY@tc##1{\textcolor[rgb]{0.10,0.09,0.49}{##1}}}
\@namedef{PY@tok@vi}{\def\PY@tc##1{\textcolor[rgb]{0.10,0.09,0.49}{##1}}}
\@namedef{PY@tok@vm}{\def\PY@tc##1{\textcolor[rgb]{0.10,0.09,0.49}{##1}}}
\@namedef{PY@tok@sa}{\def\PY@tc##1{\textcolor[rgb]{0.73,0.13,0.13}{##1}}}
\@namedef{PY@tok@sb}{\def\PY@tc##1{\textcolor[rgb]{0.73,0.13,0.13}{##1}}}
\@namedef{PY@tok@sc}{\def\PY@tc##1{\textcolor[rgb]{0.73,0.13,0.13}{##1}}}
\@namedef{PY@tok@dl}{\def\PY@tc##1{\textcolor[rgb]{0.73,0.13,0.13}{##1}}}
\@namedef{PY@tok@s2}{\def\PY@tc##1{\textcolor[rgb]{0.73,0.13,0.13}{##1}}}
\@namedef{PY@tok@sh}{\def\PY@tc##1{\textcolor[rgb]{0.73,0.13,0.13}{##1}}}
\@namedef{PY@tok@s1}{\def\PY@tc##1{\textcolor[rgb]{0.73,0.13,0.13}{##1}}}
\@namedef{PY@tok@mb}{\def\PY@tc##1{\textcolor[rgb]{0.40,0.40,0.40}{##1}}}
\@namedef{PY@tok@mf}{\def\PY@tc##1{\textcolor[rgb]{0.40,0.40,0.40}{##1}}}
\@namedef{PY@tok@mh}{\def\PY@tc##1{\textcolor[rgb]{0.40,0.40,0.40}{##1}}}
\@namedef{PY@tok@mi}{\def\PY@tc##1{\textcolor[rgb]{0.40,0.40,0.40}{##1}}}
\@namedef{PY@tok@il}{\def\PY@tc##1{\textcolor[rgb]{0.40,0.40,0.40}{##1}}}
\@namedef{PY@tok@mo}{\def\PY@tc##1{\textcolor[rgb]{0.40,0.40,0.40}{##1}}}
\@namedef{PY@tok@ch}{\let\PY@it=\textit\def\PY@tc##1{\textcolor[rgb]{0.25,0.50,0.50}{##1}}}
\@namedef{PY@tok@cm}{\let\PY@it=\textit\def\PY@tc##1{\textcolor[rgb]{0.25,0.50,0.50}{##1}}}
\@namedef{PY@tok@cpf}{\let\PY@it=\textit\def\PY@tc##1{\textcolor[rgb]{0.25,0.50,0.50}{##1}}}
\@namedef{PY@tok@c1}{\let\PY@it=\textit\def\PY@tc##1{\textcolor[rgb]{0.25,0.50,0.50}{##1}}}
\@namedef{PY@tok@cs}{\let\PY@it=\textit\def\PY@tc##1{\textcolor[rgb]{0.25,0.50,0.50}{##1}}}

\def\PYZbs{\char`\\}
\def\PYZus{\char`\_}
\def\PYZob{\char`\{}
\def\PYZcb{\char`\}}
\def\PYZca{\char`\^}
\def\PYZam{\char`\&}
\def\PYZlt{\char`\<}
\def\PYZgt{\char`\>}
\def\PYZsh{\char`\#}
\def\PYZpc{\char`\%}
\def\PYZdl{\char`\$}
\def\PYZhy{\char`\-}
\def\PYZsq{\char`\'}
\def\PYZdq{\char`\"}
\def\PYZti{\char`\~}
% for compatibility with earlier versions
\def\PYZat{@}
\def\PYZlb{[}
\def\PYZrb{]}
\makeatother


    % For linebreaks inside Verbatim environment from package fancyvrb. 
    \makeatletter
        \newbox\Wrappedcontinuationbox 
        \newbox\Wrappedvisiblespacebox 
        \newcommand*\Wrappedvisiblespace {\textcolor{red}{\textvisiblespace}} 
        \newcommand*\Wrappedcontinuationsymbol {\textcolor{red}{\llap{\tiny$\m@th\hookrightarrow$}}} 
        \newcommand*\Wrappedcontinuationindent {3ex } 
        \newcommand*\Wrappedafterbreak {\kern\Wrappedcontinuationindent\copy\Wrappedcontinuationbox} 
        % Take advantage of the already applied Pygments mark-up to insert 
        % potential linebreaks for TeX processing. 
        %        {, <, #, %, $, ' and ": go to next line. 
        %        _, }, ^, &, >, - and ~: stay at end of broken line. 
        % Use of \textquotesingle for straight quote. 
        \newcommand*\Wrappedbreaksatspecials {% 
            \def\PYGZus{\discretionary{\char`\_}{\Wrappedafterbreak}{\char`\_}}% 
            \def\PYGZob{\discretionary{}{\Wrappedafterbreak\char`\{}{\char`\{}}% 
            \def\PYGZcb{\discretionary{\char`\}}{\Wrappedafterbreak}{\char`\}}}% 
            \def\PYGZca{\discretionary{\char`\^}{\Wrappedafterbreak}{\char`\^}}% 
            \def\PYGZam{\discretionary{\char`\&}{\Wrappedafterbreak}{\char`\&}}% 
            \def\PYGZlt{\discretionary{}{\Wrappedafterbreak\char`\<}{\char`\<}}% 
            \def\PYGZgt{\discretionary{\char`\>}{\Wrappedafterbreak}{\char`\>}}% 
            \def\PYGZsh{\discretionary{}{\Wrappedafterbreak\char`\#}{\char`\#}}% 
            \def\PYGZpc{\discretionary{}{\Wrappedafterbreak\char`\%}{\char`\%}}% 
            \def\PYGZdl{\discretionary{}{\Wrappedafterbreak\char`\$}{\char`\$}}% 
            \def\PYGZhy{\discretionary{\char`\-}{\Wrappedafterbreak}{\char`\-}}% 
            \def\PYGZsq{\discretionary{}{\Wrappedafterbreak\textquotesingle}{\textquotesingle}}% 
            \def\PYGZdq{\discretionary{}{\Wrappedafterbreak\char`\"}{\char`\"}}% 
            \def\PYGZti{\discretionary{\char`\~}{\Wrappedafterbreak}{\char`\~}}% 
        } 
        % Some characters . , ; ? ! / are not pygmentized. 
        % This macro makes them "active" and they will insert potential linebreaks 
        \newcommand*\Wrappedbreaksatpunct {% 
            \lccode`\~`\.\lowercase{\def~}{\discretionary{\hbox{\char`\.}}{\Wrappedafterbreak}{\hbox{\char`\.}}}% 
            \lccode`\~`\,\lowercase{\def~}{\discretionary{\hbox{\char`\,}}{\Wrappedafterbreak}{\hbox{\char`\,}}}% 
            \lccode`\~`\;\lowercase{\def~}{\discretionary{\hbox{\char`\;}}{\Wrappedafterbreak}{\hbox{\char`\;}}}% 
            \lccode`\~`\:\lowercase{\def~}{\discretionary{\hbox{\char`\:}}{\Wrappedafterbreak}{\hbox{\char`\:}}}% 
            \lccode`\~`\?\lowercase{\def~}{\discretionary{\hbox{\char`\?}}{\Wrappedafterbreak}{\hbox{\char`\?}}}% 
            \lccode`\~`\!\lowercase{\def~}{\discretionary{\hbox{\char`\!}}{\Wrappedafterbreak}{\hbox{\char`\!}}}% 
            \lccode`\~`\/\lowercase{\def~}{\discretionary{\hbox{\char`\/}}{\Wrappedafterbreak}{\hbox{\char`\/}}}% 
            \catcode`\.\active
            \catcode`\,\active 
            \catcode`\;\active
            \catcode`\:\active
            \catcode`\?\active
            \catcode`\!\active
            \catcode`\/\active 
            \lccode`\~`\~ 	
        }
    \makeatother

    \let\OriginalVerbatim=\Verbatim
    \makeatletter
    \renewcommand{\Verbatim}[1][1]{%
        %\parskip\z@skip
        \sbox\Wrappedcontinuationbox {\Wrappedcontinuationsymbol}%
        \sbox\Wrappedvisiblespacebox {\FV@SetupFont\Wrappedvisiblespace}%
        \def\FancyVerbFormatLine ##1{\hsize\linewidth
            \vtop{\raggedright\hyphenpenalty\z@\exhyphenpenalty\z@
                \doublehyphendemerits\z@\finalhyphendemerits\z@
                \strut ##1\strut}%
        }%
        % If the linebreak is at a space, the latter will be displayed as visible
        % space at end of first line, and a continuation symbol starts next line.
        % Stretch/shrink are however usually zero for typewriter font.
        \def\FV@Space {%
            \nobreak\hskip\z@ plus\fontdimen3\font minus\fontdimen4\font
            \discretionary{\copy\Wrappedvisiblespacebox}{\Wrappedafterbreak}
            {\kern\fontdimen2\font}%
        }%
        
        % Allow breaks at special characters using \PYG... macros.
        \Wrappedbreaksatspecials
        % Breaks at punctuation characters . , ; ? ! and / need catcode=\active 	
        \OriginalVerbatim[#1,codes*=\Wrappedbreaksatpunct]%
    }
    \makeatother

    % Exact colors from NB
    \definecolor{incolor}{HTML}{303F9F}
    \definecolor{outcolor}{HTML}{D84315}
    \definecolor{cellborder}{HTML}{CFCFCF}
    \definecolor{cellbackground}{HTML}{F7F7F7}
    
    % prompt
    \makeatletter
    \newcommand{\boxspacing}{\kern\kvtcb@left@rule\kern\kvtcb@boxsep}
    \makeatother
    \newcommand{\prompt}[4]{
        {\ttfamily\llap{{\color{#2}[#3]:\hspace{3pt}#4}}\vspace{-\baselineskip}}
    }
    

    
    % Prevent overflowing lines due to hard-to-break entities
    \sloppy 
    % Setup hyperref package
    \hypersetup{
      breaklinks=true,  % so long urls are correctly broken across lines
      colorlinks=true,
      urlcolor=urlcolor,
      linkcolor=linkcolor,
      citecolor=citecolor,
      }
    % Slightly bigger margins than the latex defaults
    
    \geometry{verbose,tmargin=1in,bmargin=1in,lmargin=1in,rmargin=1in}
    \date{}
    
    \renewcommand{\thesubsection}{\arabic{subsection}}
\begin{document}
    
    \maketitle
    
    

    
    \hypertarget{part-one-numpy-routines}{%
\section*{\center Part one: numpy routines:}\label{part-one-numpy-routines}}

  \hypertarget{import-numpy-and-get-documentations}{%
  \subsection{\texorpdfstring{\emph{Import numpy and get
  documentations}}{Import numpy and get documentations}}\label{import-numpy-and-get-documentations}}

  \begin{itemize}
  \item
    Write the instructions to import the NumPy module and alias it as
    np.
  \item
    How to get help on a NumPy function?
  \end{itemize}
  \hypertarget{array-creation-shapes-and-types}{%
  \subsection{\texorpdfstring{\emph{Array creation (shapes and
  types)}}{Array creation (shapes and types)}}\label{array-creation-shapes-and-types}}

  \begin{itemize}
  \item
    Create a 1D NumPy array containing integers from 0 to 9.
  \item
    Create a 2D NumPy array containing integers from 0 to 15 arranged in
    a 4x4 grid.
  \item
    Create a 1D NumPy array containing the values 2, 4, 6, 8, and 10
    using the \texttt{np.arange()} function.
  \item
    Create a 3D NumPy array containing random floating-point numbers
    between 0 and 1 with shape (2, 3, 4).
  \item
    Create a 1D NumPy array of length 5 containing all zeros.
  \item
    Create a 1D NumPy array of length 8 containing all ones.
  \item
    Create a 2D NumPy array with shape (3, 2) containing all zeros.
  \item
    Create a 2D NumPy array with shape (4, 3) containing all ones.
  \item
    Create a 1D NumPy array containing 30 equally spaced values between
    0 and 1.
  \item
    Create a 4x4 2D array with ones on the diagonal and zeroes
    elsewhere.
  \item
    Create a 5x2 array of flaot numbers, filled with 7.
  \item
    Let \texttt{x\ =\ np.arange(7,\ dtype=np.int64)}. Create an array of
    \texttt{8} with the same shape and type as X.
  \item
    Show the documentation of the following NumPy functions then try to
    use it, \texttt{np.asscalar},\texttt{np.asarray},
    \texttt{np.asmatrix}.
  \item
    Let \texttt{x\ =\ np.array({[}1,\ 2,\ 3{]})}. Create an array copy
    of x, which has a different id from x.
  \item
    Create a 1-D array of 50 element spaced evenly on a log scale
    between 3. and 10. .
  \item
  Use the function \texttt{np.diagflat} to create a matrix with
    the diagonal \texttt{{[}1,2,3,4,5,6,7{]}}.
  \item
  Use the function \texttt{np.tri}; \texttt{np.tril} and \texttt{np.triu} to
    create the following arrays.

\begin{itemize}
  \item 
  
  \begin{Shaded}
  \begin{tcolorbox}[breakable, size=fbox, boxrule=1pt, pad at break*=1mm,colback=cellbackground, colframe=cellborder]
    \begin{Highlighting}[]   
  \NormalTok{array([[ }\FloatTok{0.}\NormalTok{, }\FloatTok{0.}\NormalTok{, }\FloatTok{0.}\NormalTok{, }\FloatTok{0.}\NormalTok{, }\FloatTok{0.}\NormalTok{],}
  \NormalTok{       [ }\FloatTok{1.}\NormalTok{, }\FloatTok{0.}\NormalTok{, }\FloatTok{0.}\NormalTok{, }\FloatTok{0.}\NormalTok{, }\FloatTok{0.}\NormalTok{],}
  \NormalTok{       [ }\FloatTok{1.}\NormalTok{, }\FloatTok{1.}\NormalTok{, }\FloatTok{0.}\NormalTok{, }\FloatTok{0.}\NormalTok{, }\FloatTok{0.}\NormalTok{]])}
\end{Highlighting}
\end{tcolorbox}
\end{Shaded}
  
  \item
      
  
  \begin{Shaded}
  \begin{tcolorbox}[breakable, size=fbox, boxrule=1pt, pad at break*=1mm,colback=cellbackground, colframe=cellborder]
    \begin{Highlighting}[]
  \NormalTok{array([[ }\DecValTok{0}\NormalTok{,  }\DecValTok{0}\NormalTok{,  }\DecValTok{0}\NormalTok{],}
  \NormalTok{        [ }\DecValTok{4}\NormalTok{,  }\DecValTok{0}\NormalTok{,  }\DecValTok{0}\NormalTok{],}
  \NormalTok{        [ }\DecValTok{7}\NormalTok{,  }\DecValTok{8}\NormalTok{,  }\DecValTok{0}\NormalTok{],}
  \NormalTok{        [}\DecValTok{10}\NormalTok{, }\DecValTok{11}\NormalTok{, }\DecValTok{12}\NormalTok{]])}
  \end{Highlighting}
  \end{tcolorbox}
  \end{Shaded}
  \item
  \begin{Shaded}
    \begin{tcolorbox}[breakable, size=fbox, boxrule=1pt, pad at break*=1mm,colback=cellbackground, colframe=cellborder]
    \begin{Highlighting}[]
    \NormalTok{array([[ }\DecValTok{1}\NormalTok{,  }\DecValTok{2}\NormalTok{,  }\DecValTok{3}\NormalTok{],}
    \NormalTok{        [ }\DecValTok{4}\NormalTok{,  }\DecValTok{5}\NormalTok{,  }\DecValTok{6}\NormalTok{],}
    \NormalTok{        [ }\DecValTok{0}\NormalTok{,  }\DecValTok{8}\NormalTok{,  }\DecValTok{9}\NormalTok{],}
    \NormalTok{        [}\DecValTok{0}\NormalTok{, }\DecValTok{0}\NormalTok{, }\DecValTok{12}\NormalTok{]])}
    \end{Highlighting}
    \end{tcolorbox}
    \end{Shaded}
  
\end{itemize}
  \end{itemize}
  \hypertarget{array-manipulation}{%
  \subsection{\texorpdfstring{\emph{Array
  manipulation}}{Array manipulation}}\label{array-manipulation}}

  \begin{itemize}
  \tightlist
  \item
    Let x be a \texttt{ndarray\ {[}10,\ 10,\ 3{]}} with all elements set
    to one. Reshape x so that the size of the second dimension equals
    150.
  \item
    Execute the following instruction then guess what the negative index do:

    \begin{itemize}
    \tightlist
    \item
      \texttt{np.reshape(x,\ {[}-1,3,\ 25{]})}
    \item
      \texttt{np.reshape(x,\ {[}-1,\ 50{]})}
    \end{itemize}
  \item
    Let x be array
    \texttt{{[}{[}\ 0,\ \ 1{]},\ {[}\ 2,\ \ 3{]},\ {[}\ 4,\ \ 5{]},\ {[}\ 6,\ \ 7{]},\ {[}\ 8,\ \ 9{]},\ {[}10,\ 11{]}{]}}.
    Convert it to
    \texttt{{[}\ 0,\ 1,\ 2,\ 3,\ 4,\ 5,\ 6,\ 7,\ 8,\ 9,\ 10,\ 11{]}}.
  \item
    Execute the following instructions then guess what the
    \texttt{np.squeeze} do
  \end{itemize}

\begin{Shaded}
\begin{tcolorbox}[breakable, size=fbox, boxrule=1pt, pad at break*=1mm,colback=cellbackground, colframe=cellborder]
  \begin{Highlighting}[]
\NormalTok{x}\OperatorTok{=}\NormalTok{np.ones((}\DecValTok{3}\NormalTok{,}\DecValTok{1}\NormalTok{,}\DecValTok{2}\NormalTok{))}
\BuiltInTok{print}\NormalTok{(x)}
\NormalTok{np.squeeze(x)}
\end{Highlighting}
\end{tcolorbox}
\end{Shaded}

  \begin{itemize}
  \tightlist
  \item
    What is the instruction that transpose a matrix?
  \end{itemize}

  \hypertarget{solutions}{%
\subsection*{Solutions}\label{solutions}}
\setcounter{subsection}{0}
  \hypertarget{import-numpy-and-get-documentations}{%
  \subsection{\texorpdfstring{\emph{Import numpy and get
  documentations}}{Import numpy and get documentations}}\label{import-numpy-and-get-documentations}}

  To import the NumPy module and alias it as np, you can use the
  following statement:


\begin{Shaded}
\begin{tcolorbox}[breakable, size=fbox, boxrule=1pt, pad at break*=1mm,colback=cellbackground, colframe=cellborder]
  \begin{Highlighting}[]
\ImportTok{import}\NormalTok{ numpy }\ImportTok{as}\NormalTok{ np}
\end{Highlighting}
\end{tcolorbox}
\end{Shaded}

This will allow you to access all the functions and classes in NumPy by
prefixing them with np.

To get help on a NumPy function, you can use the \texttt{np.info()}
function or the ? operator in the Python console. For example, to get
help on the np.arange() function, you can type:

\begin{Shaded}
\begin{tcolorbox}[breakable, size=fbox, boxrule=1pt, pad at break*=1mm,colback=cellbackground, colframe=cellborder]
  \begin{Highlighting}[]
\NormalTok{np.info(np.arange)}
\end{Highlighting}
\end{tcolorbox}
\end{Shaded}

or

\begin{Shaded}
\begin{tcolorbox}[breakable, size=fbox, boxrule=1pt, pad at break*=1mm,colback=cellbackground, colframe=cellborder]
  \begin{Highlighting}[]
\NormalTok{np.arange?}
\end{Highlighting}
\end{tcolorbox}
\end{Shaded}

Both of these commands will display the documentation for the
np.arange() function, including its parameters, return value, and usage
examples. - \#\# \emph{Array creation (shapes and types)}
\hypertarget{array-creation-shapes-and-types}{%
\subsection{\texorpdfstring{\emph{Array creation (shapes and
types)}}{Array creation (shapes and types)}}\label{array-creation-shapes-and-types}}
\begin{Shaded}
\begin{tcolorbox}[breakable, size=fbox, boxrule=1pt, pad at break*=1mm,colback=cellbackground, colframe=cellborder]
  \begin{Highlighting}[]
\ImportTok{import}\NormalTok{ numpy }\ImportTok{as}\NormalTok{ np}

\CommentTok{\# Create a 1D NumPy array containing integers from 0 to 9.}
\NormalTok{arr1d }\OperatorTok{=}\NormalTok{ np.arange(}\DecValTok{10}\NormalTok{)}
\BuiltInTok{print}\NormalTok{(arr1d)}

\CommentTok{\# Create a 2D NumPy array containing integers from 0 to 15 arranged in a 4x4 grid.}
\NormalTok{arr2d }\OperatorTok{=}\NormalTok{ np.arange(}\DecValTok{16}\NormalTok{).reshape((}\DecValTok{4}\NormalTok{, }\DecValTok{4}\NormalTok{))}
\BuiltInTok{print}\NormalTok{(arr2d)}

\CommentTok{\# Create a 1D NumPy array containing the values 2, 4, 6, 8, and 10 using}
\CommentTok{\ the np.arange() function.}
\NormalTok{arr1d\_custom }\OperatorTok{=}\NormalTok{ np.arange(}\DecValTok{2}\NormalTok{, }\DecValTok{11}\NormalTok{, }\DecValTok{2}\NormalTok{)}
\BuiltInTok{print}\NormalTok{(arr1d\_custom)}

\CommentTok{\# Create a 3D NumPy array containing random floating{-}point numbers between 0}
\CommentTok{\  and 1 with shape (2, 3, 4).}
\NormalTok{arr3d }\OperatorTok{=}\NormalTok{ np.random.rand(}\DecValTok{2}\NormalTok{, }\DecValTok{3}\NormalTok{, }\DecValTok{4}\NormalTok{)}
\BuiltInTok{print}\NormalTok{(arr3d)}

\CommentTok{\# Create a 1D NumPy array of length 5 containing all zeros.}
\NormalTok{arr\_zeros\_1d }\OperatorTok{=}\NormalTok{ np.zeros(}\DecValTok{5}\NormalTok{)}
\BuiltInTok{print}\NormalTok{(arr\_zeros\_1d)}

\CommentTok{\# Create a 1D NumPy array of length 8 containing all ones.}
\NormalTok{arr\_ones\_1d }\OperatorTok{=}\NormalTok{ np.ones(}\DecValTok{8}\NormalTok{)}
\BuiltInTok{print}\NormalTok{(arr\_ones\_1d)}

\CommentTok{\# Create a 2D NumPy array with shape (3, 2) containing all zeros.}
\NormalTok{arr\_zeros\_2d }\OperatorTok{=}\NormalTok{ np.zeros((}\DecValTok{3}\NormalTok{, }\DecValTok{2}\NormalTok{))}
\BuiltInTok{print}\NormalTok{(arr\_zeros\_2d)}

\CommentTok{\# Create a 2D NumPy array with shape (4, 3) containing all ones.}
\NormalTok{arr\_ones\_2d }\OperatorTok{=}\NormalTok{ np.ones((}\DecValTok{4}\NormalTok{, }\DecValTok{3}\NormalTok{))}
\BuiltInTok{print}\NormalTok{(arr\_ones\_2d)}

\CommentTok{\# Create a 1D NumPy array containing 30 equally spaced values between 0 and 1.}
\NormalTok{arr\_linspace }\OperatorTok{=}\NormalTok{ np.linspace(}\DecValTok{0}\NormalTok{, }\DecValTok{1}\NormalTok{, }\DecValTok{30}\NormalTok{)}
\BuiltInTok{print}\NormalTok{(arr\_linspace)}

\CommentTok{\# Create a 4x4 2D array with ones on the diagonal and zeroes elsewhere.}
\NormalTok{arr\_eye }\OperatorTok{=}\NormalTok{ np.eye(}\DecValTok{4}\NormalTok{)}
\BuiltInTok{print}\NormalTok{(arr\_eye)}

\CommentTok{\# Create a 5x2 array of float numbers, filled with 7.}
\NormalTok{arr\_custom }\OperatorTok{=}\NormalTok{ np.full((}\DecValTok{5}\NormalTok{, }\DecValTok{2}\NormalTok{), }\FloatTok{7.0}\NormalTok{)}
\BuiltInTok{print}\NormalTok{(arr\_custom)}

\CommentTok{\# Let x = np.arange(7, dtype=np.int64). Create an array of 8 with the }
\CommentTok{\ same shape and type as X.}
\NormalTok{x }\OperatorTok{=}\NormalTok{ np.arange(}\DecValTok{7}\NormalTok{, dtype}\OperatorTok{=}\NormalTok{np.int64)}
\NormalTok{arr\_same }\OperatorTok{=}\NormalTok{ np.ones\_like(x)}\OperatorTok{*}\DecValTok{8}
\BuiltInTok{print}\NormalTok{(arr\_same)}

\CommentTok{\# Show the documentation of the following NumPy functions then try to }
\CommentTok{\ use it, np.asscalar,np.asarray, np.asmatrix.}
\BuiltInTok{print}\NormalTok{(np.info(np.asscalar))}
\BuiltInTok{print}\NormalTok{(np.info(np.asarray))}
\BuiltInTok{print}\NormalTok{(np.info(np.asmatrix))}

\CommentTok{\# Let x = np.array([1, 2, 3]). Create an array copy of x, which has a different}
\CommentTok{\  id from x.}
\NormalTok{x }\OperatorTok{=}\NormalTok{ np.array([}\DecValTok{1}\NormalTok{, }\DecValTok{2}\NormalTok{, }\DecValTok{3}\NormalTok{])}
\NormalTok{arr\_copy }\OperatorTok{=}\NormalTok{ np.copy(x)}
\BuiltInTok{print}\NormalTok{(}\StringTok{"id of x is"}\NormalTok{,}\BuiltInTok{id}\NormalTok{(x))}
\BuiltInTok{print}\NormalTok{(}\StringTok{"id of the copy of x is"}\NormalTok{,}\BuiltInTok{id}\NormalTok{(arr\_copy))}

\CommentTok{\# Create a 1{-}D array of 50 element spaced evenly on a log scale between 3. and 10.}
\NormalTok{arr\_logspace }\OperatorTok{=}\NormalTok{ np.logspace(}\DecValTok{3}\NormalTok{, }\DecValTok{10}\NormalTok{, }\DecValTok{50}\NormalTok{)}
\BuiltInTok{print}\NormalTok{(arr\_logspace)}

\CommentTok{\# Use the function np.diagflat to create a matrix with the }
\CommentTok{\ diagonal [1,2,3,4,5,6,7].}
\NormalTok{arr\_diag }\OperatorTok{=}\NormalTok{ np.diagflat([}\DecValTok{1}\NormalTok{, }\DecValTok{2}\NormalTok{, }\DecValTok{3}\NormalTok{, }\DecValTok{4}\NormalTok{, }\DecValTok{5}\NormalTok{, }\DecValTok{6}\NormalTok{, }\DecValTok{7}\NormalTok{])}
\BuiltInTok{print}\NormalTok{(arr\_diag)}

\CommentTok{\# Use the function np.tri, np.tril and np.triu to create the following arrays.}
\CommentTok{\# array([[ 0., 0., 0., 0., 0.],}
\CommentTok{\#        [ 1., 0., 0., 0., 0.],}
\CommentTok{\#        [ 1., 1., 0., 0., 0.]])}
\NormalTok{arr1 }\OperatorTok{=}\NormalTok{ np.tril(np.ones((}\DecValTok{3}\NormalTok{, }\DecValTok{3}\NormalTok{)), }\OperatorTok{{-}}\DecValTok{1}\NormalTok{)}

\CommentTok{\# array([[ 0,  0,  0],}
\CommentTok{\#        [ 4,  0,  0],}
\CommentTok{\#        [ 7,  8,  0],}
\CommentTok{\#        [10, 11, 12]])}
\NormalTok{arr2 }\OperatorTok{=}\NormalTok{ np.tril(np.arange(}\DecValTok{1}\NormalTok{, }\DecValTok{13}\NormalTok{).reshape(}\DecValTok{4}\NormalTok{, }\DecValTok{3}\NormalTok{), }\OperatorTok{{-}}\DecValTok{1}\NormalTok{)}

\CommentTok{\# array([[ 1, 2, 3],}
\CommentTok{\#        [ 4, 5, 6],}
\CommentTok{\#        [ 0, 8, 9],}
\CommentTok{\#        [ 0, 0, 12]])}
\NormalTok{arr3 }\OperatorTok{=}\NormalTok{ np.triu(np.arange(}\DecValTok{1}\NormalTok{,}\DecValTok{13}\NormalTok{).reshape(}\DecValTok{4}\NormalTok{,}\DecValTok{3}\NormalTok{), }\OperatorTok{{-}}\DecValTok{1}\NormalTok{)}
\BuiltInTok{print}\NormalTok{(arr1,}\StringTok{"}\CharTok{\textbackslash{}n}\StringTok{"}\NormalTok{,arr2,}\StringTok{"}\CharTok{\textbackslash{}n}\StringTok{"}\NormalTok{,arr3)}
\end{Highlighting}
\end{tcolorbox}
\end{Shaded}



  \hypertarget{array-manipulation}{%
  \subsection{\texorpdfstring{\emph{Array
  manipulation}}{Array manipulation}}\label{array-manipulation}}

  \begin{itemize}
  \tightlist
  \item
    Let x be a ndarray {[}10, 10, 3{]} with all elements set to one.
    Reshape x so that the size of the second dimension equals 150.
  \end{itemize}

\begin{Shaded}
\begin{tcolorbox}[breakable, size=fbox, boxrule=1pt, pad at break*=1mm,colback=cellbackground, colframe=cellborder]
  \begin{Highlighting}[]
\NormalTok{x }\OperatorTok{=}\NormalTok{ np.ones((}\DecValTok{10}\NormalTok{, }\DecValTok{10}\NormalTok{, }\DecValTok{3}\NormalTok{))}
\NormalTok{x }\OperatorTok{=}\NormalTok{ np.reshape(x, (}\DecValTok{2}\NormalTok{, }\DecValTok{150}\NormalTok{))}
\end{Highlighting}
\end{tcolorbox}
\end{Shaded}

  \begin{itemize}
  \item
    Execute the following instruction then guess what the negative index do:

    \begin{itemize}
    \tightlist
    \item
      \texttt{np.reshape(x,\ {[}-1,3,\ 25{]})}: Reshapes x into a new 3D
      array with the first dimension determined automatically and the
      second and third dimensions set to 3 and 25, respectively. The
      negative first dimension argument means that its size is inferred
      from the input size and the other two dimensions.
    \item
      \texttt{np.reshape(x,\ {[}-1,\ 50{]})}: Reshapes x into a new 2D
      array with the first dimension determined automatically and the
      second dimension set to 50. The negative first dimension argument
      means that its size is inferred from the input size and the second
      dimension.
    \end{itemize}
  \item
    Let x be array
    \texttt{{[}{[}\ 0,\ 1{]},\ {[}\ 2,\ 3{]},\ {[}\ 4,\ 5{]},\ {[}\ 6,\ 7{]},\ {[}\ 8,\ 9{]},\ {[}10,\ 11{]}{]}}.
    Convert it to
    \texttt{{[}\ 0,\ 1,\ 2,\ 3,\ 4,\ 5,\ 6,\ 7,\ 8,\ 9,\ 10,\ 11{]}}.
  \end{itemize}

\begin{Shaded}
\begin{tcolorbox}[breakable, size=fbox, boxrule=1pt, pad at break*=1mm,colback=cellbackground, colframe=cellborder]
  \begin{Highlighting}[]
\NormalTok{x }\OperatorTok{=}\NormalTok{ np.array([[}\DecValTok{0}\NormalTok{, }\DecValTok{1}\NormalTok{], [}\DecValTok{2}\NormalTok{, }\DecValTok{3}\NormalTok{], [}\DecValTok{4}\NormalTok{, }\DecValTok{5}\NormalTok{], [}\DecValTok{6}\NormalTok{, }\DecValTok{7}\NormalTok{], [}\DecValTok{8}\NormalTok{, }\DecValTok{9}\NormalTok{], [}\DecValTok{10}\NormalTok{, }\DecValTok{11}\NormalTok{]])}
\NormalTok{x }\OperatorTok{=}\NormalTok{ np.ravel(x) }\CommentTok{\#Or x.flatten()}
\BuiltInTok{print}\NormalTok{(x)}
\end{Highlighting}
\end{tcolorbox}
\end{Shaded}

  \begin{itemize}
  \tightlist
  \item
    Execute the following instructions then guess what the
    \texttt{np.squeeze} do.
  \end{itemize}

\begin{Shaded}
\begin{tcolorbox}[breakable, size=fbox, boxrule=1pt, pad at break*=1mm,colback=cellbackground, colframe=cellborder]
  \begin{Highlighting}[]
\NormalTok{x }\OperatorTok{=}\NormalTok{ np.ones((}\DecValTok{3}\NormalTok{, }\DecValTok{1}\NormalTok{, }\DecValTok{2}\NormalTok{))}
\BuiltInTok{print}\NormalTok{(x)}
\NormalTok{np.squeeze(x)}
\end{Highlighting}
\end{tcolorbox}
\end{Shaded}

  \begin{itemize}
  \tightlist
  \item
    The \texttt{np.squeeze} function removes dimensions of size 1 from
    an array. In the given example, \texttt{x} is a 3D array with
    dimensions (3, 1, 2). The second dimension has size 1, so
    \texttt{np.squeeze(x)} removes it and returns a 2D array with
    dimensions (3, 2).
  \item
    The NumPy instruction to transpose a matrix is
    \texttt{numpy.transpose} or simply \texttt{ndarray.T}. For example,
    if you have a 2D NumPy array \texttt{A}, you can transpose it using
    \texttt{A.T}.
  \end{itemize}

\newpage
    \hypertarget{part-two-using-numpy-arrays-operations-and-image-manipulation-application}{%
\section*{\center Part Two: Using numpy arrays' operations and image
manipulation
(application):}\label{part-two-using-numpy-arrays-operations-and-image-manipulation-application}}

\begin{enumerate}
\def\labelenumi{\arabic{enumi}.}
\item
  To begin, let us load the image into numpy. This can be done by using
  the imread() function from the matplotlib library. This function
  allows numpy to read graphic files with different extensions. The
  output is a two-dimensional array with the dimensions equal to the
  dimensions of the image, and the values corresponding to the colors of
  the pixels. Here we will work with a grayscale image, so the elements
  in the array will be integers ranging from 0 to 255 in the numpy
  integer format uint8. Type in the following code to load the file
  ``fibonacci.jpg'' into numpy:

\begin{Shaded}
\begin{tcolorbox}[breakable, size=fbox, boxrule=1pt, pad at break*=1mm,colback=cellbackground, colframe=cellborder]
  \begin{Highlighting}[]
\ImportTok{from}\NormalTok{ matplotlib }\ImportTok{import}\NormalTok{ pyplot }\ImportTok{as}\NormalTok{ plt}
\NormalTok{ImJPG }\OperatorTok{=}\NormalTok{ plt.imread(}\StringTok{\textquotesingle{}fibonacci.jpg\textquotesingle{}}\NormalTok{)}
\CommentTok{\#you can get the image from the link}
\CommentTok{\  https://ydjemmada.github.io/fibonacci.jpg}
\end{Highlighting}
\end{tcolorbox}
\end{Shaded}

  The array ImJPG is a two-dimensional array of the type uint8 which
  contains values from 0 to 255 corresponding to the color of each
  individual pixel in the image, where 0 corresponds to black and 255 to
  white. You can visualize this array by printing it in the console:

\begin{Shaded}
\begin{tcolorbox}[breakable, size=fbox, boxrule=1pt, pad at break*=1mm,colback=cellbackground, colframe=cellborder]
  \begin{Highlighting}[]
\BuiltInTok{print}\NormalTok{(ImJPG) }\CommentTok{\#prints the array values}
\NormalTok{plt.imshow(ImJPG,cmap}\OperatorTok{=}\StringTok{\textquotesingle{}gray\textquotesingle{}}\NormalTok{)}\CommentTok{\#shows the image}
\end{Highlighting}
\end{tcolorbox}
\end{Shaded}
\item
  Use the shape attribute to check the dimensions of the obtained array
  ImJPG:

\begin{Shaded}
\begin{tcolorbox}[breakable, size=fbox, boxrule=1pt, pad at break*=1mm,colback=cellbackground, colframe=cellborder]
  \begin{Highlighting}[]
\NormalTok{m, n }\OperatorTok{=}\NormalTok{ ImJPG.shape}
\end{Highlighting}
\end{tcolorbox}
\end{Shaded}

  \begin{enumerate}
  \def\labelenumii{\arabic{enumii}.}
  \tightlist
  \item
    What are the dimensions of the image?
  \end{enumerate}
\item
  Check the type of the array ImJPG by using the dtype attribute:
  ImJPG.dtype The output of the dtype attribute is a numpy data type.
\item
  Find the range of colors in the image by using the amin and amax
  functions and save those elements as maxImJPG and minImJPG:

\begin{Shaded}
\begin{tcolorbox}[breakable, size=fbox, boxrule=1pt, pad at break*=1mm,colback=cellbackground, colframe=cellborder]
  \begin{Highlighting}[]
\NormalTok{maxImJPG }\OperatorTok{=}\NormalTok{ np.amax(ImJPG)}
\NormalTok{minImJPG }\OperatorTok{=}\NormalTok{ np.amin(ImJPG)}
\end{Highlighting}
\end{tcolorbox}
\end{Shaded}
\item
  Finally, display the image on the screen by using imshow:

\begin{Shaded}
\begin{tcolorbox}[breakable, size=fbox, boxrule=1pt, pad at break*=1mm,colback=cellbackground, colframe=cellborder]
  \begin{Highlighting}[]
\NormalTok{plt.imshow(ImJPG, cmap}\OperatorTok{=}\StringTok{\textquotesingle{}gray\textquotesingle{}}\NormalTok{)}
\end{Highlighting}
\end{tcolorbox}
\end{Shaded}

  If you did everything correctly, you should see the image displayed on
  your screen in a separate window.
\item
  To crop the image in numpy, we can select a subarray from the original
  array \texttt{ImJPG}. The rows and columns we want to keep from the
  original array can be specified using indexing. The following code
  will select the central part of the image leaving out 100 pixels from
  the top and bottom, and 100 pixels on the left and 70 pixels on the
  right, and display the result using matplotlib:

\begin{Shaded}
\begin{tcolorbox}[breakable, size=fbox, boxrule=1pt, pad at break*=1mm,colback=cellbackground, colframe=cellborder]
  \begin{Highlighting}[]
\NormalTok{ImJPG\_center }\OperatorTok{=}\NormalTok{ ImJPG[}\DecValTok{100}\NormalTok{:m}\OperatorTok{{-}}\DecValTok{100}\NormalTok{, }\DecValTok{100}\NormalTok{:n}\OperatorTok{{-}}\DecValTok{70}\NormalTok{]}
\ImportTok{import}\NormalTok{ matplotlib.pyplot }\ImportTok{as}\NormalTok{ plt}
\NormalTok{plt.imshow(ImJPG\_center, cmap}\OperatorTok{=}\StringTok{\textquotesingle{}gray\textquotesingle{}}\NormalTok{)}
\NormalTok{plt.show()}
\end{Highlighting}
\end{tcolorbox}
\end{Shaded}

  This will create a new figure window displaying the cropped image.
\item
  We can paste the selected part of the image into another image. To do
  this, create a zero matrix using the command:

\begin{Shaded}
\begin{tcolorbox}[breakable, size=fbox, boxrule=1pt, pad at break*=1mm,colback=cellbackground, colframe=cellborder]
  \begin{Highlighting}[]
\NormalTok{ImJPG\_border }\OperatorTok{=}\NormalTok{ np.zeros((m, n), dtype}\OperatorTok{=}\NormalTok{np.uint8)}
\end{Highlighting}
\end{tcolorbox}
\end{Shaded}

  Then paste the preselected matrix ImJPG\_center into matrix
  ImJPG\_border and display the image:

\begin{Shaded}
\begin{tcolorbox}[breakable, size=fbox, boxrule=1pt, pad at break*=1mm,colback=cellbackground, colframe=cellborder]
  \begin{Highlighting}[]
\NormalTok{ImJPG\_border[}\DecValTok{100}\NormalTok{:m}\OperatorTok{{-}}\DecValTok{100}\NormalTok{, }\DecValTok{100}\NormalTok{:n}\OperatorTok{{-}}\DecValTok{70}\NormalTok{] }\OperatorTok{=}\NormalTok{ ImJPG\_center}
\NormalTok{plt.figure()}
\NormalTok{plt.imshow(ImJPG\_border, cmap}\OperatorTok{=}\StringTok{\textquotesingle{}gray\textquotesingle{}}\NormalTok{)}
\end{Highlighting}
\end{tcolorbox}
\end{Shaded}

  Notice the use of the data type np.uint8. It is necessary to use this
  data type because by default the array will be of the type float, and
  imshow command does not work correctly with this type of array.
\item
  To flip the image vertically using NumPy, we can use the flipud
  function:

\begin{Shaded}
\begin{tcolorbox}[breakable, size=fbox, boxrule=1pt, pad at break*=1mm,colback=cellbackground, colframe=cellborder]
  \begin{Highlighting}[]
\NormalTok{ImJPG\_vertflip }\OperatorTok{=}\NormalTok{ np.flipud(ImJPG)}
\NormalTok{plt.imshow(ImJPG\_vertflip, cmap}\OperatorTok{=}\StringTok{\textquotesingle{}gray\textquotesingle{}}\NormalTok{)}
\end{Highlighting}
\end{tcolorbox}
\end{Shaded}

  This will create a new array ImJPG\_vertflip that is a vertically
  flipped version of the original array ImJPG.
\item
  To transpose the matrix using NumPy, we can use the transpose
  attribute:

\begin{Shaded}
\begin{tcolorbox}[breakable, size=fbox, boxrule=1pt, pad at break*=1mm,colback=cellbackground, colframe=cellborder]
  \begin{Highlighting}[]
\NormalTok{ImJPG\_transpose }\OperatorTok{=}\NormalTok{ ImJPG.transpose()}
\NormalTok{plt.imshow(ImJPG\_transpose, cmap}\OperatorTok{=}\StringTok{\textquotesingle{}gray\textquotesingle{}}\NormalTok{)}
\end{Highlighting}
\end{tcolorbox}
\end{Shaded}
\item
  To flip the image horizontally using NumPy, we can combine the
  transpose attribute and the fliplr function:

\begin{Shaded}
\begin{tcolorbox}[breakable, size=fbox, boxrule=1pt, pad at break*=1mm,colback=cellbackground, colframe=cellborder]
  \begin{Highlighting}[]
\NormalTok{ImJPG\_horflip }\OperatorTok{=}\NormalTok{ np.fliplr(ImJPG\_transpose).transpose()}
\NormalTok{plt.imshow(ImJPG\_horflip, cmap}\OperatorTok{=}\StringTok{\textquotesingle{}gray\textquotesingle{}}\NormalTok{)}
\end{Highlighting}
\end{tcolorbox}
\end{Shaded}
\item
  To rotate the image by 90 degrees using NumPy, we can use the rot90
  function:

\begin{Shaded}
\begin{tcolorbox}[breakable, size=fbox, boxrule=1pt, pad at break*=1mm,colback=cellbackground, colframe=cellborder]
  \begin{Highlighting}[]
\NormalTok{ImJPG90 }\OperatorTok{=}\NormalTok{ np.rot90(ImJPG)}
\NormalTok{plt.imshow(ImJPG90, cmap}\OperatorTok{=}\StringTok{\textquotesingle{}gray\textquotesingle{}}\NormalTok{)}
\end{Highlighting}
\end{tcolorbox}
\end{Shaded}
\item
  Execute the following numpy commands:

\begin{Shaded}
\begin{tcolorbox}[breakable, size=fbox, boxrule=1pt, pad at break*=1mm,colback=cellbackground, colframe=cellborder]
  \begin{Highlighting}[]
\NormalTok{ImJPG\_inv }\OperatorTok{=} \DecValTok{255}\OperatorTok{{-}}\NormalTok{ImJPG}
\NormalTok{plt.imshow(ImJPG\_inv)}
\NormalTok{plt.show()}
\end{Highlighting}
\end{tcolorbox}
\end{Shaded}

  Display the resulting image using matplotlib's imshow function in a
  new figure window. Note that the constant 255 is subtracted from the
  array ImJPG, which mathematically does not make sense. However, in
  numpy, the constant 255 is treated as an array of the same size as
  ImJPG with all the elements equal to 255. Explain what happened to the
  image.
\item
  It is also easy to lighten or darken images using matrix addition. For
  instance, the following code will create a darker image:

\begin{Shaded}
\begin{tcolorbox}[breakable, size=fbox, boxrule=1pt, pad at break*=1mm,colback=cellbackground, colframe=cellborder]
  \begin{Highlighting}[]
\NormalTok{ImJPG\_dark}\OperatorTok{=}\NormalTok{np.clip(np.array(ImJPG, dtype}\OperatorTok{=}\StringTok{\textquotesingle{}int16\textquotesingle{}}\NormalTok{) }\OperatorTok{{-}} \DecValTok{50}\NormalTok{, }\DecValTok{0}\NormalTok{, }\DecValTok{255}\NormalTok{)}
\NormalTok{plt.imshow(ImJPG\_dark,cmap}\OperatorTok{=}\StringTok{\textquotesingle{}gray\textquotesingle{}}\NormalTok{)}
\NormalTok{plt.show()}
\end{Highlighting}
\end{tcolorbox}
\end{Shaded}

  You can darken the image even more by changing the constant to a
  number larger than 50. Note that this command can technically make
  some of the elements of the array to become negative. However, because
  the ImJPG array type is int16, with the function clip those elements
  are automatically rounded to zero.
\item
  Let us create Andy Warhol style art with the image provided. To do so
  we will arrange four copies of the image into a 2×2 matrix. For the
  top left corner we will take the unaltered image. For the top right
  corner we will darken the image by 50 shades of gray. For the bottom
  left corner, lighten the image by 100 shades of gray, and finally, for
  the bottom right corner, lighten the image by 50 shades of gray. Then
  we will arrange the images together in one larger matrix using numpy's
  concatenation function. Finally, display the resulting block matrix as
  a single image using matplotlib's imshow function.

\begin{Shaded}
\begin{tcolorbox}[breakable, size=fbox, boxrule=1pt, pad at break*=1mm,colback=cellbackground, colframe=cellborder]
  \begin{Highlighting}[]
\NormalTok{im1 }\OperatorTok{=}\NormalTok{ ImJPG}
\NormalTok{im2 }\OperatorTok{=}\NormalTok{ np.clip(np.array(ImJPG, dtype}\OperatorTok{=}\StringTok{\textquotesingle{}int16\textquotesingle{}}\NormalTok{) }\OperatorTok{{-}} \DecValTok{50}\NormalTok{, }\DecValTok{0}\NormalTok{, }\DecValTok{255}\NormalTok{)}
\NormalTok{im3 }\OperatorTok{=}\NormalTok{ np.clip(np.array(ImJPG, dtype}\OperatorTok{=}\StringTok{\textquotesingle{}int16\textquotesingle{}}\NormalTok{) }\OperatorTok{+} \DecValTok{100}\NormalTok{, }\DecValTok{0}\NormalTok{, }\DecValTok{255}\NormalTok{)}
\NormalTok{im4 }\OperatorTok{=}\NormalTok{ np.clip(np.array(ImJPG, dtype}\OperatorTok{=}\StringTok{\textquotesingle{}int16\textquotesingle{}}\NormalTok{) }\OperatorTok{+} \DecValTok{50}\NormalTok{, }\DecValTok{0}\NormalTok{, }\DecValTok{255}\NormalTok{)}
\NormalTok{row1 }\OperatorTok{=}\NormalTok{ np.concatenate((im1, im2), axis}\OperatorTok{=}\DecValTok{1}\NormalTok{)}
\NormalTok{row2 }\OperatorTok{=}\NormalTok{ np.concatenate((im3, im4), axis}\OperatorTok{=}\DecValTok{1}\NormalTok{)}
\NormalTok{ImJPG\_warhol }\OperatorTok{=}\NormalTok{ np.concatenate((row1, row2), axis}\OperatorTok{=}\DecValTok{0}\NormalTok{)}
\NormalTok{plt.imshow(ImJPG\_warhol,cmap}\OperatorTok{=}\StringTok{\textquotesingle{}gray\textquotesingle{}}\NormalTok{)}
\NormalTok{plt.show()}
\end{Highlighting}
\end{tcolorbox}
\end{Shaded}
\item
  Numpy has several functions which allow one to round any number to the
  nearest integer or a decimal fraction with a given number of digits
  after the decimal point. Those functions include: floor which rounds
  the number towards negative infinity (to the smaller value), ceil
  which rounds towards positive infinity (to the larger value), round
  which rounds towards the nearest decimal or integer, and fix which
  rounds towards zero.

  A naive way to obtain black and white conversion of the image can be
  accomplished by making all the gray shades which are darker or equal
  to a medium gray (described by a value 128) to appear as a complete
  black, and all the shades of gray which are lighter than this medium
  gray to appear as white. This can be done, for instance, by using the
  code:

\begin{Shaded}
\begin{tcolorbox}[breakable, size=fbox, boxrule=1pt, pad at break*=1mm,colback=cellbackground, colframe=cellborder]
  \begin{Highlighting}[]
\NormalTok{ImJPG\_bw }\OperatorTok{=}\NormalTok{ np.uint8(}\DecValTok{255}\OperatorTok{*}\NormalTok{np.floor(ImJPG}\OperatorTok{/}\DecValTok{128}\NormalTok{))}
\NormalTok{plt.imshow(ImJPG\_bw, cmap}\OperatorTok{=}\StringTok{\textquotesingle{}gray\textquotesingle{}}\NormalTok{)}
\NormalTok{plt.show()}
\end{Highlighting}
\end{tcolorbox}
\end{Shaded}

  Note that this conversion to black and white results in a loss of many
  details of the image. There are possibilities to create black and
  white conversions without losing so many details. Also, notice the
  function np.uint8 used to convert the result back to the integer
  format.
\item
  Write code to reduce the number of shades in the image from 256 to 8
  using the round function. Save the resulting array as `ImJPG8' and
  display it in a separate window.

\begin{Shaded}
\begin{tcolorbox}[breakable, size=fbox, boxrule=1pt, pad at break*=1mm,colback=cellbackground, colframe=cellborder]
  \begin{Highlighting}[]
\NormalTok{ImJPG8 }\OperatorTok{=}\NormalTok{ np.}\BuiltInTok{round}\NormalTok{(ImJPG }\OperatorTok{/} \DecValTok{32}\NormalTok{)}
\NormalTok{plt.imshow(ImJPG8.astype(}\StringTok{\textquotesingle{}uint8\textquotesingle{}}\NormalTok{), cmap}\OperatorTok{=}\StringTok{\textquotesingle{}gray\textquotesingle{}}\NormalTok{)}
\NormalTok{plt.show()}
\end{Highlighting}
\end{tcolorbox}
\end{Shaded}
\item
  Increase the contrast of the image by changing the range of possible
  shades of gray. One way to do this is to scalar multiply the array by
  a constant. Use the following code:

\begin{Shaded}
\begin{tcolorbox}[breakable, size=fbox, boxrule=1pt, pad at break*=1mm,colback=cellbackground, colframe=cellborder]
  \begin{Highlighting}[]
\NormalTok{ImJPG\_HighContrast }\OperatorTok{=}\NormalTok{ np.clip((}\FloatTok{1.25} \OperatorTok{*}\NormalTok{ ImJPG),}\DecValTok{0}\NormalTok{,}\DecValTok{255}\NormalTok{)}
\NormalTok{plt.imshow(ImJPG\_HighContrast, cmap}\OperatorTok{=}\StringTok{\textquotesingle{}gray\textquotesingle{}}\NormalTok{)}
\NormalTok{plt.show()}
\end{Highlighting}
\end{tcolorbox}
\end{Shaded}

  Observe the result by displaying the image. You can manipulate the
  contrast by increasing or decreasing the constant (we use 1.25 in this
  case). Note that this operation may cause some elements of the array
  to become outside the 0-255 range, potentially leading to data loss.
  Save the resulting array as `HighContrast'.
\item
  Apply gamma correction to the image using the following code:

\begin{Shaded}
\begin{tcolorbox}[breakable, size=fbox, boxrule=1pt, pad at break*=1mm,colback=cellbackground, colframe=cellborder]
  \begin{Highlighting}[]
\NormalTok{ImJPG\_Gamma05 }\OperatorTok{=}\NormalTok{ np.clip(ImJPG}\OperatorTok{**} \FloatTok{0.95}\NormalTok{,}\DecValTok{0}\NormalTok{,}\DecValTok{255}\NormalTok{)}
\NormalTok{plt.imshow(ImJPG\_Gamma05, cmap}\OperatorTok{=}\StringTok{\textquotesingle{}gray\textquotesingle{}}\NormalTok{)}
\NormalTok{plt.show()}
\NormalTok{ImJPG\_Gamma15 }\OperatorTok{=}\NormalTok{ np.clip(ImJPG }\OperatorTok{**} \FloatTok{1.15}\NormalTok{,}\DecValTok{0}\NormalTok{,}\DecValTok{255}\NormalTok{)}
\NormalTok{plt.imshow(ImJPG\_Gamma15, cmap}\OperatorTok{=}\StringTok{\textquotesingle{}gray\textquotesingle{}}\NormalTok{)}
\NormalTok{plt.show()}
\end{Highlighting}
\end{tcolorbox}
\end{Shaded}

  Observe the results by displaying the images. The above code will
  produce two images, one with gamma equal to 0.95 (ImJPG\_Gamma05) and
  one with gamma equal to 1.05 (ImJPG\_Gamma15). Gamma correction is a
  nonlinear operation that can be used to adjust the brightness and
  contrast of an image.
\end{enumerate}    
\end{document}\begin{tcolorbox}[breakable, size=fbox, boxrule=1pt, pad at break*=1mm,colback=cellbackground, colframe=cellborder]
  \prompt{In}{incolor}{ }{\boxspacing}
  
