\question
{\center \bf Solving the Assignment Problem using the Hungarian Method\\}
The assignment problem is a classic optimization problem in which a set
of agents must be assigned to a set of tasks, with each agent being
assigned to exactly one task and each task being assigned to exactly one
agent. The goal is to minimize the total cost or time required for the
assignments, which can be represented as a square cost matrix, where the
element in row i and column j represents the cost of assigning agent i
to task j.

\begin{itemize}
\item
  \textbf{Task 1}: Problem Description Write a brief description of the
  Assignment Problem, including its goal, and how it can be represented
  as a square cost matrix.
\item
  \textbf{Task 2}: Research the Hungarian Method Research the Hungarian
  Method and write a brief explanation of how it works, including the
  steps involved in solving the Assignment Problem.
\item
  \textbf{Task 3}: Implement the Hungarian Method Implement the
  Hungarian Method in Python, using NumPy to represent the cost matrix
  and perform the necessary calculations.
\item
  \textbf{Task 4}: Test the Implementation Test the implementation using
  a small example problem and verify that it produces the correct
  solution.
\item
  \textbf{Task 5}: Write a Report Write a report summarizing the
  project, including the problem description, the solution approach
  using the Hungarian Method, the implementation details, and the test
  results.
\end{itemize}

\textbf{References}:

\begin{enumerate}

\item
  \href{http://biblio.univ-antananarivo.mg/pdfs/randrianatoandroFenoarisoaT\_MP\_MAST\_21.pdf}{http://biblio.univ-antananarivo.mg/pdfs/randrianatoandroFenoarisoaT\\\_MP\_MAST\_21.pdf}
\item
  \href{https://en.wikipedia.org/wiki/Hungarian\_algorithm}{https://en.wikipedia.org/wiki/Hungarian\_algorithm}
\end{enumerate}
