\documentclass[11pt]{article}

    \usepackage[breakable]{tcolorbox}
    \usepackage{parskip} % Stop auto-indenting (to mimic markdown behaviour)
    
    \usepackage{iftex}
    \ifPDFTeX
    	\usepackage[T1]{fontenc}
    	\usepackage{mathpazo}
    \else
    	\usepackage{fontspec}
    \fi

    % Basic figure setup, for now with no caption control since it's done
    % automatically by Pandoc (which extracts ![](path) syntax from Markdown).
    \usepackage{graphicx}
    % Maintain compatibility with old templates. Remove in nbconvert 6.0
    \let\Oldincludegraphics\includegraphics
    % Ensure that by default, figures have no caption (until we provide a
    % proper Figure object with a Caption API and a way to capture that
    % in the conversion process - todo).
    \usepackage{caption}
    \DeclareCaptionFormat{nocaption}{}
    \captionsetup{format=nocaption,aboveskip=0pt,belowskip=0pt}

    \usepackage{float}
    \floatplacement{figure}{H} % forces figures to be placed at the correct location
    \usepackage{xcolor} % Allow colors to be defined
    \usepackage{enumerate} % Needed for markdown enumerations to work
    \usepackage{geometry} % Used to adjust the document margins
    \usepackage{amsmath} % Equations
    \usepackage{amssymb} % Equations
    \usepackage{textcomp} % defines textquotesingle
    % Hack from http://tex.stackexchange.com/a/47451/13684:
    \AtBeginDocument{%
        \def\PYZsq{\textquotesingle}% Upright quotes in Pygmentized code
    }
    \usepackage{upquote} % Upright quotes for verbatim code
    \usepackage{eurosym} % defines \euro
    \usepackage[mathletters]{ucs} % Extended unicode (utf-8) support
    \usepackage{fancyvrb} % verbatim replacement that allows latex
    \usepackage{grffile} % extends the file name processing of package graphics 
                         % to support a larger range
    \makeatletter % fix for old versions of grffile with XeLaTeX
    \@ifpackagelater{grffile}{2019/11/01}
    {
      % Do nothing on new versions
    }
    {
      \def\Gread@@xetex#1{%
        \IfFileExists{"\Gin@base".bb}%
        {\Gread@eps{\Gin@base.bb}}%
        {\Gread@@xetex@aux#1}%
      }
    }
    \makeatother
    \usepackage[Export]{adjustbox} % Used to constrain images to a maximum size
    \adjustboxset{max size={0.9\linewidth}{0.9\paperheight}}

    % The hyperref package gives us a pdf with properly built
    % internal navigation ('pdf bookmarks' for the table of contents,
    % internal cross-reference links, web links for URLs, etc.)
    \usepackage{hyperref}
    % The default LaTeX title has an obnoxious amount of whitespace. By default,
    % titling removes some of it. It also provides customization options.
    \usepackage{titling}
    \usepackage{longtable} % longtable support required by pandoc >1.10
    \usepackage{booktabs}  % table support for pandoc > 1.12.2
    \usepackage[inline]{enumitem} % IRkernel/repr support (it uses the enumerate* environment)
    \usepackage[normalem]{ulem} % ulem is needed to support strikethroughs (\sout)
                                % normalem makes italics be italics, not underlines
    \usepackage{mathrsfs}
    

    
    % Colors for the hyperref package
    \definecolor{urlcolor}{rgb}{0,.145,.698}
    \definecolor{linkcolor}{rgb}{.71,0.21,0.01}
    \definecolor{citecolor}{rgb}{.12,.54,.11}

    % ANSI colors
    \definecolor{ansi-black}{HTML}{3E424D}
    \definecolor{ansi-black-intense}{HTML}{282C36}
    \definecolor{ansi-red}{HTML}{E75C58}
    \definecolor{ansi-red-intense}{HTML}{B22B31}
    \definecolor{ansi-green}{HTML}{00A250}
    \definecolor{ansi-green-intense}{HTML}{007427}
    \definecolor{ansi-yellow}{HTML}{DDB62B}
    \definecolor{ansi-yellow-intense}{HTML}{B27D12}
    \definecolor{ansi-blue}{HTML}{208FFB}
    \definecolor{ansi-blue-intense}{HTML}{0065CA}
    \definecolor{ansi-magenta}{HTML}{D160C4}
    \definecolor{ansi-magenta-intense}{HTML}{A03196}
    \definecolor{ansi-cyan}{HTML}{60C6C8}
    \definecolor{ansi-cyan-intense}{HTML}{258F8F}
    \definecolor{ansi-white}{HTML}{C5C1B4}
    \definecolor{ansi-white-intense}{HTML}{A1A6B2}
    \definecolor{ansi-default-inverse-fg}{HTML}{FFFFFF}
    \definecolor{ansi-default-inverse-bg}{HTML}{000000}

    % common color for the border for error outputs.
    \definecolor{outerrorbackground}{HTML}{FFDFDF}

    % commands and environments needed by pandoc snippets
    % extracted from the output of `pandoc -s`
    \providecommand{\tightlist}{%
      \setlength{\itemsep}{0pt}\setlength{\parskip}{0pt}}
    \DefineVerbatimEnvironment{Highlighting}{Verbatim}{commandchars=\\\{\}}
    % Add ',fontsize=\small' for more characters per line
    \newenvironment{Shaded}{}{}
    \newcommand{\KeywordTok}[1]{\textcolor[rgb]{0.00,0.44,0.13}{\textbf{{#1}}}}
    \newcommand{\DataTypeTok}[1]{\textcolor[rgb]{0.56,0.13,0.00}{{#1}}}
    \newcommand{\DecValTok}[1]{\textcolor[rgb]{0.25,0.63,0.44}{{#1}}}
    \newcommand{\BaseNTok}[1]{\textcolor[rgb]{0.25,0.63,0.44}{{#1}}}
    \newcommand{\FloatTok}[1]{\textcolor[rgb]{0.25,0.63,0.44}{{#1}}}
    \newcommand{\CharTok}[1]{\textcolor[rgb]{0.25,0.44,0.63}{{#1}}}
    \newcommand{\StringTok}[1]{\textcolor[rgb]{0.25,0.44,0.63}{{#1}}}
    \newcommand{\CommentTok}[1]{\textcolor[rgb]{0.38,0.63,0.69}{\textit{{#1}}}}
    \newcommand{\OtherTok}[1]{\textcolor[rgb]{0.00,0.44,0.13}{{#1}}}
    \newcommand{\AlertTok}[1]{\textcolor[rgb]{1.00,0.00,0.00}{\textbf{{#1}}}}
    \newcommand{\FunctionTok}[1]{\textcolor[rgb]{0.02,0.16,0.49}{{#1}}}
    \newcommand{\RegionMarkerTok}[1]{{#1}}
    \newcommand{\ErrorTok}[1]{\textcolor[rgb]{1.00,0.00,0.00}{\textbf{{#1}}}}
    \newcommand{\NormalTok}[1]{{#1}}
    
    % Additional commands for more recent versions of Pandoc
    \newcommand{\ConstantTok}[1]{\textcolor[rgb]{0.53,0.00,0.00}{{#1}}}
    \newcommand{\SpecialCharTok}[1]{\textcolor[rgb]{0.25,0.44,0.63}{{#1}}}
    \newcommand{\VerbatimStringTok}[1]{\textcolor[rgb]{0.25,0.44,0.63}{{#1}}}
    \newcommand{\SpecialStringTok}[1]{\textcolor[rgb]{0.73,0.40,0.53}{{#1}}}
    \newcommand{\ImportTok}[1]{{#1}}
    \newcommand{\DocumentationTok}[1]{\textcolor[rgb]{0.73,0.13,0.13}{\textit{{#1}}}}
    \newcommand{\AnnotationTok}[1]{\textcolor[rgb]{0.38,0.63,0.69}{\textbf{\textit{{#1}}}}}
    \newcommand{\CommentVarTok}[1]{\textcolor[rgb]{0.38,0.63,0.69}{\textbf{\textit{{#1}}}}}
    \newcommand{\VariableTok}[1]{\textcolor[rgb]{0.10,0.09,0.49}{{#1}}}
    \newcommand{\ControlFlowTok}[1]{\textcolor[rgb]{0.00,0.44,0.13}{\textbf{{#1}}}}
    \newcommand{\OperatorTok}[1]{\textcolor[rgb]{0.40,0.40,0.40}{{#1}}}
    \newcommand{\BuiltInTok}[1]{{#1}}
    \newcommand{\ExtensionTok}[1]{{#1}}
    \newcommand{\PreprocessorTok}[1]{\textcolor[rgb]{0.74,0.48,0.00}{{#1}}}
    \newcommand{\AttributeTok}[1]{\textcolor[rgb]{0.49,0.56,0.16}{{#1}}}
    \newcommand{\InformationTok}[1]{\textcolor[rgb]{0.38,0.63,0.69}{\textbf{\textit{{#1}}}}}
    \newcommand{\WarningTok}[1]{\textcolor[rgb]{0.38,0.63,0.69}{\textbf{\textit{{#1}}}}}
    
    
    % Define a nice break command that doesn't care if a line doesn't already
    % exist.
    \def\br{\hspace*{\fill} \\* }
    % Math Jax compatibility definitions
    \def\gt{>}
    \def\lt{<}
    \let\Oldtex\TeX
    \let\Oldlatex\LaTeX
    \renewcommand{\TeX}{\textrm{\Oldtex}}
    \renewcommand{\LaTeX}{\textrm{\Oldlatex}}
    % Document parameters
    % Document title
    \title{Files and Excepitons}
    
    
    
    
    
% Pygments definitions
\makeatletter
\def\PY@reset{\let\PY@it=\relax \let\PY@bf=\relax%
    \let\PY@ul=\relax \let\PY@tc=\relax%
    \let\PY@bc=\relax \let\PY@ff=\relax}
\def\PY@tok#1{\csname PY@tok@#1\endcsname}
\def\PY@toks#1+{\ifx\relax#1\empty\else%
    \PY@tok{#1}\expandafter\PY@toks\fi}
\def\PY@do#1{\PY@bc{\PY@tc{\PY@ul{%
    \PY@it{\PY@bf{\PY@ff{#1}}}}}}}
\def\PY#1#2{\PY@reset\PY@toks#1+\relax+\PY@do{#2}}

\@namedef{PY@tok@w}{\def\PY@tc##1{\textcolor[rgb]{0.73,0.73,0.73}{##1}}}
\@namedef{PY@tok@c}{\let\PY@it=\textit\def\PY@tc##1{\textcolor[rgb]{0.25,0.50,0.50}{##1}}}
\@namedef{PY@tok@cp}{\def\PY@tc##1{\textcolor[rgb]{0.74,0.48,0.00}{##1}}}
\@namedef{PY@tok@k}{\let\PY@bf=\textbf\def\PY@tc##1{\textcolor[rgb]{0.00,0.50,0.00}{##1}}}
\@namedef{PY@tok@kp}{\def\PY@tc##1{\textcolor[rgb]{0.00,0.50,0.00}{##1}}}
\@namedef{PY@tok@kt}{\def\PY@tc##1{\textcolor[rgb]{0.69,0.00,0.25}{##1}}}
\@namedef{PY@tok@o}{\def\PY@tc##1{\textcolor[rgb]{0.40,0.40,0.40}{##1}}}
\@namedef{PY@tok@ow}{\let\PY@bf=\textbf\def\PY@tc##1{\textcolor[rgb]{0.67,0.13,1.00}{##1}}}
\@namedef{PY@tok@nb}{\def\PY@tc##1{\textcolor[rgb]{0.00,0.50,0.00}{##1}}}
\@namedef{PY@tok@nf}{\def\PY@tc##1{\textcolor[rgb]{0.00,0.00,1.00}{##1}}}
\@namedef{PY@tok@nc}{\let\PY@bf=\textbf\def\PY@tc##1{\textcolor[rgb]{0.00,0.00,1.00}{##1}}}
\@namedef{PY@tok@nn}{\let\PY@bf=\textbf\def\PY@tc##1{\textcolor[rgb]{0.00,0.00,1.00}{##1}}}
\@namedef{PY@tok@ne}{\let\PY@bf=\textbf\def\PY@tc##1{\textcolor[rgb]{0.82,0.25,0.23}{##1}}}
\@namedef{PY@tok@nv}{\def\PY@tc##1{\textcolor[rgb]{0.10,0.09,0.49}{##1}}}
\@namedef{PY@tok@no}{\def\PY@tc##1{\textcolor[rgb]{0.53,0.00,0.00}{##1}}}
\@namedef{PY@tok@nl}{\def\PY@tc##1{\textcolor[rgb]{0.63,0.63,0.00}{##1}}}
\@namedef{PY@tok@ni}{\let\PY@bf=\textbf\def\PY@tc##1{\textcolor[rgb]{0.60,0.60,0.60}{##1}}}
\@namedef{PY@tok@na}{\def\PY@tc##1{\textcolor[rgb]{0.49,0.56,0.16}{##1}}}
\@namedef{PY@tok@nt}{\let\PY@bf=\textbf\def\PY@tc##1{\textcolor[rgb]{0.00,0.50,0.00}{##1}}}
\@namedef{PY@tok@nd}{\def\PY@tc##1{\textcolor[rgb]{0.67,0.13,1.00}{##1}}}
\@namedef{PY@tok@s}{\def\PY@tc##1{\textcolor[rgb]{0.73,0.13,0.13}{##1}}}
\@namedef{PY@tok@sd}{\let\PY@it=\textit\def\PY@tc##1{\textcolor[rgb]{0.73,0.13,0.13}{##1}}}
\@namedef{PY@tok@si}{\let\PY@bf=\textbf\def\PY@tc##1{\textcolor[rgb]{0.73,0.40,0.53}{##1}}}
\@namedef{PY@tok@se}{\let\PY@bf=\textbf\def\PY@tc##1{\textcolor[rgb]{0.73,0.40,0.13}{##1}}}
\@namedef{PY@tok@sr}{\def\PY@tc##1{\textcolor[rgb]{0.73,0.40,0.53}{##1}}}
\@namedef{PY@tok@ss}{\def\PY@tc##1{\textcolor[rgb]{0.10,0.09,0.49}{##1}}}
\@namedef{PY@tok@sx}{\def\PY@tc##1{\textcolor[rgb]{0.00,0.50,0.00}{##1}}}
\@namedef{PY@tok@m}{\def\PY@tc##1{\textcolor[rgb]{0.40,0.40,0.40}{##1}}}
\@namedef{PY@tok@gh}{\let\PY@bf=\textbf\def\PY@tc##1{\textcolor[rgb]{0.00,0.00,0.50}{##1}}}
\@namedef{PY@tok@gu}{\let\PY@bf=\textbf\def\PY@tc##1{\textcolor[rgb]{0.50,0.00,0.50}{##1}}}
\@namedef{PY@tok@gd}{\def\PY@tc##1{\textcolor[rgb]{0.63,0.00,0.00}{##1}}}
\@namedef{PY@tok@gi}{\def\PY@tc##1{\textcolor[rgb]{0.00,0.63,0.00}{##1}}}
\@namedef{PY@tok@gr}{\def\PY@tc##1{\textcolor[rgb]{1.00,0.00,0.00}{##1}}}
\@namedef{PY@tok@ge}{\let\PY@it=\textit}
\@namedef{PY@tok@gs}{\let\PY@bf=\textbf}
\@namedef{PY@tok@gp}{\let\PY@bf=\textbf\def\PY@tc##1{\textcolor[rgb]{0.00,0.00,0.50}{##1}}}
\@namedef{PY@tok@go}{\def\PY@tc##1{\textcolor[rgb]{0.53,0.53,0.53}{##1}}}
\@namedef{PY@tok@gt}{\def\PY@tc##1{\textcolor[rgb]{0.00,0.27,0.87}{##1}}}
\@namedef{PY@tok@err}{\def\PY@bc##1{{\setlength{\fboxsep}{\string -\fboxrule}\fcolorbox[rgb]{1.00,0.00,0.00}{1,1,1}{\strut ##1}}}}
\@namedef{PY@tok@kc}{\let\PY@bf=\textbf\def\PY@tc##1{\textcolor[rgb]{0.00,0.50,0.00}{##1}}}
\@namedef{PY@tok@kd}{\let\PY@bf=\textbf\def\PY@tc##1{\textcolor[rgb]{0.00,0.50,0.00}{##1}}}
\@namedef{PY@tok@kn}{\let\PY@bf=\textbf\def\PY@tc##1{\textcolor[rgb]{0.00,0.50,0.00}{##1}}}
\@namedef{PY@tok@kr}{\let\PY@bf=\textbf\def\PY@tc##1{\textcolor[rgb]{0.00,0.50,0.00}{##1}}}
\@namedef{PY@tok@bp}{\def\PY@tc##1{\textcolor[rgb]{0.00,0.50,0.00}{##1}}}
\@namedef{PY@tok@fm}{\def\PY@tc##1{\textcolor[rgb]{0.00,0.00,1.00}{##1}}}
\@namedef{PY@tok@vc}{\def\PY@tc##1{\textcolor[rgb]{0.10,0.09,0.49}{##1}}}
\@namedef{PY@tok@vg}{\def\PY@tc##1{\textcolor[rgb]{0.10,0.09,0.49}{##1}}}
\@namedef{PY@tok@vi}{\def\PY@tc##1{\textcolor[rgb]{0.10,0.09,0.49}{##1}}}
\@namedef{PY@tok@vm}{\def\PY@tc##1{\textcolor[rgb]{0.10,0.09,0.49}{##1}}}
\@namedef{PY@tok@sa}{\def\PY@tc##1{\textcolor[rgb]{0.73,0.13,0.13}{##1}}}
\@namedef{PY@tok@sb}{\def\PY@tc##1{\textcolor[rgb]{0.73,0.13,0.13}{##1}}}
\@namedef{PY@tok@sc}{\def\PY@tc##1{\textcolor[rgb]{0.73,0.13,0.13}{##1}}}
\@namedef{PY@tok@dl}{\def\PY@tc##1{\textcolor[rgb]{0.73,0.13,0.13}{##1}}}
\@namedef{PY@tok@s2}{\def\PY@tc##1{\textcolor[rgb]{0.73,0.13,0.13}{##1}}}
\@namedef{PY@tok@sh}{\def\PY@tc##1{\textcolor[rgb]{0.73,0.13,0.13}{##1}}}
\@namedef{PY@tok@s1}{\def\PY@tc##1{\textcolor[rgb]{0.73,0.13,0.13}{##1}}}
\@namedef{PY@tok@mb}{\def\PY@tc##1{\textcolor[rgb]{0.40,0.40,0.40}{##1}}}
\@namedef{PY@tok@mf}{\def\PY@tc##1{\textcolor[rgb]{0.40,0.40,0.40}{##1}}}
\@namedef{PY@tok@mh}{\def\PY@tc##1{\textcolor[rgb]{0.40,0.40,0.40}{##1}}}
\@namedef{PY@tok@mi}{\def\PY@tc##1{\textcolor[rgb]{0.40,0.40,0.40}{##1}}}
\@namedef{PY@tok@il}{\def\PY@tc##1{\textcolor[rgb]{0.40,0.40,0.40}{##1}}}
\@namedef{PY@tok@mo}{\def\PY@tc##1{\textcolor[rgb]{0.40,0.40,0.40}{##1}}}
\@namedef{PY@tok@ch}{\let\PY@it=\textit\def\PY@tc##1{\textcolor[rgb]{0.25,0.50,0.50}{##1}}}
\@namedef{PY@tok@cm}{\let\PY@it=\textit\def\PY@tc##1{\textcolor[rgb]{0.25,0.50,0.50}{##1}}}
\@namedef{PY@tok@cpf}{\let\PY@it=\textit\def\PY@tc##1{\textcolor[rgb]{0.25,0.50,0.50}{##1}}}
\@namedef{PY@tok@c1}{\let\PY@it=\textit\def\PY@tc##1{\textcolor[rgb]{0.25,0.50,0.50}{##1}}}
\@namedef{PY@tok@cs}{\let\PY@it=\textit\def\PY@tc##1{\textcolor[rgb]{0.25,0.50,0.50}{##1}}}

\def\PYZbs{\char`\\}
\def\PYZus{\char`\_}
\def\PYZob{\char`\{}
\def\PYZcb{\char`\}}
\def\PYZca{\char`\^}
\def\PYZam{\char`\&}
\def\PYZlt{\char`\<}
\def\PYZgt{\char`\>}
\def\PYZsh{\char`\#}
\def\PYZpc{\char`\%}
\def\PYZdl{\char`\$}
\def\PYZhy{\char`\-}
\def\PYZsq{\char`\'}
\def\PYZdq{\char`\"}
\def\PYZti{\char`\~}
% for compatibility with earlier versions
\def\PYZat{@}
\def\PYZlb{[}
\def\PYZrb{]}
\makeatother


    % For linebreaks inside Verbatim environment from package fancyvrb. 
    \makeatletter
        \newbox\Wrappedcontinuationbox 
        \newbox\Wrappedvisiblespacebox 
        \newcommand*\Wrappedvisiblespace {\textcolor{red}{\textvisiblespace}} 
        \newcommand*\Wrappedcontinuationsymbol {\textcolor{red}{\llap{\tiny$\m@th\hookrightarrow$}}} 
        \newcommand*\Wrappedcontinuationindent {3ex } 
        \newcommand*\Wrappedafterbreak {\kern\Wrappedcontinuationindent\copy\Wrappedcontinuationbox} 
        % Take advantage of the already applied Pygments mark-up to insert 
        % potential linebreaks for TeX processing. 
        %        {, <, #, %, $, ' and ": go to next line. 
        %        _, }, ^, &, >, - and ~: stay at end of broken line. 
        % Use of \textquotesingle for straight quote. 
        \newcommand*\Wrappedbreaksatspecials {% 
            \def\PYGZus{\discretionary{\char`\_}{\Wrappedafterbreak}{\char`\_}}% 
            \def\PYGZob{\discretionary{}{\Wrappedafterbreak\char`\{}{\char`\{}}% 
            \def\PYGZcb{\discretionary{\char`\}}{\Wrappedafterbreak}{\char`\}}}% 
            \def\PYGZca{\discretionary{\char`\^}{\Wrappedafterbreak}{\char`\^}}% 
            \def\PYGZam{\discretionary{\char`\&}{\Wrappedafterbreak}{\char`\&}}% 
            \def\PYGZlt{\discretionary{}{\Wrappedafterbreak\char`\<}{\char`\<}}% 
            \def\PYGZgt{\discretionary{\char`\>}{\Wrappedafterbreak}{\char`\>}}% 
            \def\PYGZsh{\discretionary{}{\Wrappedafterbreak\char`\#}{\char`\#}}% 
            \def\PYGZpc{\discretionary{}{\Wrappedafterbreak\char`\%}{\char`\%}}% 
            \def\PYGZdl{\discretionary{}{\Wrappedafterbreak\char`\$}{\char`\$}}% 
            \def\PYGZhy{\discretionary{\char`\-}{\Wrappedafterbreak}{\char`\-}}% 
            \def\PYGZsq{\discretionary{}{\Wrappedafterbreak\textquotesingle}{\textquotesingle}}% 
            \def\PYGZdq{\discretionary{}{\Wrappedafterbreak\char`\"}{\char`\"}}% 
            \def\PYGZti{\discretionary{\char`\~}{\Wrappedafterbreak}{\char`\~}}% 
        } 
        % Some characters . , ; ? ! / are not pygmentized. 
        % This macro makes them "active" and they will insert potential linebreaks 
        \newcommand*\Wrappedbreaksatpunct {% 
            \lccode`\~`\.\lowercase{\def~}{\discretionary{\hbox{\char`\.}}{\Wrappedafterbreak}{\hbox{\char`\.}}}% 
            \lccode`\~`\,\lowercase{\def~}{\discretionary{\hbox{\char`\,}}{\Wrappedafterbreak}{\hbox{\char`\,}}}% 
            \lccode`\~`\;\lowercase{\def~}{\discretionary{\hbox{\char`\;}}{\Wrappedafterbreak}{\hbox{\char`\;}}}% 
            \lccode`\~`\:\lowercase{\def~}{\discretionary{\hbox{\char`\:}}{\Wrappedafterbreak}{\hbox{\char`\:}}}% 
            \lccode`\~`\?\lowercase{\def~}{\discretionary{\hbox{\char`\?}}{\Wrappedafterbreak}{\hbox{\char`\?}}}% 
            \lccode`\~`\!\lowercase{\def~}{\discretionary{\hbox{\char`\!}}{\Wrappedafterbreak}{\hbox{\char`\!}}}% 
            \lccode`\~`\/\lowercase{\def~}{\discretionary{\hbox{\char`\/}}{\Wrappedafterbreak}{\hbox{\char`\/}}}% 
            \catcode`\.\active
            \catcode`\,\active 
            \catcode`\;\active
            \catcode`\:\active
            \catcode`\?\active
            \catcode`\!\active
            \catcode`\/\active 
            \lccode`\~`\~ 	
        }
    \makeatother

    \let\OriginalVerbatim=\Verbatim
    \makeatletter
    \renewcommand{\Verbatim}[1][1]{%
        %\parskip\z@skip
        \sbox\Wrappedcontinuationbox {\Wrappedcontinuationsymbol}%
        \sbox\Wrappedvisiblespacebox {\FV@SetupFont\Wrappedvisiblespace}%
        \def\FancyVerbFormatLine ##1{\hsize\linewidth
            \vtop{\raggedright\hyphenpenalty\z@\exhyphenpenalty\z@
                \doublehyphendemerits\z@\finalhyphendemerits\z@
                \strut ##1\strut}%
        }%
        % If the linebreak is at a space, the latter will be displayed as visible
        % space at end of first line, and a continuation symbol starts next line.
        % Stretch/shrink are however usually zero for typewriter font.
        \def\FV@Space {%
            \nobreak\hskip\z@ plus\fontdimen3\font minus\fontdimen4\font
            \discretionary{\copy\Wrappedvisiblespacebox}{\Wrappedafterbreak}
            {\kern\fontdimen2\font}%
        }%
        
        % Allow breaks at special characters using \PYG... macros.
        \Wrappedbreaksatspecials
        % Breaks at punctuation characters . , ; ? ! and / need catcode=\active 	
        \OriginalVerbatim[#1,codes*=\Wrappedbreaksatpunct]%
    }
    \makeatother

    % Exact colors from NB
    \definecolor{incolor}{HTML}{303F9F}
    \definecolor{outcolor}{HTML}{D84315}
    \definecolor{cellborder}{HTML}{CFCFCF}
    \definecolor{cellbackground}{HTML}{F7F7F7}
    
    % prompt
    \makeatletter
    \newcommand{\boxspacing}{\kern\kvtcb@left@rule\kern\kvtcb@boxsep}
    \makeatother
    \newcommand{\prompt}[4]{
        {\ttfamily\llap{{\color{#2}[#3]:\hspace{3pt}#4}}\vspace{-\baselineskip}}
    }
    

    
    % Prevent overflowing lines due to hard-to-break entities
    \sloppy 
    % Setup hyperref package
    \hypersetup{
      breaklinks=true,  % so long urls are correctly broken across lines
      colorlinks=true,
      urlcolor=urlcolor,
      linkcolor=linkcolor,
      citecolor=citecolor,
      }
    % Slightly bigger margins than the latex defaults
    
    \geometry{verbose,tmargin=1in,bmargin=1in,lmargin=1in,rmargin=1in}
    
    

\begin{document}
    \date{}
    \maketitle
    
    

    
    \hypertarget{whats-a-file}{%
\section{What's a file?}\label{whats-a-file}}

\begin{itemize}
\tightlist
\item
  A file is a sequence of bytes stored on a secondary memory device,
  such as a disk drive.
\item
  Files can be text documents, spreadsheets, HTML files, Python modules,
  executable applications, images, or audio files.
\item
  Text files contain a sequence of characters that are encoded using
  some encoding, while binary files are just a sequence of bytes and
  have no encoding.
\end{itemize}

\hypertarget{whats-a-file-system}{%
\section{What's a file system?}\label{whats-a-file-system}}

\begin{itemize}
\tightlist
\item
  The file system is the component of a computer system that organizes
  files and provides ways to create, access, and modify files.
\item
  The file system organizes files and folders into a tree structure,
  with the root directory at the top.
\item
  Folders can contain other folders and regular files.
\item
  Every file and folder in a file system has a name, but a name is not
  sufficient to locate a file efficiently. - - Instead, every file can
  be specified using a pathname.

  \begin{itemize}
  \tightlist
  \item
    The absolute pathname of a file consists of the sequence of folders,
    starting from the root directory, that must be traversed to get to
    the file.
  \item
    The relative pathname of a file is the sequence of directories that
    must be traversed, starting from the current working directory, to
    get to the file.
  \end{itemize}
\item
  The double-period notation (\texttt{..}) is used to refer to the
  parent folder, which is the folder containing the current working
  directory.
\end{itemize}

\hypertarget{processing-a-file-in-python}{%
\section{Processing a file in
python}\label{processing-a-file-in-python}}

Processing a file consists of these three steps: 1. Opening a file for
reading or writing 2. Reading from the file and/or writing to the file 3.
Closing the file \#\# 1. Opening a file Python introduces the
\texttt{file} object in order to perform some file operations. Python
has a built-in \texttt{open()} function to open files from the
directory. Two arguments that are mainly needed by the \texttt{open()}
function are:

\begin{itemize}
\tightlist
\item
  \textbf{file name} or \textbf{file path}: is a string that specifies
  the name of the file to be accessed.
\item
  \textbf{access\_mode}: parameter determines the mode in which the file
  will be opened, such as read, write, or append. We can also specify
  whether a file should be opened in the text mode or the binary mode
  (deals with bites in the case of non-text file). In the
\end{itemize}
$$
\begin{array}{l|l}
\text{Access mode}&\text{Description}\\
\hline
`r' & \text{Reading mode, file pointer is placed at the beginning of the file
(default mode).} \\ \hline
`rb' &\text{ Reading mode in binary format, file pointer is placed at the}\\ 
&\text{ beginning of the file. }\\ \hline
`r+' &\text{ Reading and writing mode, file pointer is placed at the beginning}\\ 
&\text{of the file. }\\ \hline
`rb+' &\text{ Reading and writing mode in binary format, file pointer is}\\ 
&\text{placed at the beginning of the file. }\\ \hline
`w' &\text{ Writing only mode, overwrites the existing file or creates a new}\\ 
&\text{file. }\\ \hline
`wb' &\text{ Writing only mode in binary format, overwrites the existing file}\\ 
&\text{or creates a new file. }\\ \hline
`w+' &\text{ Reading and writing mode, overwrites the existing file or creates}\\ 
&\text{a new file. }\\ \hline
`wb+' &\text{ Reading and writing mode in binary format, overwrites the}\\ 
&\text{existing file or creates a new file. }\\ \hline
`a' &\text{ Appending mode, file pointer is placed at the end of the file.}\\ \hline 
&\text{If the file does not exist, a new file is created. }\\ \hline
`ab' &\text{ Appending mode in binary format, file pointer is placed at the}\\ 
&\text{end of the file. If the file does not exist, a new file is created. }\\ \hline
`a+' &\text{ Appending and reading mode, file pointer is placed at the end of}\\ 
&\text{the file. If the file does not exist, a new file is created. }\\ \hline
`ab+' &\text{ Appending and reading mode in binary format, file pointer is}\\ 
&\text{placed at the end of the file. If the file does not exist, a new file is}\\ 
&\text{created. }\\
\end{array}
$$
\hypertarget{the-syntax-for-opening-a-file}{%
\subsubsection{The syntax for opening a
file}\label{the-syntax-for-opening-a-file}}

The syntax for opening a file is:

\begin{verbatim}
file_object = open(file_name [, access_mode])
\end{verbatim}

\hypertarget{examples}{%
\subsubsection{Examples:}\label{examples}}

\begin{Shaded}
\begin{Highlighting}[]
\NormalTok{f }\OperatorTok{=} \BuiltInTok{open}\NormalTok{(}\StringTok{"test.txt"}\NormalTok{) }\CommentTok{\# opens in r mode(reading only)}
\NormalTok{f }\OperatorTok{=} \BuiltInTok{open}\NormalTok{(}\StringTok{"test.txt"}\NormalTok{,}\StringTok{\textquotesingle{}w\textquotesingle{}}\NormalTok{) }\CommentTok{\# opens in w mode(writing only)}
\NormalTok{f }\OperatorTok{=} \BuiltInTok{open}\NormalTok{(}\StringTok{"image.bmp"}\NormalTok{,}\StringTok{\textquotesingle{}rb+\textquotesingle{}}\NormalTok{) }\CommentTok{\# read and write in binary mode}
\end{Highlighting}
\end{Shaded}

\hypertarget{reading-from-the-uxfb01le}{%
\subsection{2. Reading from the file}\label{reading-from-the-uxfb01le}}

To read from a file in Python, you need to open the file in read mode.
Once the file is opened, you can read its contents using various methods
such as:

Method \textbar Explanation
:-----------------\textbar:-------------------------------------------------------
\texttt{fname.read(n)} \textbar{} Read n characters from the file infile
or until the end of the file is reached, and return characters read as a
string \texttt{fname.read()} \textbar{} Read characters from file infile
until the end of the file and return characters read as a string
\texttt{fname.readline()} \textbar{} Read file infile until (and
including) the new line character or until end of file, whichever is
first, and return characters read as a string
\texttt{fname.readlines()}\textbar{} Read file infile until the end of
the file and return the characters read as a list lines

\hypertarget{examples-1}{%
\subsubsection{Examples}\label{examples-1}}

\begin{Shaded}
\begin{Highlighting}[]
\NormalTok{f}\OperatorTok{=}\BuiltInTok{open}\NormalTok{(}\StringTok{"example.txt"}\NormalTok{)}
\BuiltInTok{print}\NormalTok{(f.read(}\DecValTok{5}\NormalTok{))}
\CommentTok{\#print(f.read())}
\BuiltInTok{print}\NormalTok{(f.readline())}
\BuiltInTok{print}\NormalTok{(f.readlines())}
\CommentTok{\#f.close()}
\CommentTok{\#f=open("example.txt")}
\CommentTok{\#for line in f:}
\CommentTok{\#        print(line)}
\end{Highlighting}
\end{Shaded}

\hypertarget{writing-to-the-uxfb01le}{%
\subsection{3. Writing to the file}\label{writing-to-the-uxfb01le}}

To write to a file in Python, you need to open the file in write mode
using the built-in \texttt{open()} function with the
\texttt{\textquotesingle{}w\textquotesingle{}} or
\texttt{\textquotesingle{}wb\textquotesingle{}} mode depending on the
file's content type. Here's an example:

\begin{Shaded}
\begin{Highlighting}[]
\BuiltInTok{file} \OperatorTok{=} \BuiltInTok{open}\NormalTok{(}\StringTok{\textquotesingle{}filename.txt\textquotesingle{}}\NormalTok{, }\StringTok{\textquotesingle{}w\textquotesingle{}}\NormalTok{)}
\BuiltInTok{file}\NormalTok{.write(}\StringTok{\textquotesingle{}Hello, world!\textquotesingle{}}\NormalTok{)}
\BuiltInTok{file}\NormalTok{.close()}
\end{Highlighting}
\end{Shaded}

This code opens a file named
\texttt{\textquotesingle{}filename.txt\textquotesingle{}} in write mode
using the \texttt{\textquotesingle{}w\textquotesingle{}} mode. Then, it
writes the string
\texttt{\textquotesingle{}Hello,\ world!\textquotesingle{}} to the file
using the \texttt{write()} method. Finally, the file is closed using the
\texttt{close()} method.

If you want to write multiple lines to the file, you can use the newline
character \texttt{\textquotesingle{}\textbackslash{}n\textquotesingle{}}
to separate the lines, like this:

\begin{Shaded}
\begin{Highlighting}[]
\BuiltInTok{file} \OperatorTok{=} \BuiltInTok{open}\NormalTok{(}\StringTok{\textquotesingle{}filename.txt\textquotesingle{}}\NormalTok{, }\StringTok{\textquotesingle{}w\textquotesingle{}}\NormalTok{)}
\BuiltInTok{file}\NormalTok{.write(}\StringTok{\textquotesingle{}Line 1}\CharTok{\textbackslash{}n}\StringTok{\textquotesingle{}}\NormalTok{)}
\BuiltInTok{file}\NormalTok{.write(}\StringTok{\textquotesingle{}Line 2}\CharTok{\textbackslash{}n}\StringTok{\textquotesingle{}}\NormalTok{)}
\BuiltInTok{file}\NormalTok{.close()}
\end{Highlighting}
\end{Shaded}

This code will write two lines to the file, each on a separate line.

Note that using the \texttt{\textquotesingle{}w\textquotesingle{}} mode
will overwrite any existing content in the file. If you want to append
to an existing file, you can use the
\texttt{\textquotesingle{}a\textquotesingle{}} or
\texttt{\textquotesingle{}ab\textquotesingle{}} mode instead of
\texttt{\textquotesingle{}w\textquotesingle{}}.

We can also write many lines using \texttt{writelines(list\_of\_line)}
method.

\hypertarget{line-endings}{%
\subsubsection{Line Endings}\label{line-endings}}

In Python, the new line character is represented by the escape sequence
\texttt{\textbackslash{}n} . However text file formats are platform
dependent, and different operating systems use a different byte sequence
to encode a new line: - MS Windows uses the
\texttt{\textbackslash{}r\textbackslash{}n} 2-character sequence. -
Linux/UNIX and Mac OS X use the \texttt{\textbackslash{}n} character. -
Mac OS up to version 9 uses the \texttt{\textbackslash{}r} character.
\#\# 4. Closing files In Python, it is important to close files after
you have finished using them.

This releases the resources that were used by the file, and ensures that
any data that was buffered in memory is written to the file. To close a
file, you can use the \texttt{close()} method of the file object.

For example, if you opened a file using
\texttt{open(\textquotesingle{}example.txt\textquotesingle{},\ \textquotesingle{}r\textquotesingle{})},
you can close it using \texttt{close()} method:

\begin{Shaded}
\begin{Highlighting}[]
\NormalTok{file\_object }\OperatorTok{=} \BuiltInTok{open}\NormalTok{(}\StringTok{\textquotesingle{}example.txt\textquotesingle{}}\NormalTok{, }\StringTok{\textquotesingle{}r\textquotesingle{}}\NormalTok{)}
\CommentTok{\# Do some operations with the file}
\NormalTok{file\_object.close()}
\end{Highlighting}
\end{Shaded}

It is recommended to always close files after you are done working with
them, even though Python automatically closes the file when the program
terminates or the file object is destroyed. \# Attributes of a file
object in Python A file object has several attributes that provide
information about the file, including:

$$\begin{array}[]{|l|l|}
    \hline
\texttt{Attribute} & \texttt{Explanation} \\
\hline
\texttt{name} &\texttt{the name of the file}\\
\hline
\texttt{mode} &\texttt{the mode in which the file is opened}\\
\hline
\texttt{closed} &\texttt{a boolean indicating whether the file is closed or
not}\\
\hline
\texttt{encoding} &\texttt{the encoding used to read or write the file (only
applicable for text files)}\\
\hline
\end{array}$$

    \begin{tcolorbox}[breakable, size=fbox, boxrule=1pt, pad at break*=1mm,colback=cellbackground, colframe=cellborder]
\prompt{In}{incolor}{ }{\boxspacing}
\begin{Verbatim}[commandchars=\\\{\}]
\PY{n}{f}\PY{o}{=}\PY{n+nb}{open}\PY{p}{(}\PY{l+s+s2}{\PYZdq{}}\PY{l+s+s2}{example.txt}\PY{l+s+s2}{\PYZdq{}}\PY{p}{)}
\PY{n+nb}{print}\PY{p}{(}\PY{n}{f}\PY{o}{.}\PY{n}{name}\PY{p}{)}
\PY{n+nb}{print}\PY{p}{(}\PY{n}{f}\PY{o}{.}\PY{n}{mode}\PY{p}{)}
\PY{n+nb}{print}\PY{p}{(}\PY{n}{f}\PY{o}{.}\PY{n}{closed}\PY{p}{)}
\PY{n+nb}{print}\PY{p}{(}\PY{n}{f}\PY{o}{.}\PY{n}{encoding}\PY{p}{)}
\end{Verbatim}
\end{tcolorbox}

    \hypertarget{file-positions}{%
\section{File Positions}\label{file-positions}}

The \texttt{tell()} method of a file object returns the current position
of the file pointer in bytes from the beginning of the file. It tells
you where the next read or write operation will occur.

For example:

\begin{Shaded}
\begin{Highlighting}[]
\BuiltInTok{file} \OperatorTok{=} \BuiltInTok{open}\NormalTok{(}\StringTok{"example.txt"}\NormalTok{, }\StringTok{"r"}\NormalTok{)}
\BuiltInTok{print}\NormalTok{(}\BuiltInTok{file}\NormalTok{.tell()) }\CommentTok{\# prints the current position of the file pointer}
\BuiltInTok{file}\NormalTok{.close()}
\end{Highlighting}
\end{Shaded}

The \texttt{seek(offset{[},\ from{]})} method changes the current file
position to the position specified by offset. The from argument is
optional and specifies the reference position from where the bytes are
to be moved. If from is not specified, it defaults to \texttt{0} (the
beginning of the file).

For example:

\begin{Shaded}
\begin{Highlighting}[]

\BuiltInTok{file} \OperatorTok{=} \BuiltInTok{open}\NormalTok{(}\StringTok{"example.txt"}\NormalTok{, }\StringTok{"r"}\NormalTok{)}
\BuiltInTok{file}\NormalTok{.seek(}\DecValTok{10}\NormalTok{) }\CommentTok{\# move the file pointer to the 10th byte from the beginning of the file}
\BuiltInTok{print}\NormalTok{(}\BuiltInTok{file}\NormalTok{.tell)}
\BuiltInTok{file}\NormalTok{.close()}
\CommentTok{\#Binary files}
\BuiltInTok{file} \OperatorTok{=} \BuiltInTok{open}\NormalTok{(}\StringTok{"example.txt"}\NormalTok{, }\StringTok{"rb"}\NormalTok{)}
\BuiltInTok{file}\NormalTok{.read(}\DecValTok{10}\NormalTok{)}
\BuiltInTok{print}\NormalTok{(}\BuiltInTok{file}\NormalTok{.tell())}
\BuiltInTok{file}\NormalTok{.seek(}\DecValTok{10}\NormalTok{,}\DecValTok{1}\NormalTok{) }\CommentTok{\# move the file pointer to the 10th byte from the current position of the file}
\BuiltInTok{print}\NormalTok{(}\BuiltInTok{file}\NormalTok{.tell())}
\BuiltInTok{file}\NormalTok{.read()}
\BuiltInTok{print}\NormalTok{(}\BuiltInTok{file}\NormalTok{.tell())}
\BuiltInTok{file}\NormalTok{.seek(}\OperatorTok{{-}}\DecValTok{20}\NormalTok{,}\DecValTok{2}\NormalTok{) }\CommentTok{\# move the file pointer 20 bytes backwards from the end of the file}
\BuiltInTok{print}\NormalTok{(}\BuiltInTok{file}\NormalTok{.tell())}
\BuiltInTok{file}\NormalTok{.close()}
\end{Highlighting}
\end{Shaded}

\hypertarget{renaming-and-deleting-files}{%
\section{Renaming and Deleting
Files}\label{renaming-and-deleting-files}}

Python's os module provides functions to perform file-related operations
such as renaming and deleting files. To use these functions, you need to
import the os module first.

Example of renaming a file using \texttt{os.rename()}:

\begin{Shaded}
\begin{Highlighting}[]
\ImportTok{import}\NormalTok{ os}

\CommentTok{\# current file name}
\NormalTok{current\_name }\OperatorTok{=} \StringTok{"old\_file.txt"}

\CommentTok{\# new file name}
\NormalTok{new\_name }\OperatorTok{=} \StringTok{"new\_file.txt"}
\CommentTok{\# renaming the file}
\NormalTok{os.rename(current\_name, new\_name)}
\end{Highlighting}
\end{Shaded}

Example of deleting a file using \texttt{os.remove()}:

\begin{Shaded}
\begin{Highlighting}[]
\ImportTok{import}\NormalTok{ os}

\CommentTok{\# file name to be deleted}
\NormalTok{file\_name }\OperatorTok{=} \StringTok{"file\_to\_delete.txt"}

\CommentTok{\# deleting the file}
\NormalTok{os.remove(file\_name)}
\end{Highlighting}
\end{Shaded}

\hypertarget{directories-in-python}{%
\section{Directories in Python}\label{directories-in-python}}

All files are contained within various directories, and Python has no
problem handling these too. The \texttt{os} module has several methods
that help you create, remove, and change directories.

\hypertarget{the-mkdir-method}{%
\subsection{\texorpdfstring{The \texttt{mkdir()}
Method}{The mkdir() Method}}\label{the-mkdir-method}}

You can use the \texttt{mkdir()} method of the os module to create
directories in the current directory. You need to supply an argument to
this method which contains the name of the directory to be created. For
example, to create a new directory named \texttt{"new\_folder"}, you can
use the following code:

\begin{Shaded}
\begin{Highlighting}[]
\ImportTok{import}\NormalTok{ os}

\NormalTok{os.mkdir(}\StringTok{"new\_folder"}\NormalTok{)}
\end{Highlighting}
\end{Shaded}

\hypertarget{the-chdir-method}{%
\subsection{\texorpdfstring{The \texttt{chdir()}
Method}{The chdir() Method}}\label{the-chdir-method}}

You can use the \texttt{chdir()} method to change the current directory.
The \texttt{chdir()} method takes an argument, which is the name of the
directory that you want to make the current directory. For example, to
change the current directory to \texttt{"new\_folder"}, you can use the
following code:

\begin{Shaded}
\begin{Highlighting}[]
\ImportTok{import}\NormalTok{ os}

\NormalTok{os.chdir(}\StringTok{"new\_folder"}\NormalTok{)}
\end{Highlighting}
\end{Shaded}

\hypertarget{the-getcwd-method}{%
\subsection{\texorpdfstring{The \texttt{getcwd()}
Method}{The getcwd() Method}}\label{the-getcwd-method}}

The \texttt{getcwd()} method displays the current working directory. For
example, to print the current working directory, you can use the
following code:

\begin{Shaded}
\begin{Highlighting}[]
\ImportTok{import}\NormalTok{ os}

\BuiltInTok{print}\NormalTok{(os.getcwd())}
\end{Highlighting}
\end{Shaded}

\hypertarget{the-rmdir-method}{%
\subsection{\texorpdfstring{The \texttt{rmdir()}
Method}{The rmdir() Method}}\label{the-rmdir-method}}

The \texttt{rmdir()} method deletes the directory, which is passed as an
argument in the method. Before removing a directory, all the contents in
it should be removed. For example, to remove the \texttt{"new\_folder"}
directory, you can use the following code:

\begin{Shaded}
\begin{Highlighting}[]
\ImportTok{import}\NormalTok{ os}

\NormalTok{os.rmdir(}\StringTok{"new\_folder"}\NormalTok{)}
\end{Highlighting}
\end{Shaded}

    \begin{tcolorbox}[breakable, size=fbox, boxrule=1pt, pad at break*=1mm,colback=cellbackground, colframe=cellborder]
\prompt{In}{incolor}{1}{\boxspacing}
\begin{Verbatim}[commandchars=\\\{\}]
\PY{n}{file} \PY{o}{=} \PY{n+nb}{open}\PY{p}{(}\PY{l+s+s2}{\PYZdq{}}\PY{l+s+s2}{example.txt}\PY{l+s+s2}{\PYZdq{}}\PY{p}{,} \PY{l+s+s2}{\PYZdq{}}\PY{l+s+s2}{rb}\PY{l+s+s2}{\PYZdq{}}\PY{p}{)}
\PY{n}{file}\PY{o}{.}\PY{n}{read}\PY{p}{(}\PY{l+m+mi}{10}\PY{p}{)}
\PY{n+nb}{print}\PY{p}{(}\PY{n}{file}\PY{o}{.}\PY{n}{tell}\PY{p}{(}\PY{p}{)}\PY{p}{)}
\PY{n}{file}\PY{o}{.}\PY{n}{seek}\PY{p}{(}\PY{l+m+mi}{10}\PY{p}{,}\PY{l+m+mi}{1}\PY{p}{)} \PY{c+c1}{\PYZsh{} move the file pointer to the 10th byte from the current position of the file}
\PY{n+nb}{print}\PY{p}{(}\PY{n}{file}\PY{o}{.}\PY{n}{tell}\PY{p}{(}\PY{p}{)}\PY{p}{)}
\PY{n}{file}\PY{o}{.}\PY{n}{read}\PY{p}{(}\PY{p}{)}
\PY{n+nb}{print}\PY{p}{(}\PY{n}{file}\PY{o}{.}\PY{n}{tell}\PY{p}{(}\PY{p}{)}\PY{p}{)}
\PY{n}{file}\PY{o}{.}\PY{n}{seek}\PY{p}{(}\PY{o}{\PYZhy{}}\PY{l+m+mi}{20}\PY{p}{,}\PY{l+m+mi}{2}\PY{p}{)} \PY{c+c1}{\PYZsh{} move the file pointer to the 10th byte from the current position of the file}
\PY{n+nb}{print}\PY{p}{(}\PY{n}{file}\PY{o}{.}\PY{n}{tell}\PY{p}{(}\PY{p}{)}\PY{p}{)}
\PY{n}{file}\PY{o}{.}\PY{n}{close}\PY{p}{(}\PY{p}{)}
\end{Verbatim}
\end{tcolorbox}

    \begin{Verbatim}[commandchars=\\\{\}]
10
20
679
659
    \end{Verbatim}

    \hypertarget{open-files-using-with-statement}{%
\subsubsection{\texorpdfstring{open files using \texttt{with}
statement}{open files using with statement}}\label{open-files-using-with-statement}}

The basic syntax for opening a file using the \texttt{with} statement is
as follows:

\begin{Shaded}
\begin{Highlighting}[]
\ControlFlowTok{with} \BuiltInTok{open}\NormalTok{(}\StringTok{\textquotesingle{}filename.txt\textquotesingle{}}\NormalTok{, }\StringTok{\textquotesingle{}r\textquotesingle{}}\NormalTok{) }\ImportTok{as} \BuiltInTok{file}\NormalTok{:}
    \CommentTok{\# do something with the file}
\end{Highlighting}
\end{Shaded}

In this example, we open the file \texttt{"filename.txt"} in read mode
(\texttt{\textquotesingle{}r\textquotesingle{}}) using the
\texttt{open()} function. The with statement creates a block of code
where the file is open and available to use, and automatically closes
the file when the block of code is exited.

This ensures that the file is properly closed even if an exception is
raised within the block of code. Example:

\begin{Shaded}
\begin{Highlighting}[]
\ControlFlowTok{with} \BuiltInTok{open}\NormalTok{(}\StringTok{\textquotesingle{}output.txt\textquotesingle{}}\NormalTok{, }\StringTok{\textquotesingle{}a+\textquotesingle{}}\NormalTok{) }\ImportTok{as} \BuiltInTok{file}\NormalTok{:}
    \BuiltInTok{file}\NormalTok{.write(}\StringTok{\textquotesingle{}Hello, world!\textquotesingle{}}\NormalTok{)}
\end{Highlighting}
\end{Shaded}

\hypertarget{introduction-to-exceptions-and-error-handling}{%
\section{Introduction to Exceptions and Error
Handling}\label{introduction-to-exceptions-and-error-handling}}

An exception is a signal that an error has occurred, and Python's
built-in exception handling mechanism allows you to gracefully handle
these errors and prevent your program from crashing.

\hypertarget{using-try-except-blocks-to-handle-exceptions}{%
\subsection{\texorpdfstring{Using \texttt{try-except} Blocks to Handle
Exceptions}{Using try-except Blocks to Handle Exceptions}}\label{using-try-except-blocks-to-handle-exceptions}}

In Python, you can use the try and except statements to handle
exceptions. The basic syntax is as follows:

\begin{Shaded}
\begin{Highlighting}[]
\ControlFlowTok{try}\NormalTok{:}
    \CommentTok{\# code that might raise an exception}
\ControlFlowTok{except}\NormalTok{ ExceptionType:}
    \CommentTok{\# code to handle the exception}
\end{Highlighting}
\end{Shaded}

When the code inside the \texttt{try} block raises an exception (error),
the interpreter jumps to the corresponding except block. The
\texttt{ExceptionType} argument specifies the type of exception to
catch. For example, to catch all exceptions, you can use the
\texttt{Exception} base class:

\begin{Shaded}
\begin{Highlighting}[]
\ControlFlowTok{try}\NormalTok{:}
    \CommentTok{\# code that might raise an exception}
\ControlFlowTok{except} \PreprocessorTok{Exception}\NormalTok{:}
    \CommentTok{\# code to handle the exception}
\end{Highlighting}
\end{Shaded}

Here's an example of how to use a \texttt{try-except} block to handle a
ZeroDivisionError:

\begin{Shaded}
\begin{Highlighting}[]
\ControlFlowTok{try}\NormalTok{:}
\NormalTok{    result }\OperatorTok{=} \DecValTok{1} \OperatorTok{/} \DecValTok{0}
\ControlFlowTok{except} \PreprocessorTok{ZeroDivisionError}\NormalTok{:}
    \BuiltInTok{print}\NormalTok{(}\StringTok{"Are you serious! You can\textquotesingle{}t divide by zero"}\NormalTok{)}
\end{Highlighting}
\end{Shaded}

In this example, the \texttt{try} block attempts to divide 1 by 0, which
raises a ZeroDivisionError. The corresponding except block catches the
error and prints an error message.

\hypertarget{handling-multiple-types-of-exceptions}{%
\subsection{Handling Multiple Types of
Exceptions}\label{handling-multiple-types-of-exceptions}}

You can also catch specific types of exceptions, such as TypeError or
ValueError. To handle multiple types of exceptions, you can include
multiple except blocks:

\begin{Shaded}
\begin{Highlighting}[]
\ControlFlowTok{try}\NormalTok{:}
    \CommentTok{\# code that might raise an exception}
\ControlFlowTok{except} \PreprocessorTok{TypeError}\NormalTok{:}
    \CommentTok{\# code to handle a TypeError}
\ControlFlowTok{except} \PreprocessorTok{ValueError}\NormalTok{:}
    \CommentTok{\# code to handle a ValueError}
\end{Highlighting}
\end{Shaded}

Here's an example of how to handle multiple types of exceptions:

\begin{Shaded}
\begin{Highlighting}[]
\ControlFlowTok{try}\NormalTok{:}
\NormalTok{    result }\OperatorTok{=} \BuiltInTok{int}\NormalTok{(}\StringTok{"not a number"}\NormalTok{)}
\ControlFlowTok{except}\NormalTok{ (}\PreprocessorTok{TypeError}\NormalTok{, }\PreprocessorTok{ValueError}\NormalTok{):}
    \BuiltInTok{print}\NormalTok{(}\StringTok{"Error: Invalid input"}\NormalTok{)}
\end{Highlighting}
\end{Shaded}

In this example, the try block attempts to convert the string
\texttt{"not\ a\ number"} to an integer, which raises a
\texttt{ValueError}. The corresponding except block catches both
\texttt{TypeError} and \texttt{ValueError} exceptions and prints an
error message. \#\# Multiple \texttt{except} and \texttt{else} block

In addition to handling different types of exceptions, you can also use
multiple \texttt{except} and \texttt{else} to further customize the
behavior of your error handling code.

When using multiple except blocks, you can specify different exception
types and handle each one differently. Here's an example:

\begin{Shaded}
\begin{Highlighting}[]
\ControlFlowTok{try}\NormalTok{:}
\NormalTok{    x }\OperatorTok{=} \BuiltInTok{int}\NormalTok{(}\BuiltInTok{input}\NormalTok{(}\StringTok{"Enter a number: "}\NormalTok{))}
\NormalTok{    y }\OperatorTok{=} \DecValTok{10} \OperatorTok{/}\NormalTok{ x}
\ControlFlowTok{except} \PreprocessorTok{ValueError}\NormalTok{:}
    \BuiltInTok{print}\NormalTok{(}\StringTok{"Invalid input"}\NormalTok{)}
\ControlFlowTok{except} \PreprocessorTok{ZeroDivisionError}\NormalTok{:}
    \BuiltInTok{print}\NormalTok{(}\StringTok{"Cannot divide by zero"}\NormalTok{)}
\ControlFlowTok{else}\NormalTok{:}
    \BuiltInTok{print}\NormalTok{(}\StringTok{"Result is:"}\NormalTok{, y)}
\end{Highlighting}
\end{Shaded}

\hypertarget{using-finally-blocks}{%
\subsection{\texorpdfstring{Using \texttt{finally}
Blocks}{Using finally Blocks}}\label{using-finally-blocks}}

In addition to the try and except statements, Python provides the
finally statement to run code that should execute regardless of whether
an exception is raised:

\begin{Shaded}
\begin{Highlighting}[]
\ControlFlowTok{try}\NormalTok{:}
    \CommentTok{\# code that might raise an exception}
\ControlFlowTok{except} \PreprocessorTok{Exception}\NormalTok{:}
    \CommentTok{\# code to handle the exception}
\ControlFlowTok{finally}\NormalTok{:}
    \CommentTok{\# code that always runs, even if there was an exception}
\end{Highlighting}
\end{Shaded}

Here's an example of how to use a finally block:

\begin{Shaded}
\begin{Highlighting}[]
\BuiltInTok{file} \OperatorTok{=} \BuiltInTok{open}\NormalTok{(}\StringTok{"myfile.txt"}\NormalTok{, }\StringTok{"w"}\NormalTok{)}
\ControlFlowTok{try}\NormalTok{:}
    \CommentTok{\# write data to the file}
\ControlFlowTok{finally}\NormalTok{:}
    \BuiltInTok{file}\NormalTok{.close()}
\end{Highlighting}
\end{Shaded}

In this example, the try block writes data to a file, and the finally
block ensures that the file is closed, even if an exception is raised.

\hypertarget{raising-exceptions}{%
\subsection{Raising Exceptions}\label{raising-exceptions}}

You can use the raise statement to raise an exception of a specified
type, along with an optional error message:

\begin{Shaded}
\begin{Highlighting}[]
\ControlFlowTok{raise}\NormalTok{ ExceptionType(}\StringTok{"Error message"}\NormalTok{)}
\end{Highlighting}
\end{Shaded}

Here's an example of how to raise a \texttt{ValueError}:

\begin{Shaded}
\begin{Highlighting}[]
\ImportTok{import}\NormalTok{ sys}
\ControlFlowTok{try}\NormalTok{:}
\NormalTok{    x}\OperatorTok{=}\BuiltInTok{int}\NormalTok{(}\BuiltInTok{input}\NormalTok{(}\StringTok{"Give a positive integer \textgreater{} 19}\CharTok{\textbackslash{}n}\StringTok{"}\NormalTok{)) }
    \ControlFlowTok{if}\NormalTok{ (x}\OperatorTok{\textless{}=}\DecValTok{19}\NormalTok{):}
        \ControlFlowTok{raise} \PreprocessorTok{ValueError}\NormalTok{(}\StringTok{"Invalid input Value"}\NormalTok{)}
\ControlFlowTok{except} \PreprocessorTok{ValueError}\NormalTok{:}
    \BuiltInTok{print}\NormalTok{(}\StringTok{"Please focus a little!"}\NormalTok{)}
\NormalTok{    sys.exit(}\DecValTok{1}\NormalTok{)}
\BuiltInTok{print}\NormalTok{(}\StringTok{"Continue !"}\NormalTok{)}
\CommentTok{\# rest of code}
\end{Highlighting}
\end{Shaded}

In this example, the my\_function function checks whether the input is
valid, and raises a \texttt{ValueError}.


    % Add a bibliography block to the postdoc
    
    
    
\end{document}
