\question
{\center \bf The $0\text{-}1$ Knapsack problem\\}
The $0\text{-}1$ Knapsack problem is a classic optimization problem in computer
science and mathematics. The problem is defined as follows:

Given a set of items, each with a weight and a value, and a knapsack
with a maximum weight capacity, determine the maximum value that can be
put into the knapsack without exceeding its weight capacity.

In the $0\text{-}1$ version of the problem, each item can only be taken once
(either included in the knapsack or not), hence the name ($0\text{-}1$). The
goal is to maximize the total value of the items in the knapsack while
keeping the weight of the knapsack within the maximum capacity.

The $0\text{-}1$ Knapsack problem has important applications in various fields,
such as resource allocation, financial portfolio optimization, and
logistics planning.

\begin{itemize}
\item
  \textbf{Task 1}: Problem Description Write a brief description of the
  $0\text{-}1$ Knapsack problem, including its goal, and how it can be
  represented as a graph.
\item
  \textbf{Task 2}: Research the dynamic programming Method Research the
  dynamic programming Method and write a brief explanation of how it
  works, including the steps involved in solving the $0\text{-}1$ Knapsack
  problem.
\item
  \textbf{Task 3}: Implement the dynamic programming method Implement
  the dynamic programming in Python, using NumPy to perform the
  necessary calculations.
\item
  \textbf{Task 4}: Test the Implementation Test the implementation using
  a small example problem and verify that it produces the correct
  solution.
\item
  \textbf{Task 5}: Write a Report Write a report summarizing the
  project, including the problem description, the solution approach, the
  implementation details, and the test results.
\end{itemize}

\textbf{References}:

\begin{enumerate}
\def\labelenumi{\arabic{enumi}.}
\item
\href{https://en.wikipedia.org/wiki/Knapsack\_problem}{https://en.wikipedia.org/wiki/Knapsack\_problem}
\end{enumerate}
\newpage