\begin{frame}[fragile]{Math functions and Strings}
\tit{Mathematical functions}
Python has a limited number of {\bf built-in mathematical functions}:
    \begin{lstlisting}[numbers=left,showstringspaces=false,language=python]
    abs( number ) #Return the absolute value of the argument, |x|.
    pow( x , y , [ z ]) #Raise x to the y power. If z is present, this is done 
                      #modulo z , x y % z .
    round( number , [ ndigits ]) #Round number to ndigits beyond the decimal 
                                #point (rounds a number to the nearest 
                                #whole number).
    cmp( x , y ) #Compare x and y , returning a number.     
    hex( number )#Create a hexadecimal string representation of number . A 
                   #leading '0x' is placed on the string as a reminder that 
                   #this is hexadecimal.
    oct( number )#Create a octal string representation of number .
                  #A leading '0' is placed on the string as a reminder that this
                  #is octal not decimal. 
    int( string , [ base ]) #Generates an integer from the string x . If base is
                         #supplied, x must be in the given base. If base is 
                         #omitted, x must be decimal.
    max( sequence ) #Return the largest value in sequence .
    min( sequence ) #Return the smallest value in sequence .
    ord( character ) #Returns the Unicode code from a given character. 
    chr( character ) #Returns a character whose Unicode code point is an integer.
    \end{lstlisting}      
\end{frame}
\begin{frame}[fragile]{Math functions and Strings}
    \tit{ math module functions}
More advanced mathematical functions are contained in the {\bf \verb|math| module}.


A {\bf module} is a file that contains a collection of related functions.

Before we can use the functions from a module, we have to import them:
\begin{lstlisting}[numbers=left,showstringspaces=false,language=python]
import math # call module before use
angle = 90 * 2 * math.pi / 360.0
math.sin(angle)
\end{lstlisting}
\end{frame}
\begin{frame}[fragile]{Math functions and Strings}
    \tit{Examples}
\begin{lstlisting}[numbers=left,showstringspaces=false,language=python]
math.ceil(x) #Returns the smallest integer greater than or equal to x.
math.fabs(x) #Returns the absolute value of x
math.factorial(x) #Returns the factorial of x
math.floor(x) #Returns the largest integer less than or equal to x
math.fmod(x, y) #Returns the remainder when x is divided by y
math.exp(x) #Returns e**x
math.log(x[, b]) #Returns the logarithm of x to the base b (defaults to e)
math.log2(x) #Returns the base-2 logarithm of x
math.log10(x) #Returns the base-10 logarithm of x
math.pow(x, y) #Returns x raised to the power y
math.sqrt(x) #Returns the square root of x
math.acos(x) #Returns the arc cosine of x
math.asin(x) #Returns the arc sine of x
math.atan(x) #Returns the arc tangent of x
\end{lstlisting}
\end{frame}
\begin{frame}[fragile]{Math functions and Strings}
    \tit{Examples}
    \begin{lstlisting}[numbers=left,showstringspaces=false,language=python]
math.cos(x) #Returns the cosine of x
math.sin(x) #Returns the sine of x
math.tan(x) #Returns the tangent of x
math.degrees(x)	#Converts angle x from radians to degrees
math.radians(x)	#Converts angle x from degrees to radians
math.acosh(x) #Returns the inverse hyperbolic cosine of x
math.asinh(x) #Returns the inverse hyperbolic sine of x
math.atanh(x) #Returns the inverse hyperbolic tangent of x
math.cosh(x) #Returns the hyperbolic cosine of x
math.sinh(x) #Returns the hyperbolic cosine of x
math.tanh(x) #Returns the hyperbolic tangent of x
math.pi	#Mathematical constant, the ratio of circumference of a 
        #circle to it's diameter (3.14159...)
math.e #Mathematical constant e (2.71828...)
\end{lstlisting}
\end{frame}
\begin{frame}[fragile]{Math functions and Strings}
    \tit{What is String in Python?}
    A string is a sequence of characters.

    \tit{How to create a string in Python?}
    Strings can be created by enclosing characters inside a single quote or double-quotes.
    
    
    Even triple quotes can be used in Python but generally used to represent multiline strings and docstrings.
\begin{lstlisting}[numbers=left,showstringspaces=false,language=python]
    # defining strings in Python, all of the following are equivalent
    say_hello = 'Hello'

    say_hello = "Hello"

    say_hello = '''Hello'''
    
    # triple quotes string can extend multiple lines
    say_hello = """Hello, welcome to
               the world of Python"""
\end{lstlisting}
\end{frame}
\begin{frame}[fragile]{Math functions and Strings}
\tit{How to access characters in a string?}
\begin{itemize}[<+->]
    \item We can access individual characters using {\bf indexing} and a range of characters using {\bf slicing}. 
    
    
    \item Python is zero-indexing language, Index starts from 0. 
    
    \item Trying to access a character out of index range will raise an \verb|IndexError|.
    \item The index must be an integer. We can't use floats or other types, this will result into \verb|TypeError|.
   \item Python allows negative indexing for its sequences.
   \item The index of $-1$ refers to the last item, $-2$ to the second last item and so on. We can access a range of items in a string by using the slicing operator :(colon).
\end{itemize}
\end{frame}
\begin{frame}[fragile]{Math functions and Strings}
\begin{lstlisting}[numbers=left,showstringspaces=false,language=python]
#Accessing string characters in Python
my_word = 'ThisIsALongWord'
print('my_word = ', my_word)

#first character
print('my_word[0] = ', my_word[0])

#last character
print('my_word[-1] = ', my_word[-1])

#slicing 2nd to 5th character
print('my_word[1:5] = ', my_word[1:5])

#slicing 6th to 2nd last character
print('my_word[5:-2] = ', my_word[5:-2])
\end{lstlisting}
\pause
Now, we try to access an index out of the range or use numbers other than an integer
\begin{lstlisting}[numbers=left,showstringspaces=false,language=python]
# index must be in range
my_word[16]  
# index must be an integer
my_word[1+0j] 
\end{lstlisting}

\end{frame}
\begin{frame}[fragile]{Math functions and Strings}
    \tit{Common Operations on Strings}
    {\center \bf Concatenation of Two or More Strings\\}
        Joining of two or more strings into a single one is called concatenation.


        The \verb|+| operator does this in Python. Simply writing two string literals together also concatenates them.

        The \verb|*| operator can be used to repeat a string for a given number of times.
\begin{lstlisting}[numbers=left,showstringspaces=false,language=python]
# Python String Operations
str1 = 'Hello'
str2 ='NHSM!'

# using +
print('str1 + str2 = ', str1 + str2)
# using *
print('str2 * 3 =', str2 * 3)
# two string literals together
str1='Hello ''World!'
print('str1  =', str1)
\end{lstlisting}
\end{frame}
\begin{frame}[fragile]{Math functions and Strings}

    {\center \bf String Membership Test\\}

    We can test if a substring exists within a string or not, using the keyword \verb|in|
\begin{lstlisting}[numbers=left,showstringspaces=false,language=python]
    'me' in 'home'
    'ho' not in 'home'
\end{lstlisting}
{\center \bf Built-in functions to Work with Python\\}
The \verb|enumerate()| function returns an enumerate object.


The \verb|len()| returns the length (number of characters) of the string.
\begin{lstlisting}[numbers=left,showstringspaces=false,language=python]
    word="Hello NHSM"
    print(enumerate(word))
    print(len(word))
\end{lstlisting}
\end{frame}
\begin{frame}[fragile]{Math functions and Strings}
    {\center \bf Common String methods\\}
    Some of the commonly used methods are \verb|lower()|, \verb|upper()|, \verb|join()|, \verb|split()|, \verb|find()|, \verb|replace()| etc. 
    \begin{lstlisting}[numbers=left,showstringspaces=false,language=python]
        word="PythonIsNotFunny"
        print(word.lower())
        print(word.upper())
        sentence="This will split all words into a list"
        print(sentence.split())
        sentence="This will split, all words, into a list"
        print(sentence.split(','))
        word_list=['This', 'will', 'join', 'all', 'words', 'into', 'a', 'string']
        delimiter="*"
        print(delimiter.join(word_list))
        print(' '.join(word_list))
        print('Happy Day with python'.find('ay'))
        print('Happy Day with python'.replace('Happy','Brilliant'))
    \end{lstlisting}
\end{frame}
