\documentclass[11pt]{article}

    \usepackage[breakable]{tcolorbox}
    \usepackage{parskip} % Stop auto-indenting (to mimic markdown behaviour)
    
    \usepackage{iftex}
    \ifPDFTeX
    	\usepackage[T1]{fontenc}
    	\usepackage{mathpazo}
    \else
    	\usepackage{fontspec}
    \fi

    % Basic figure setup, for now with no caption control since it's done
    % automatically by Pandoc (which extracts ![](path) syntax from Markdown).
    \usepackage{graphicx}
    % Maintain compatibility with old templates. Remove in nbconvert 6.0
    \let\Oldincludegraphics\includegraphics
    % Ensure that by default, figures have no caption (until we provide a
    % proper Figure object with a Caption API and a way to capture that
    % in the conversion process - todo).
    \usepackage{caption}
    \DeclareCaptionFormat{nocaption}{}
    \captionsetup{format=nocaption,aboveskip=0pt,belowskip=0pt}

    \usepackage{float}
    \floatplacement{figure}{H} % forces figures to be placed at the correct location
    \usepackage{xcolor} % Allow colors to be defined
    \usepackage{enumerate} % Needed for markdown enumerations to work
    \usepackage{geometry} % Used to adjust the document margins
    \usepackage{amsmath} % Equations
    \usepackage{amssymb} % Equations
    \usepackage{textcomp} % defines textquotesingle
    % Hack from http://tex.stackexchange.com/a/47451/13684:
    \AtBeginDocument{%
        \def\PYZsq{\textquotesingle}% Upright quotes in Pygmentized code
    }
    \usepackage{upquote} % Upright quotes for verbatim code
    \usepackage{eurosym} % defines \euro
    \usepackage[mathletters]{ucs} % Extended unicode (utf-8) support
    \usepackage{fancyvrb} % verbatim replacement that allows latex
    \usepackage{grffile} % extends the file name processing of package graphics 
                         % to support a larger range
    \makeatletter % fix for old versions of grffile with XeLaTeX
    \@ifpackagelater{grffile}{2019/11/01}
    {
      % Do nothing on new versions
    }
    {
      \def\Gread@@xetex#1{%
        \IfFileExists{"\Gin@base".bb}%
        {\Gread@eps{\Gin@base.bb}}%
        {\Gread@@xetex@aux#1}%
      }
    }
    \makeatother
    \usepackage[Export]{adjustbox} % Used to constrain images to a maximum size
    \adjustboxset{max size={0.9\linewidth}{0.9\paperheight}}

    % The hyperref package gives us a pdf with properly built
    % internal navigation ('pdf bookmarks' for the table of contents,
    % internal cross-reference links, web links for URLs, etc.)
    \usepackage{hyperref}
    % The default LaTeX title has an obnoxious amount of whitespace. By default,
    % titling removes some of it. It also provides customization options.
    \usepackage{titling}
    \usepackage{longtable} % longtable support required by pandoc >1.10
    \usepackage{booktabs}  % table support for pandoc > 1.12.2
    \usepackage[inline]{enumitem} % IRkernel/repr support (it uses the enumerate* environment)
    \usepackage[normalem]{ulem} % ulem is needed to support strikethroughs (\sout)
                                % normalem makes italics be italics, not underlines
    \usepackage{mathrsfs}
    

    
    % Colors for the hyperref package
    \definecolor{urlcolor}{rgb}{0,.145,.698}
    \definecolor{linkcolor}{rgb}{.71,0.21,0.01}
    \definecolor{citecolor}{rgb}{.12,.54,.11}

    % ANSI colors
    \definecolor{ansi-black}{HTML}{3E424D}
    \definecolor{ansi-black-intense}{HTML}{282C36}
    \definecolor{ansi-red}{HTML}{E75C58}
    \definecolor{ansi-red-intense}{HTML}{B22B31}
    \definecolor{ansi-green}{HTML}{00A250}
    \definecolor{ansi-green-intense}{HTML}{007427}
    \definecolor{ansi-yellow}{HTML}{DDB62B}
    \definecolor{ansi-yellow-intense}{HTML}{B27D12}
    \definecolor{ansi-blue}{HTML}{208FFB}
    \definecolor{ansi-blue-intense}{HTML}{0065CA}
    \definecolor{ansi-magenta}{HTML}{D160C4}
    \definecolor{ansi-magenta-intense}{HTML}{A03196}
    \definecolor{ansi-cyan}{HTML}{60C6C8}
    \definecolor{ansi-cyan-intense}{HTML}{258F8F}
    \definecolor{ansi-white}{HTML}{C5C1B4}
    \definecolor{ansi-white-intense}{HTML}{A1A6B2}
    \definecolor{ansi-default-inverse-fg}{HTML}{FFFFFF}
    \definecolor{ansi-default-inverse-bg}{HTML}{000000}

    % common color for the border for error outputs.
    \definecolor{outerrorbackground}{HTML}{FFDFDF}

    % commands and environments needed by pandoc snippets
    % extracted from the output of `pandoc -s`
    \providecommand{\tightlist}{%
      \setlength{\itemsep}{0pt}\setlength{\parskip}{0pt}}
    \DefineVerbatimEnvironment{Highlighting}{Verbatim}{commandchars=\\\{\}}
    % Add ',fontsize=\small' for more characters per line
    \newenvironment{Shaded}{}{}
    \newcommand{\KeywordTok}[1]{\textcolor[rgb]{0.00,0.44,0.13}{\textbf{{#1}}}}
    \newcommand{\DataTypeTok}[1]{\textcolor[rgb]{0.56,0.13,0.00}{{#1}}}
    \newcommand{\DecValTok}[1]{\textcolor[rgb]{0.25,0.63,0.44}{{#1}}}
    \newcommand{\BaseNTok}[1]{\textcolor[rgb]{0.25,0.63,0.44}{{#1}}}
    \newcommand{\FloatTok}[1]{\textcolor[rgb]{0.25,0.63,0.44}{{#1}}}
    \newcommand{\CharTok}[1]{\textcolor[rgb]{0.25,0.44,0.63}{{#1}}}
    \newcommand{\StringTok}[1]{\textcolor[rgb]{0.25,0.44,0.63}{{#1}}}
    \newcommand{\CommentTok}[1]{\textcolor[rgb]{0.38,0.63,0.69}{\textit{{#1}}}}
    \newcommand{\OtherTok}[1]{\textcolor[rgb]{0.00,0.44,0.13}{{#1}}}
    \newcommand{\AlertTok}[1]{\textcolor[rgb]{1.00,0.00,0.00}{\textbf{{#1}}}}
    \newcommand{\FunctionTok}[1]{\textcolor[rgb]{0.02,0.16,0.49}{{#1}}}
    \newcommand{\RegionMarkerTok}[1]{{#1}}
    \newcommand{\ErrorTok}[1]{\textcolor[rgb]{1.00,0.00,0.00}{\textbf{{#1}}}}
    \newcommand{\NormalTok}[1]{{#1}}
    
    % Additional commands for more recent versions of Pandoc
    \newcommand{\ConstantTok}[1]{\textcolor[rgb]{0.53,0.00,0.00}{{#1}}}
    \newcommand{\SpecialCharTok}[1]{\textcolor[rgb]{0.25,0.44,0.63}{{#1}}}
    \newcommand{\VerbatimStringTok}[1]{\textcolor[rgb]{0.25,0.44,0.63}{{#1}}}
    \newcommand{\SpecialStringTok}[1]{\textcolor[rgb]{0.73,0.40,0.53}{{#1}}}
    \newcommand{\ImportTok}[1]{{#1}}
    \newcommand{\DocumentationTok}[1]{\textcolor[rgb]{0.73,0.13,0.13}{\textit{{#1}}}}
    \newcommand{\AnnotationTok}[1]{\textcolor[rgb]{0.38,0.63,0.69}{\textbf{\textit{{#1}}}}}
    \newcommand{\CommentVarTok}[1]{\textcolor[rgb]{0.38,0.63,0.69}{\textbf{\textit{{#1}}}}}
    \newcommand{\VariableTok}[1]{\textcolor[rgb]{0.10,0.09,0.49}{{#1}}}
    \newcommand{\ControlFlowTok}[1]{\textcolor[rgb]{0.00,0.44,0.13}{\textbf{{#1}}}}
    \newcommand{\OperatorTok}[1]{\textcolor[rgb]{0.40,0.40,0.40}{{#1}}}
    \newcommand{\BuiltInTok}[1]{{#1}}
    \newcommand{\ExtensionTok}[1]{{#1}}
    \newcommand{\PreprocessorTok}[1]{\textcolor[rgb]{0.74,0.48,0.00}{{#1}}}
    \newcommand{\AttributeTok}[1]{\textcolor[rgb]{0.49,0.56,0.16}{{#1}}}
    \newcommand{\InformationTok}[1]{\textcolor[rgb]{0.38,0.63,0.69}{\textbf{\textit{{#1}}}}}
    \newcommand{\WarningTok}[1]{\textcolor[rgb]{0.38,0.63,0.69}{\textbf{\textit{{#1}}}}}
    
    
    % Define a nice break command that doesn't care if a line doesn't already
    % exist.
    \def\br{\hspace*{\fill} \\* }
    % Math Jax compatibility definitions
    \def\gt{>}
    \def\lt{<}
    \let\Oldtex\TeX
    \let\Oldlatex\LaTeX
    \renewcommand{\TeX}{\textrm{\Oldtex}}
    \renewcommand{\LaTeX}{\textrm{\Oldlatex}}
    % Document parameters
    % Document title
    \title{Lecture 0 - Part 3: Comprehensions and Functions}
    \date{}
    
    
    
    
% Pygments definitions
\makeatletter
\def\PY@reset{\let\PY@it=\relax \let\PY@bf=\relax%
    \let\PY@ul=\relax \let\PY@tc=\relax%
    \let\PY@bc=\relax \let\PY@ff=\relax}
\def\PY@tok#1{\csname PY@tok@#1\endcsname}
\def\PY@toks#1+{\ifx\relax#1\empty\else%
    \PY@tok{#1}\expandafter\PY@toks\fi}
\def\PY@do#1{\PY@bc{\PY@tc{\PY@ul{%
    \PY@it{\PY@bf{\PY@ff{#1}}}}}}}
\def\PY#1#2{\PY@reset\PY@toks#1+\relax+\PY@do{#2}}

\@namedef{PY@tok@w}{\def\PY@tc##1{\textcolor[rgb]{0.73,0.73,0.73}{##1}}}
\@namedef{PY@tok@c}{\let\PY@it=\textit\def\PY@tc##1{\textcolor[rgb]{0.25,0.50,0.50}{##1}}}
\@namedef{PY@tok@cp}{\def\PY@tc##1{\textcolor[rgb]{0.74,0.48,0.00}{##1}}}
\@namedef{PY@tok@k}{\let\PY@bf=\textbf\def\PY@tc##1{\textcolor[rgb]{0.00,0.50,0.00}{##1}}}
\@namedef{PY@tok@kp}{\def\PY@tc##1{\textcolor[rgb]{0.00,0.50,0.00}{##1}}}
\@namedef{PY@tok@kt}{\def\PY@tc##1{\textcolor[rgb]{0.69,0.00,0.25}{##1}}}
\@namedef{PY@tok@o}{\def\PY@tc##1{\textcolor[rgb]{0.40,0.40,0.40}{##1}}}
\@namedef{PY@tok@ow}{\let\PY@bf=\textbf\def\PY@tc##1{\textcolor[rgb]{0.67,0.13,1.00}{##1}}}
\@namedef{PY@tok@nb}{\def\PY@tc##1{\textcolor[rgb]{0.00,0.50,0.00}{##1}}}
\@namedef{PY@tok@nf}{\def\PY@tc##1{\textcolor[rgb]{0.00,0.00,1.00}{##1}}}
\@namedef{PY@tok@nc}{\let\PY@bf=\textbf\def\PY@tc##1{\textcolor[rgb]{0.00,0.00,1.00}{##1}}}
\@namedef{PY@tok@nn}{\let\PY@bf=\textbf\def\PY@tc##1{\textcolor[rgb]{0.00,0.00,1.00}{##1}}}
\@namedef{PY@tok@ne}{\let\PY@bf=\textbf\def\PY@tc##1{\textcolor[rgb]{0.82,0.25,0.23}{##1}}}
\@namedef{PY@tok@nv}{\def\PY@tc##1{\textcolor[rgb]{0.10,0.09,0.49}{##1}}}
\@namedef{PY@tok@no}{\def\PY@tc##1{\textcolor[rgb]{0.53,0.00,0.00}{##1}}}
\@namedef{PY@tok@nl}{\def\PY@tc##1{\textcolor[rgb]{0.63,0.63,0.00}{##1}}}
\@namedef{PY@tok@ni}{\let\PY@bf=\textbf\def\PY@tc##1{\textcolor[rgb]{0.60,0.60,0.60}{##1}}}
\@namedef{PY@tok@na}{\def\PY@tc##1{\textcolor[rgb]{0.49,0.56,0.16}{##1}}}
\@namedef{PY@tok@nt}{\let\PY@bf=\textbf\def\PY@tc##1{\textcolor[rgb]{0.00,0.50,0.00}{##1}}}
\@namedef{PY@tok@nd}{\def\PY@tc##1{\textcolor[rgb]{0.67,0.13,1.00}{##1}}}
\@namedef{PY@tok@s}{\def\PY@tc##1{\textcolor[rgb]{0.73,0.13,0.13}{##1}}}
\@namedef{PY@tok@sd}{\let\PY@it=\textit\def\PY@tc##1{\textcolor[rgb]{0.73,0.13,0.13}{##1}}}
\@namedef{PY@tok@si}{\let\PY@bf=\textbf\def\PY@tc##1{\textcolor[rgb]{0.73,0.40,0.53}{##1}}}
\@namedef{PY@tok@se}{\let\PY@bf=\textbf\def\PY@tc##1{\textcolor[rgb]{0.73,0.40,0.13}{##1}}}
\@namedef{PY@tok@sr}{\def\PY@tc##1{\textcolor[rgb]{0.73,0.40,0.53}{##1}}}
\@namedef{PY@tok@ss}{\def\PY@tc##1{\textcolor[rgb]{0.10,0.09,0.49}{##1}}}
\@namedef{PY@tok@sx}{\def\PY@tc##1{\textcolor[rgb]{0.00,0.50,0.00}{##1}}}
\@namedef{PY@tok@m}{\def\PY@tc##1{\textcolor[rgb]{0.40,0.40,0.40}{##1}}}
\@namedef{PY@tok@gh}{\let\PY@bf=\textbf\def\PY@tc##1{\textcolor[rgb]{0.00,0.00,0.50}{##1}}}
\@namedef{PY@tok@gu}{\let\PY@bf=\textbf\def\PY@tc##1{\textcolor[rgb]{0.50,0.00,0.50}{##1}}}
\@namedef{PY@tok@gd}{\def\PY@tc##1{\textcolor[rgb]{0.63,0.00,0.00}{##1}}}
\@namedef{PY@tok@gi}{\def\PY@tc##1{\textcolor[rgb]{0.00,0.63,0.00}{##1}}}
\@namedef{PY@tok@gr}{\def\PY@tc##1{\textcolor[rgb]{1.00,0.00,0.00}{##1}}}
\@namedef{PY@tok@ge}{\let\PY@it=\textit}
\@namedef{PY@tok@gs}{\let\PY@bf=\textbf}
\@namedef{PY@tok@gp}{\let\PY@bf=\textbf\def\PY@tc##1{\textcolor[rgb]{0.00,0.00,0.50}{##1}}}
\@namedef{PY@tok@go}{\def\PY@tc##1{\textcolor[rgb]{0.53,0.53,0.53}{##1}}}
\@namedef{PY@tok@gt}{\def\PY@tc##1{\textcolor[rgb]{0.00,0.27,0.87}{##1}}}
\@namedef{PY@tok@err}{\def\PY@bc##1{{\setlength{\fboxsep}{\string -\fboxrule}\fcolorbox[rgb]{1.00,0.00,0.00}{1,1,1}{\strut ##1}}}}
\@namedef{PY@tok@kc}{\let\PY@bf=\textbf\def\PY@tc##1{\textcolor[rgb]{0.00,0.50,0.00}{##1}}}
\@namedef{PY@tok@kd}{\let\PY@bf=\textbf\def\PY@tc##1{\textcolor[rgb]{0.00,0.50,0.00}{##1}}}
\@namedef{PY@tok@kn}{\let\PY@bf=\textbf\def\PY@tc##1{\textcolor[rgb]{0.00,0.50,0.00}{##1}}}
\@namedef{PY@tok@kr}{\let\PY@bf=\textbf\def\PY@tc##1{\textcolor[rgb]{0.00,0.50,0.00}{##1}}}
\@namedef{PY@tok@bp}{\def\PY@tc##1{\textcolor[rgb]{0.00,0.50,0.00}{##1}}}
\@namedef{PY@tok@fm}{\def\PY@tc##1{\textcolor[rgb]{0.00,0.00,1.00}{##1}}}
\@namedef{PY@tok@vc}{\def\PY@tc##1{\textcolor[rgb]{0.10,0.09,0.49}{##1}}}
\@namedef{PY@tok@vg}{\def\PY@tc##1{\textcolor[rgb]{0.10,0.09,0.49}{##1}}}
\@namedef{PY@tok@vi}{\def\PY@tc##1{\textcolor[rgb]{0.10,0.09,0.49}{##1}}}
\@namedef{PY@tok@vm}{\def\PY@tc##1{\textcolor[rgb]{0.10,0.09,0.49}{##1}}}
\@namedef{PY@tok@sa}{\def\PY@tc##1{\textcolor[rgb]{0.73,0.13,0.13}{##1}}}
\@namedef{PY@tok@sb}{\def\PY@tc##1{\textcolor[rgb]{0.73,0.13,0.13}{##1}}}
\@namedef{PY@tok@sc}{\def\PY@tc##1{\textcolor[rgb]{0.73,0.13,0.13}{##1}}}
\@namedef{PY@tok@dl}{\def\PY@tc##1{\textcolor[rgb]{0.73,0.13,0.13}{##1}}}
\@namedef{PY@tok@s2}{\def\PY@tc##1{\textcolor[rgb]{0.73,0.13,0.13}{##1}}}
\@namedef{PY@tok@sh}{\def\PY@tc##1{\textcolor[rgb]{0.73,0.13,0.13}{##1}}}
\@namedef{PY@tok@s1}{\def\PY@tc##1{\textcolor[rgb]{0.73,0.13,0.13}{##1}}}
\@namedef{PY@tok@mb}{\def\PY@tc##1{\textcolor[rgb]{0.40,0.40,0.40}{##1}}}
\@namedef{PY@tok@mf}{\def\PY@tc##1{\textcolor[rgb]{0.40,0.40,0.40}{##1}}}
\@namedef{PY@tok@mh}{\def\PY@tc##1{\textcolor[rgb]{0.40,0.40,0.40}{##1}}}
\@namedef{PY@tok@mi}{\def\PY@tc##1{\textcolor[rgb]{0.40,0.40,0.40}{##1}}}
\@namedef{PY@tok@il}{\def\PY@tc##1{\textcolor[rgb]{0.40,0.40,0.40}{##1}}}
\@namedef{PY@tok@mo}{\def\PY@tc##1{\textcolor[rgb]{0.40,0.40,0.40}{##1}}}
\@namedef{PY@tok@ch}{\let\PY@it=\textit\def\PY@tc##1{\textcolor[rgb]{0.25,0.50,0.50}{##1}}}
\@namedef{PY@tok@cm}{\let\PY@it=\textit\def\PY@tc##1{\textcolor[rgb]{0.25,0.50,0.50}{##1}}}
\@namedef{PY@tok@cpf}{\let\PY@it=\textit\def\PY@tc##1{\textcolor[rgb]{0.25,0.50,0.50}{##1}}}
\@namedef{PY@tok@c1}{\let\PY@it=\textit\def\PY@tc##1{\textcolor[rgb]{0.25,0.50,0.50}{##1}}}
\@namedef{PY@tok@cs}{\let\PY@it=\textit\def\PY@tc##1{\textcolor[rgb]{0.25,0.50,0.50}{##1}}}

\def\PYZbs{\char`\\}
\def\PYZus{\char`\_}
\def\PYZob{\char`\{}
\def\PYZcb{\char`\}}
\def\PYZca{\char`\^}
\def\PYZam{\char`\&}
\def\PYZlt{\char`\<}
\def\PYZgt{\char`\>}
\def\PYZsh{\char`\#}
\def\PYZpc{\char`\%}
\def\PYZdl{\char`\$}
\def\PYZhy{\char`\-}
\def\PYZsq{\char`\'}
\def\PYZdq{\char`\"}
\def\PYZti{\char`\~}
% for compatibility with earlier versions
\def\PYZat{@}
\def\PYZlb{[}
\def\PYZrb{]}
\makeatother


    % For linebreaks inside Verbatim environment from package fancyvrb. 
    \makeatletter
        \newbox\Wrappedcontinuationbox 
        \newbox\Wrappedvisiblespacebox 
        \newcommand*\Wrappedvisiblespace {\textcolor{red}{\textvisiblespace}} 
        \newcommand*\Wrappedcontinuationsymbol {\textcolor{red}{\llap{\tiny$\m@th\hookrightarrow$}}} 
        \newcommand*\Wrappedcontinuationindent {3ex } 
        \newcommand*\Wrappedafterbreak {\kern\Wrappedcontinuationindent\copy\Wrappedcontinuationbox} 
        % Take advantage of the already applied Pygments mark-up to insert 
        % potential linebreaks for TeX processing. 
        %        {, <, #, %, $, ' and ": go to next line. 
        %        _, }, ^, &, >, - and ~: stay at end of broken line. 
        % Use of \textquotesingle for straight quote. 
        \newcommand*\Wrappedbreaksatspecials {% 
            \def\PYGZus{\discretionary{\char`\_}{\Wrappedafterbreak}{\char`\_}}% 
            \def\PYGZob{\discretionary{}{\Wrappedafterbreak\char`\{}{\char`\{}}% 
            \def\PYGZcb{\discretionary{\char`\}}{\Wrappedafterbreak}{\char`\}}}% 
            \def\PYGZca{\discretionary{\char`\^}{\Wrappedafterbreak}{\char`\^}}% 
            \def\PYGZam{\discretionary{\char`\&}{\Wrappedafterbreak}{\char`\&}}% 
            \def\PYGZlt{\discretionary{}{\Wrappedafterbreak\char`\<}{\char`\<}}% 
            \def\PYGZgt{\discretionary{\char`\>}{\Wrappedafterbreak}{\char`\>}}% 
            \def\PYGZsh{\discretionary{}{\Wrappedafterbreak\char`\#}{\char`\#}}% 
            \def\PYGZpc{\discretionary{}{\Wrappedafterbreak\char`\%}{\char`\%}}% 
            \def\PYGZdl{\discretionary{}{\Wrappedafterbreak\char`\$}{\char`\$}}% 
            \def\PYGZhy{\discretionary{\char`\-}{\Wrappedafterbreak}{\char`\-}}% 
            \def\PYGZsq{\discretionary{}{\Wrappedafterbreak\textquotesingle}{\textquotesingle}}% 
            \def\PYGZdq{\discretionary{}{\Wrappedafterbreak\char`\"}{\char`\"}}% 
            \def\PYGZti{\discretionary{\char`\~}{\Wrappedafterbreak}{\char`\~}}% 
        } 
        % Some characters . , ; ? ! / are not pygmentized. 
        % This macro makes them "active" and they will insert potential linebreaks 
        \newcommand*\Wrappedbreaksatpunct {% 
            \lccode`\~`\.\lowercase{\def~}{\discretionary{\hbox{\char`\.}}{\Wrappedafterbreak}{\hbox{\char`\.}}}% 
            \lccode`\~`\,\lowercase{\def~}{\discretionary{\hbox{\char`\,}}{\Wrappedafterbreak}{\hbox{\char`\,}}}% 
            \lccode`\~`\;\lowercase{\def~}{\discretionary{\hbox{\char`\;}}{\Wrappedafterbreak}{\hbox{\char`\;}}}% 
            \lccode`\~`\:\lowercase{\def~}{\discretionary{\hbox{\char`\:}}{\Wrappedafterbreak}{\hbox{\char`\:}}}% 
            \lccode`\~`\?\lowercase{\def~}{\discretionary{\hbox{\char`\?}}{\Wrappedafterbreak}{\hbox{\char`\?}}}% 
            \lccode`\~`\!\lowercase{\def~}{\discretionary{\hbox{\char`\!}}{\Wrappedafterbreak}{\hbox{\char`\!}}}% 
            \lccode`\~`\/\lowercase{\def~}{\discretionary{\hbox{\char`\/}}{\Wrappedafterbreak}{\hbox{\char`\/}}}% 
            \catcode`\.\active
            \catcode`\,\active 
            \catcode`\;\active
            \catcode`\:\active
            \catcode`\?\active
            \catcode`\!\active
            \catcode`\/\active 
            \lccode`\~`\~ 	
        }
    \makeatother

    \let\OriginalVerbatim=\Verbatim
    \makeatletter
    \renewcommand{\Verbatim}[1][1]{%
        %\parskip\z@skip
        \sbox\Wrappedcontinuationbox {\Wrappedcontinuationsymbol}%
        \sbox\Wrappedvisiblespacebox {\FV@SetupFont\Wrappedvisiblespace}%
        \def\FancyVerbFormatLine ##1{\hsize\linewidth
            \vtop{\raggedright\hyphenpenalty\z@\exhyphenpenalty\z@
                \doublehyphendemerits\z@\finalhyphendemerits\z@
                \strut ##1\strut}%
        }%
        % If the linebreak is at a space, the latter will be displayed as visible
        % space at end of first line, and a continuation symbol starts next line.
        % Stretch/shrink are however usually zero for typewriter font.
        \def\FV@Space {%
            \nobreak\hskip\z@ plus\fontdimen3\font minus\fontdimen4\font
            \discretionary{\copy\Wrappedvisiblespacebox}{\Wrappedafterbreak}
            {\kern\fontdimen2\font}%
        }%
        
        % Allow breaks at special characters using \PYG... macros.
        \Wrappedbreaksatspecials
        % Breaks at punctuation characters . , ; ? ! and / need catcode=\active 	
        \OriginalVerbatim[#1,codes*=\Wrappedbreaksatpunct]%
    }
    \makeatother

    % Exact colors from NB
    \definecolor{incolor}{HTML}{303F9F}
    \definecolor{outcolor}{HTML}{D84315}
    \definecolor{cellborder}{HTML}{CFCFCF}
    \definecolor{cellbackground}{HTML}{F7F7F7}
    
    % prompt
    \makeatletter
    \newcommand{\boxspacing}{\kern\kvtcb@left@rule\kern\kvtcb@boxsep}
    \makeatother
    \newcommand{\prompt}[4]{
        {\ttfamily\llap{{\color{#2}[#3]:\hspace{3pt}#4}}\vspace{-\baselineskip}}
    }
    

    
    % Prevent overflowing lines due to hard-to-break entities
    \sloppy 
    % Setup hyperref package
    \hypersetup{
      breaklinks=true,  % so long urls are correctly broken across lines
      colorlinks=true,
      urlcolor=urlcolor,
      linkcolor=linkcolor,
      citecolor=citecolor,
      }
    % Slightly bigger margins than the latex defaults
    
    \geometry{verbose,tmargin=1in,bmargin=1in,lmargin=1in,rmargin=1in}
    
    

\begin{document}
    
    \maketitle
    
    

    
    \hypertarget{comprehensions}{%
\section{Comprehensions}\label{comprehensions}}

\begin{itemize}
\item
  Comprehensions are a concise and powerful way to create new mutable
  object (lists, sets, and dictionaries) from a string or any other
  iterable container. They allow you to create a new collection by
  iterating over an existing one and applying a transformation or
  filtering operation.
\item
  There exist three comprehensions: list comprehension, set
  comprehension, and dictionary comprehension. The syntax of the
  comprehension is given by:
\end{itemize}

\begin{Shaded}
\begin{Highlighting}[]
\CommentTok{\# List comprehension}
\NormalTok{[}\OperatorTok{\textless{}}\NormalTok{expr}\OperatorTok{\textgreater{}} \ControlFlowTok{for} \OperatorTok{\textless{}}\NormalTok{loop var}\OperatorTok{\textgreater{}} \KeywordTok{in} \OperatorTok{\textless{}}\NormalTok{iterable}\OperatorTok{\textgreater{}}\NormalTok{]}
\CommentTok{\# Set comprehension}
\NormalTok{\{}\OperatorTok{\textless{}}\NormalTok{expr}\OperatorTok{\textgreater{}} \ControlFlowTok{for} \OperatorTok{\textless{}}\NormalTok{loop var}\OperatorTok{\textgreater{}} \KeywordTok{in} \OperatorTok{\textless{}}\NormalTok{iterable}\OperatorTok{\textgreater{}}\NormalTok{\}}
\CommentTok{\# Dictionary comprehension}
\NormalTok{\{}\OperatorTok{\textless{}}\NormalTok{key}\OperatorTok{\textgreater{}}\NormalTok{: }\OperatorTok{\textless{}}\NormalTok{value}\OperatorTok{\textgreater{}} \ControlFlowTok{for} \OperatorTok{\textless{}}\NormalTok{loop var}\OperatorTok{\textgreater{}} \KeywordTok{in} \OperatorTok{\textless{}}\NormalTok{iterable}\OperatorTok{\textgreater{}}\NormalTok{\}}
\end{Highlighting}
\end{Shaded}

\hypertarget{for-loop-vs-comprehension}{%
\subsection{For-loop vs
comprehension:}\label{for-loop-vs-comprehension}}

A for-loop can be used to iterate over a sequence and apply a
transformation or filter operation. However, using a comprehension is
often more concise and easier to read. Here's an example of creating a
new list of squared numbers using a for-loop:

\begin{Shaded}
\begin{Highlighting}[]
\NormalTok{numbers }\OperatorTok{=}\NormalTok{ [}\DecValTok{1}\NormalTok{, }\DecValTok{2}\NormalTok{, }\DecValTok{3}\NormalTok{, }\DecValTok{4}\NormalTok{, }\DecValTok{5}\NormalTok{]}
\NormalTok{squares }\OperatorTok{=}\NormalTok{ []}
\ControlFlowTok{for}\NormalTok{ n }\KeywordTok{in}\NormalTok{ numbers:}
\NormalTok{    squares.append(n }\OperatorTok{**} \DecValTok{2}\NormalTok{)}
\end{Highlighting}
\end{Shaded}

The same thing can be achieved using a list comprehension:

\begin{Shaded}
\begin{Highlighting}[]
\NormalTok{numbers }\OperatorTok{=}\NormalTok{ [}\DecValTok{1}\NormalTok{, }\DecValTok{2}\NormalTok{, }\DecValTok{3}\NormalTok{, }\DecValTok{4}\NormalTok{, }\DecValTok{5}\NormalTok{]}
\NormalTok{squares }\OperatorTok{=}\NormalTok{ [n }\OperatorTok{**} \DecValTok{2} \ControlFlowTok{for}\NormalTok{ n }\KeywordTok{in}\NormalTok{ numbers]}
\end{Highlighting}
\end{Shaded}

\hypertarget{conditionals-in-list-comprehension}{%
\subsection{Conditionals in List
Comprehension:}\label{conditionals-in-list-comprehension}}

You can also add conditional statements to list comprehensions to filter
the items being transformed. Here's an example of creating a new list of
even numbers using a for-loop:

\begin{Shaded}
\begin{Highlighting}[]
\NormalTok{numbers }\OperatorTok{=}\NormalTok{ [}\DecValTok{1}\NormalTok{, }\DecValTok{2}\NormalTok{, }\DecValTok{3}\NormalTok{, }\DecValTok{4}\NormalTok{, }\DecValTok{5}\NormalTok{]}
\NormalTok{evens }\OperatorTok{=}\NormalTok{ []}
\ControlFlowTok{for}\NormalTok{ n }\KeywordTok{in}\NormalTok{ numbers:}
    \ControlFlowTok{if}\NormalTok{ n }\OperatorTok{\%} \DecValTok{2} \OperatorTok{==} \DecValTok{0}\NormalTok{:}
\NormalTok{        evens.append(n)}
\end{Highlighting}
\end{Shaded}

The same thing can be achieved using a list comprehension:

\begin{Shaded}
\begin{Highlighting}[]
\NormalTok{numbers }\OperatorTok{=}\NormalTok{ [}\DecValTok{1}\NormalTok{, }\DecValTok{2}\NormalTok{, }\DecValTok{3}\NormalTok{, }\DecValTok{4}\NormalTok{, }\DecValTok{5}\NormalTok{]}
\NormalTok{evens }\OperatorTok{=}\NormalTok{ [n }\ControlFlowTok{for}\NormalTok{ n }\KeywordTok{in}\NormalTok{ numbers }\ControlFlowTok{if}\NormalTok{ n }\OperatorTok{\%} \DecValTok{2} \OperatorTok{==} \DecValTok{0}\NormalTok{]}
\end{Highlighting}
\end{Shaded}

\hypertarget{nested-if-with-list-comprehension}{%
\subsection{Nested IF with List
Comprehension:}\label{nested-if-with-list-comprehension}}

You can also use nested if statements in list comprehensions to create
more complex filters. Here's an example of creating a new list of even
numbers that are greater than 2 using a for-loop:

\begin{Shaded}
\begin{Highlighting}[]
\NormalTok{numbers }\OperatorTok{=}\NormalTok{ [}\DecValTok{1}\NormalTok{, }\DecValTok{2}\NormalTok{, }\DecValTok{3}\NormalTok{, }\DecValTok{4}\NormalTok{, }\DecValTok{5}\NormalTok{]}
\NormalTok{evens\_greater\_than\_2 }\OperatorTok{=}\NormalTok{ []}
\ControlFlowTok{for}\NormalTok{ n }\KeywordTok{in}\NormalTok{ numbers:}
    \ControlFlowTok{if}\NormalTok{ n }\OperatorTok{\%} \DecValTok{2} \OperatorTok{==} \DecValTok{0}\NormalTok{:}
        \ControlFlowTok{if}\NormalTok{ n }\OperatorTok{\textgreater{}} \DecValTok{2}\NormalTok{:}
\NormalTok{            evens\_greater\_than\_2.append(n)}
\end{Highlighting}
\end{Shaded}

The same thing can be achieved using a list comprehension with nested if
statements:

\begin{Shaded}
\begin{Highlighting}[]
\NormalTok{numbers }\OperatorTok{=}\NormalTok{ [}\DecValTok{1}\NormalTok{, }\DecValTok{2}\NormalTok{, }\DecValTok{3}\NormalTok{, }\DecValTok{4}\NormalTok{, }\DecValTok{5}\NormalTok{]}
\NormalTok{evens\_greater\_than\_2 }\OperatorTok{=}\NormalTok{ [n }\ControlFlowTok{for}\NormalTok{ n }\KeywordTok{in}\NormalTok{ numbers }\ControlFlowTok{if}\NormalTok{ n }\OperatorTok{\%} \DecValTok{2} \OperatorTok{==} \DecValTok{0} \ControlFlowTok{if}\NormalTok{ n }\OperatorTok{\textgreater{}} \DecValTok{2}\NormalTok{]}
\end{Highlighting}
\end{Shaded}

\hypertarget{if-else-statement-with-list-comprehension}{%
\subsection{\texorpdfstring{\texttt{if-else} statement with List
Comprehension}{if-else statement with List Comprehension}}\label{if-else-statement-with-list-comprehension}}

If-else statements can also be used in list comprehensions to
conditionally include elements based on some condition. Here's an
example that creates a list using \texttt{if-else} statement:

\begin{Shaded}
\begin{Highlighting}[]
\NormalTok{squared\_even\_numbers }\OperatorTok{=}\NormalTok{ []}
\ControlFlowTok{for}\NormalTok{ num }\KeywordTok{in}\NormalTok{ numbers:}
    \ControlFlowTok{if}\NormalTok{ num }\OperatorTok{\%} \DecValTok{2} \OperatorTok{==} \DecValTok{0}\NormalTok{:}
\NormalTok{        squared\_even\_numbers.append(num }\OperatorTok{**} \DecValTok{2}\NormalTok{)}
    \ControlFlowTok{else}\NormalTok{:}
\NormalTok{        squared\_even\_numbers.append(num)}
\end{Highlighting}
\end{Shaded}

\hypertarget{how-if-we-have-nested-if-else-statement}{%
\paragraph{\texorpdfstring{How if we have Nested \texttt{if-else}
statement?}{How if we have Nested if-else statement?}}\label{how-if-we-have-nested-if-else-statement}}

\hypertarget{nested-loops-in-list-comprehension}{%
\subsubsection{Nested Loops in List
Comprehension}\label{nested-loops-in-list-comprehension}}

The nested loops list comprehension syntax is:

    \begin{tcolorbox}[breakable, size=fbox, boxrule=1pt, pad at break*=1mm,colback=cellbackground, colframe=cellborder]
\prompt{In}{incolor}{ }{\boxspacing}
\begin{Verbatim}[commandchars=\\\{\}]
\PY{n}{numbers}\PY{o}{=}\PY{p}{[}\PY{l+m+mi}{1}\PY{p}{,}\PY{l+m+mi}{2}\PY{p}{,}\PY{l+m+mi}{3}\PY{p}{,}\PY{l+m+mi}{4}\PY{p}{,}\PY{l+m+mi}{5}\PY{p}{,}\PY{l+m+mi}{6}\PY{p}{]}
\PY{n}{squared\PYZus{}even\PYZus{}numbers} \PY{o}{=} \PY{p}{[} \PY{n}{num}\PY{o}{*}\PY{o}{*}\PY{l+m+mi}{2} \PY{k}{if} \PY{n}{num}\PY{o}{\PYZpc{}}\PY{k}{2}==0 else num**3 if num\PYZpc{}3==0 else num for num in numbers]
\PY{n}{squared\PYZus{}even\PYZus{}numbers}
\end{Verbatim}
\end{tcolorbox}

    \hypertarget{functions}{%
\section{Functions}\label{functions}}

In our real life, to solve a big problem we have to break it into
smaller ones.

Programming languages kept the same principle; As our program grows
larger and larger, Functions help break it into smaller and modular
chunks.

\hypertarget{whats-a-function}{%
\subsection{What's a function?}\label{whats-a-function}}

A function is a block of organized, reusable code that is used to
perform a single, related action.

\hypertarget{why-functions}{%
\subsection{Why functions?}\label{why-functions}}

Functions make programs more organized and manageable. Furthermore, it
avoids repetition and makes the code reusable. \#\# Function syntax The
function syntax is:

\begin{Shaded}
\begin{Highlighting}[]
\KeywordTok{def}\NormalTok{ functionname( parameters ):}
    \CommentTok{"function\_docstring"}
\NormalTok{    function\_instructions}
    \ControlFlowTok{return}\NormalTok{ [expression]}
\end{Highlighting}
\end{Shaded}

\hypertarget{how-to-define-a-function}{%
\subsection{How to define a function?}\label{how-to-define-a-function}}

Here are simple rules to define a function in Python:

\begin{itemize}
\tightlist
\item
  Function blocks begin with the keyword \texttt{def} followed by the
  function name and parentheses ( ).
\item
  Any input parameters or arguments should be placed within these
  parentheses. You can also define parameters inside these parentheses.
\item
  The first statement of a function can be an optional statement - the
  documentation string of the function or docstring.
\item
  The code block within every function starts with a colon \texttt{:}
  and is indented.
\item
  The statement \texttt{return\ {[}expression{]}} exits a function,
  optionally passing back an expression to the caller. A return
  statement with no arguments is the same as return \texttt{None}.
\item
  By default, parameters have a positional behavior and you need to
  inform them in the same order that they were defined.
\end{itemize}

For example, the following function \texttt{printname} take a name as a
parameter and print ``Hi, \(<\)name\(>\), How was your day?''. It return
\texttt{None}

\begin{Shaded}
\begin{Highlighting}[]
    \KeywordTok{def}\NormalTok{ printname(name):}
        \CommentTok{"""This function says Hi"""}
        \BuiltInTok{print}\NormalTok{(}\StringTok{"Hi, "}\NormalTok{,name, }\StringTok{"How was your day?"}\NormalTok{)}
        \ControlFlowTok{return}
\end{Highlighting}
\end{Shaded}

\hypertarget{how-to-call-a-function}{%
\subsection{How to call a function}\label{how-to-call-a-function}}

Once the basic structure of a function is finalized, you can execute it
by calling it from another function or directly from the Python prompt.
Following is the example to call printname function

\begin{Shaded}
\begin{Highlighting}[]
\KeywordTok{def}\NormalTok{ printname(name):}
        \CommentTok{"""This function says Hi"""}
        \BuiltInTok{print}\NormalTok{(}\StringTok{"Hi, "}\NormalTok{,name, }\StringTok{"How was your day?"}\NormalTok{)}
        \ControlFlowTok{return}
\NormalTok{    printname(}\StringTok{"Ahmed"}\NormalTok{)}
\NormalTok{    printname(}\StringTok{"Ali  "}\NormalTok{)}
\end{Highlighting}
\end{Shaded}

\hypertarget{the-return-statement}{%
\subsection{The return statement}\label{the-return-statement}}

The return statement is used to exit a function and go back to the place
from where it was called.

This statement can contain an expression that gets evaluated and the
value is returned. If there is no expression in the statement or the
\texttt{return} statement itself is not present inside a function, then
the function will return the \texttt{None} object.

\begin{Shaded}
\begin{Highlighting}[]
    \BuiltInTok{print}\NormalTok{(printname(}\StringTok{"Ahmed"}\NormalTok{))}
\end{Highlighting}
\end{Shaded}

That will return \texttt{None}

\textbf{Exercise}

\begin{itemize}
\tightlist
\item
  Write a function that return the square of a number.
\item
  Write a function that return the absolute value of a number without
  using \texttt{abs()}
\end{itemize}

\hypertarget{scope-and-lifetime-of-variables}{%
\subsection{Scope and Lifetime of
variables}\label{scope-and-lifetime-of-variables}}

Scope of a variable is the portion of a program where the variable is
recognized. Parameters and variables defined inside a function are not
visible from outside the function. Hence, they have a local scope.

The lifetime of a variable is the period throughout which the variable
exists in the memory. The lifetime of variables inside a function is as
long as the function executes.

They are destroyed once we return from the function. Hence, a function
does not remember the value of a variable from its previous calls.

Here is an example to illustrate the scope of a variable inside a
function.

\begin{Shaded}
\begin{Highlighting}[]
\KeywordTok{def}\NormalTok{ mark():}
\NormalTok{    m }\OperatorTok{=} \DecValTok{10}
    \BuiltInTok{print}\NormalTok{(}\StringTok{"Value inside function:"}\NormalTok{,m)}

\NormalTok{m }\OperatorTok{=} \DecValTok{20}
\NormalTok{mark()}
\BuiltInTok{print}\NormalTok{(}\StringTok{"Value outside function:"}\NormalTok{,m)}
\end{Highlighting}
\end{Shaded}

\begin{itemize}
\item
  In the previous example, we can see that the value of m is 20
  initially. Even though the function mark() changed the value of m to
  10, it did not affect the value outside the function.
\item
  This is because the variable m inside the function is different (local
  to the function) from the one outside. Although they have the same
  names, they are two different variables with different scopes.
\item
  On the other hand, variables outside of the function are visible from
  inside. They have a global scope.
\item
  We can read these values from inside the function but cannot change
  (write) them. In order to modify the value of variables outside the
  function, they must be declared as global variables using the keyword
  global. \#\# Function Arguments
\end{itemize}

You can call a function by using the following types of formal
arguments:

\begin{itemize}
\tightlist
\item
  Required arguments
\item
  Keyword arguments
\item
  Default arguments
\item
  Variable-length arguments
\end{itemize}

\hypertarget{required-arguments}{%
\subsection{Required arguments}\label{required-arguments}}

Required arguments are the arguments passed to a function in correct
positional order. Here, the number of arguments in the function call
should match exactly with the function definition.

You definitely need to pass the same number of arguments as in
definition, otherwise it gives a syntax error.

\begin{Shaded}
\begin{Highlighting}[]
\KeywordTok{def}\NormalTok{ infos(name,age):}
        \BuiltInTok{print}\NormalTok{(}\StringTok{"my name is:"}\NormalTok{, name)}
        \BuiltInTok{print}\NormalTok{(}\StringTok{"I am"}\NormalTok{,age,}\StringTok{"years old"}\NormalTok{)}
\NormalTok{    infos(}\StringTok{"Ahmed"}\NormalTok{)}
\NormalTok{    infos(}\StringTok{"Ahmed"}\NormalTok{,}\DecValTok{20}\NormalTok{)}
\end{Highlighting}
\end{Shaded}

\hypertarget{keyword-arguments}{%
\subsection{Keyword arguments}\label{keyword-arguments}}

Keyword arguments are related to the function calls. When you use
keyword arguments in a function call, the caller identifies the
arguments by the parameter name.

This allows you to skip arguments or place them out of order because the
Python interpreter is able to use the keywords provided to match the
values with parameters.

\begin{Shaded}
\begin{Highlighting}[]
\KeywordTok{def}\NormalTok{ infos(name,age):}
        \BuiltInTok{print}\NormalTok{(}\StringTok{"my name is:"}\NormalTok{, name)}
        \BuiltInTok{print}\NormalTok{(}\StringTok{"I am"}\NormalTok{,age,}\StringTok{"years old"}\NormalTok{)}
\NormalTok{    infos(age}\OperatorTok{=}\DecValTok{20}\NormalTok{,name}\OperatorTok{=}\StringTok{"Ahmed"}\NormalTok{)}
\end{Highlighting}
\end{Shaded}

\hypertarget{default-arguments-optional}{%
\subsection{Default arguments
(optional)}\label{default-arguments-optional}}

A default argument is an argument that assumes a default value if a
value is not provided in the function call for that argument.

\begin{Shaded}
\begin{Highlighting}[]
\KeywordTok{def}\NormalTok{ infos(name,age}\OperatorTok{=}\DecValTok{19}\NormalTok{):}
        \BuiltInTok{print}\NormalTok{(}\StringTok{"my name is:"}\NormalTok{, name)}
        \BuiltInTok{print}\NormalTok{(}\StringTok{"I am"}\NormalTok{,age,}\StringTok{"years old"}\NormalTok{)}
\NormalTok{    infos(age}\OperatorTok{=}\DecValTok{20}\NormalTok{,name}\OperatorTok{=}\StringTok{"Ahmed"}\NormalTok{)}
\NormalTok{    infos(name}\OperatorTok{=}\StringTok{"Ahmed"}\NormalTok{)}
\end{Highlighting}
\end{Shaded}

\hypertarget{variable-length-arguments}{%
\subsection{Variable-length arguments}\label{variable-length-arguments}}

You may need to process a function for more arguments than you specified
while defining the function. These arguments are called variable-length
arguments and are not named in the function definition, unlike required
and default arguments.

Syntax for a function with non-keyword variable arguments is this

\begin{Shaded}
\begin{Highlighting}[]
\KeywordTok{def}\NormalTok{ functionname([formal\_args,] }\OperatorTok{*}\NormalTok{var\_args\_tuple ):}
    \CommentTok{"function\_docstring"}
\NormalTok{    function\_suite}
    \ControlFlowTok{return}\NormalTok{ [expression]}
\end{Highlighting}
\end{Shaded}

An asterisk \texttt{*} is placed before the variable name that holds the
values of all nonkeyword variable arguments. This tuple remains empty if
no additional arguments are specified during the function call.

\hypertarget{variable-length-arguments-1}{%
\subsection{Variable-length
arguments}\label{variable-length-arguments-1}}

\begin{Shaded}
\begin{Highlighting}[]
    \KeywordTok{def}\NormalTok{ infos(name,age,}\OperatorTok{*}\NormalTok{marks):}
        \BuiltInTok{print}\NormalTok{(}\StringTok{"my name is:"}\NormalTok{, name)}
        \BuiltInTok{print}\NormalTok{(}\StringTok{"I am"}\NormalTok{,age,}\StringTok{"years old"}\NormalTok{)}
        \ControlFlowTok{for}\NormalTok{ i }\KeywordTok{in} \BuiltInTok{range}\NormalTok{(}\BuiltInTok{len}\NormalTok{(marks)):}
            \BuiltInTok{print}\NormalTok{(}\StringTok{"my"}\NormalTok{,i}\OperatorTok{+}\DecValTok{1}\NormalTok{,}\StringTok{"th mark is"}\NormalTok{, marks[i])}
\NormalTok{    infos(}\StringTok{"Ahmed"}\NormalTok{,}\DecValTok{20}\NormalTok{)}
\NormalTok{    infos(}\StringTok{"Ahmed"}\NormalTok{,}\DecValTok{20}\NormalTok{,}\DecValTok{19}\NormalTok{,}\DecValTok{10}\NormalTok{,}\DecValTok{5}\NormalTok{)}
\NormalTok{    infos(}\StringTok{"Ahmed"}\NormalTok{,}\DecValTok{20}\NormalTok{,}\DecValTok{19}\NormalTok{,}\DecValTok{10}\NormalTok{,}\DecValTok{11}\NormalTok{,}\DecValTok{17}\NormalTok{,}\DecValTok{5}\NormalTok{)}
\end{Highlighting}
\end{Shaded}

\hypertarget{the-anonymous-functions-lambda-functions}{%
\subsection{The Anonymous Functions (Lambda
functions)}\label{the-anonymous-functions-lambda-functions}}

These functions are called anonymous because they are not declared in
the standard manner by using the def keyword. You can use the lambda
keyword to create small anonymous functions.

\begin{itemize}
\tightlist
\item
  Lambda forms can take any number of arguments but return just one
  value in the form of an expression. They cannot contain commands or
  multiple expressions.
\item
  An anonymous function cannot be a direct call to print because lambda
  requires an expression. \#\# Syntax
\end{itemize}

The syntax of lambda functions contains only a single statement, which
is as follows:

\begin{Shaded}
\begin{Highlighting}[]
\NormalTok{func\_name}\OperatorTok{=}\KeywordTok{lambda}\NormalTok{ [arg1 [,arg2,.....argn]]:expression}
\end{Highlighting}
\end{Shaded}

Following is the example to show how lambda form of function works:

\begin{Shaded}
\begin{Highlighting}[]
\NormalTok{special\_sum}\OperatorTok{=}\KeywordTok{lambda}\NormalTok{ a,b: }\DecValTok{2}\OperatorTok{*}\NormalTok{a}\OperatorTok{+}\DecValTok{3}\OperatorTok{*}\NormalTok{b}
    \BuiltInTok{print}\NormalTok{(special\_sum(}\DecValTok{1}\NormalTok{,}\DecValTok{1}\NormalTok{))}
    \BuiltInTok{print}\NormalTok{(special\_sum(}\DecValTok{2}\NormalTok{,}\DecValTok{9}\NormalTok{))}
\end{Highlighting}
\end{Shaded}

\hypertarget{recursive-functions}{%
\subsection{Recursive Functions}\label{recursive-functions}}

A recursive function is one that calls itself. \#\# Recursive
Definitions Every recursive function definition includes two parts:

\begin{itemize}
\tightlist
\item
  \textbf{Base case(s) (non-recursive):} One or more simple cases that
  can be done right away
\item
  \textbf{Recursive case(s):} One or more cases that require solving
  ``simpler'' version(s) of the original problem.
\end{itemize}

For example the factorial function can be defined recursively as

\begin{Shaded}
\begin{Highlighting}[]
\KeywordTok{def}\NormalTok{ fact(n):}
    \ControlFlowTok{if}\NormalTok{ n}\OperatorTok{==}\DecValTok{0}\NormalTok{:}
        \ControlFlowTok{return} \DecValTok{1}
    \ControlFlowTok{else}\NormalTok{:}
        \ControlFlowTok{return}\NormalTok{ n}\OperatorTok{*}\NormalTok{fact(n}\OperatorTok{{-}}\DecValTok{1}\NormalTok{)}
\end{Highlighting}
\end{Shaded}

\hypertarget{recursive-function-good-or-bad}{%
\subsection{Recursive function good or bad
?}\label{recursive-function-good-or-bad}}

\begin{itemize}
\tightlist
\item
  Simpler, more intuitive: For inductively defined computation,
  recursive algorithm may be natural and close to mathematical
  specification.
\item
  Easy from programming point of view.
\item
  May not efficient computation point of view.
\end{itemize}

\hypertarget{examples}{%
\subsection{Examples}\label{examples}}

\begin{itemize}
\tightlist
\item
  Write a recursive function that compute the product of \(a*b\).
\item
  Write a recursive function that compute the gcd of \(a\) and \(b\).
\item
  Write a recursive function that compute Fibonacci numbers.
\item
  Write a recursive function that compute \(x\) to the power \(n\);
  (\(x^n\)).
\end{itemize}


    % Add a bibliography block to the postdoc
    
    
    
\end{document}
