\begin{frame}[fragile]{Elementary programming}
    \tit{Comments}
    All modern programming languages have comment characters. These indicate part of
    the code that should be skipped by the interpreter.
    \tit{Why comments !}
    \begin{itemize}[<+->]
        \item Comments can be used to explain Python code.

        \item Comments can be used to make the code more readable.
        
        \item Comments can be used to prevent execution when testing code.
    \end{itemize}
    \pause
    \tit{How we write comments ?}
    \begin{block}{}
        Any characters after a sharp ({\bf \#}) on a line are skipped.
    \end{block}

\end{frame}
\begin{frame}[fragile]{Elementary programming}
\begin{lstlisting}[numbers=left,showstringspaces=false,language=python]
#This is a comment
#written in
#more than just one line
print("Hello, NHSM!")
#print("This line will be ignored ")
print("Did you understand comments ?") #This line will print "Did ...?" 
\end{lstlisting}
\pause
\tit{Python multiline comments}
Python doesn't support multiline comments.

However, you can use two triple quotes (""") as multiline comments.

Guido van Rossum, the creator of Python, also recommended this.
\pause
\begin{lstlisting}[numbers=left,showstringspaces=false,language=python]
    """ This is a comment
    written in
    more than just one line
    """
    print("Hello, NHSM!")
\end{lstlisting}
\end{frame}
\begin{frame}[fragile]{Elementary programming}
    \tit{Variables assignement}
\begin{block}{}
    Variables consist of two parts: the identifier (name)  and the value.
\end{block}


To assign a variable to a name, use a single equals sign ( = ).


A variable is created the moment you first assign a value to it.
\pause 
\begin{lstlisting}[numbers=left,showstringspaces=false,language=python]
Name = "Omar" #Create a variable named 'Name' with String value 'Omar'
Age = 5 #Create a variable named 'Age' with Integer value '25'
High = 1.73 #Create a variable named 'Age' with real value '1.73'
\end{lstlisting}
\pause 
Python allows you to assign a single or multiple values (not mandatory of the same type) to several variables simultaneously.
\begin{lstlisting}[numbers=left,showstringspaces=false,language=python]
a = b = c = 10 #assign a single 10 to a,b and c
Name,Age,High = "Omar",2,1.73 #assign 3 different values to 3 variables.
\end{lstlisting}
\end{frame}

\begin{frame}[fragile]{Elementary programming}
\tit{Type of variable}
You don’t do this in Python, however, because Python automatically figures out
the data type according to the value assigned to the variable.

\pause 
The most used types in Python are: 
\begin{lstlisting}[numbers=left,showstringspaces=false,language=python]
int #numerical type
float #numerical type
str #for textual manipulation
bool #logical type
complex #numerical type
\end{lstlisting}
You can determine the type of a variable or a literal
value by using the \lstinline{type()} function.
\begin{lstlisting}[numbers=left,showstringspaces=false,language=python]
>>> High = 1.73 
>>> type(High)
<class 'float'>
>>> type(43)
<class 'int'>
\end{lstlisting}
\end{frame}
\begin{frame}[fragile]{Elementary programming}
\tit{type conversion}
You can convert from one type to another with the \verb|int()|, \verb|float()|, and \verb|complex()| methods:
\begin{lstlisting}[numbers=left,showstringspaces=false,language=python]
x = 5    # int
y = 1.81  # float
z = 3+1j   # complex

#convert from int to float:
a = float(x)
#convert from float to int:
b = int(y)
#convert from int to complex:
c = complex(x)

print(a)
print(b)
print(c)

print(type(a))
print(type(b))
print(type(c))
\end{lstlisting}

\end{frame}
\begin{frame}[fragile]{Elementary programming}
    \tit{Dynamic types of variables}
Python is dynamically typed. This means that:
\begin{itemize}[<+->]
    \item Types are set on the variable values and not on the variable names.
    \item Variable types do not need to be known before the variables are used.
    \item Variable names can change types when their values are changed.
\end{itemize}
\onslide<4->
\begin{lstlisting}[numbers=left,showstringspaces=false,language=python]
a=3
a="yes"
a=1.73
\end{lstlisting}
\end{frame}
\begin{frame}[fragile]{Elementary programming}
\tit{Naming variables}
    \begin{itemize}[<+->]
        \item Variable names can contain only letters, numbers, and underscores \verb|(_)|.
         They can start with a letter or an underscore \verb|(_)|, not with a number.
        \item Variable names cannot contain spaces. To separate words in variables, you use underscores for example \verb|sorted_list|
        \item Variable names cannot the same as keywords, reserved words, and built-in functions in Python.
        \end{itemize}
        \onslide<4->{The following guidelines help you define good variable names:}
        \begin{itemize}[<+->]
            \item Variable names should be concise and descriptive. For example, the \verb|active_user| variable is more descriptive than the \verb|au|.
            \item Use underscores \verb|(_)| to separate multiple words in the variable names.
            \item Avoid using the letter \verb|l| and the uppercase letter \verb|O| because they look like the number \verb|1| and \verb|0|.
        \end{itemize}
\end{frame}
\begin{frame}[fragile]{Elementary programming}
    \tit{Operators \& Evaluating expressions}
    \begin{block}{}
        Operators are the syntax that Python uses to express common ways to manipulate
data and variables.
    \end{block}
    Python divides the operators in the following groups:
    \begin{center}{\bf Arithmetic operators}\end{center}
    Arithmetic operators are used with numeric values to perform common mathematical operations:
    \begin{center}
        \begin{tabular}{ |  p{3cm} | p{3cm} | p{3cm} |}
            \hline
            Operator &Name&Usage\\
            \hline
        \verb|+| & Addition & \verb|x + y|\\  
        \verb|-| & Subtraction & \verb|x - y|\\  
        \verb|*| & Multiplication & \verb|x * y|\\  
        \verb|/| & Float Division & \verb|x / y|\\  
        \verb|%| & Remainder & \verb|x % y|\\  
        \verb|**| & Exponentiation & \verb|x ** y|\\  
        \verb|//| & Integer division & \verb|x // y|\\
        \hline
        \end{tabular}
    \end{center}
\end{frame}
\begin{frame}[fragile]{Elementary programming}
    \tit{Operators \& Evaluating expressions}
    \begin{center}{\bf Assignment operators}\end{center}
    Assignment operators are used to assign values to variables:
    \begin{center}
        \begin{tabular}{ |  p{3cm} | p{4cm} | p{3cm} |}
            \hline
            Operator &Name&Usage\\
            \hline
            \verb|=|&\text{assignment}&\verb|x=3|\\
            \verb|+=|&\text{addition assignment}&\verb|x+=3|\\
            \verb|-=|&\text{subtraction assignment}&\verb|x-=3|\\
            \verb|*=|&\text{multiplication assignment}&\verb|x*=3|\\
            \verb|/=|&\text{float division assignment}&\verb|x/=3|\\
            \verb|//=|&\text{integer division assignment}&\verb|x//=3|\\
            \verb|%=|&\text{remainder assignment}&\verb|x%=3|\\
            \verb|**=|&\text{exponent assignment}&\verb|x**=3|\\
            \hline
        \end{tabular}
    \end{center}
\end{frame}
\begin{frame}[fragile]{Elementary programming}
    \tit{Operators \& Evaluating expressions}
    \begin{center}{\bf Relational operators}\end{center}
    Logical operators are used to combine conditional statements:

    \begin{center}
        \begin{tabular}{ |  p{3cm} | p{4cm} | p{3cm} |}
            \hline
            Operator &Name&Usage\\
            \hline
            \verb|<|&\text{less than}&\verb|x<y|\\
            \verb|<=|&\text{less than or equal to}&\verb|x<=y|\\
            \verb|>|&\text{greater than}&\verb|x>y|\\
            \verb|>=|&\text{greater than or equal to}&\verb|x>=y|\\
            \verb|==|&\text{equal to}&\verb|x==y|\\
            \verb|!=|&\text{not equal}&\verb|x!=y|\\
            \hline
        \end{tabular}
    \end{center}
\end{frame}

\begin{frame}[fragile]{Elementary programming}
    \tit{Operators \& Evaluating expressions}
    \begin{center}{\bf Logical operators}\end{center}

    Logical operators are used to combine conditional statements:

    \begin{center}
    \begin{tabular}{ |  p{2cm} | p{3cm} | p{4cm} |}
        \hline
        Operator &Name&Usage\\
        \hline
        \verb|and|&\text{logical conjunction}&\verb|x < 5 and  x < 10|\\
        \verb|or|&\text{logical disjunction}&\verb|x < 5 or x < 4|\\ 
        \verb|not|&\text{logical negation}&\verb|not(x < 5 and x < 10)|\\
        \hline
    \end{tabular}
\end{center}
\end{frame}
\begin{frame}[fragile]{Elementary programming}
\tit{Python Operators Precedence Rule}
\begin{itemize}[<+->]
    \item An expression is made with combinations of variables, values, operators and function calls.
    
    
    \item The Python interpreter evaluates the valid expression.
    
    
    \item Python uses a type of rule known as {\bf PEMDAS}.
\begin{enumerate}[<+->]
     \item {\bf P}:  Parentheses
    \item {\bf E}:  Exponentiation
    \item {\bf M}:  Multiplication
    \item {\bf D}:  Division
    \item {\bf A}:  Addition
    \item {\bf S}:  Subtraction
\end{enumerate}
\item Operators with the same precedence (except for \verb|**|) are evaluated from left-to-right.
\end{itemize}
\end{frame}
\begin{frame}[fragile]{Elementary programming}
    \tit{Input / output data}
    \begin{lstlisting}[numbers=left,showstringspaces=false,language=python]
        Age=eval(input("How old are you ?"))# input statement
        print(Age)#Output (display) statement
    \end{lstlisting}
\end{frame}
\begin{frame}[fragile]{Elementary programming}
    \tit{Types of Errors}
    \begin{center}{\bf Syntax Errors}\end{center}
  Syntax errors are the most basic type of error. They arise when the Python parser is unable to understand a line of code. Syntax errors are almost always fatal, i.e. there is almost never a way to successfully execute a piece of code containing syntax errors
\end{frame}
\begin{frame}[fragile]{Elementary programming}
  
\begin{center}{\bf Runtime Errors}\end{center}

 A program with a runtime error is one that passed the interpreter's syntax checks, and started to execute. However, during the execution of one of the statements in the program, an error occurred that caused the interpreter to stop executing the program and display an error message.  Runtime errors are also called exceptions because they usually indicate that something exceptional (and bad) has happened.



 Some examples of Python runtime errors:
 \begin{enumerate}[<+->]
   \item  division by zero 
   \item performing an operation on incompatible types 
   \item using an identifier which has not been defined 
   \item accessing a list element, dictionary value or object attribute which doesn't exist 
   \item trying to access a file which doesn't exist    
 \end{enumerate}

\end{frame}
\begin{frame}[fragile]{Elementary programming}
\begin{center}{\bf Logic Errors}\end{center}
These are the most difficult type of error to find, because they will give unpredictable results and may crash your program.  A lot of different things can happen if you have a logic error. However these are very easy to fix as you can use a debugger, which will run through the program and fix any problems.


Here are some examples of mistakes which lead to logical errors: 
\begin{enumerate}[<+->]
  \item using the wrong variable name
  \item indenting a block to the wrong level
  \item using integer division instead of floating point division
  \item getting operator precedence wrong
  \item  making a mistake in a boolean expression
  \item off-by-one, and other numerical errors
\end{enumerate}
\end{frame}
\begin{frame}[fragile]{Elementary programming}
    \tit{Documentation}
   There is online and offline versions of Python documentation 
   \begin{itemize}
     \item online: \url{https://www.python.org/doc/}
     \item offline: Zeal \url{https://zealdocs.org/}
   \end{itemize}
  \end{frame}
  \begin{frame}[fragile]{Elementary programming}
    \tit{How to find a solution for a problem}
  \begin{itemize}
    \item Write the Errors on Google as it is.
    \item Ask a question on a forum.
  \end{itemize}
  \end{frame}
% \begin{frame}[fragile]{Elementary programming}
% \end{frame}
% \begin{frame}[fragile]{Elementary programming}
% \end{frame}
% \begin{frame}[fragile]{Elementary programming}
% \end{frame}
% \begin{frame}[fragile]{Elementary programming}
% \end{frame}
% \begin{frame}[fragile]{Elementary programming}
% \end{frame}


